\documentclass[11pt]{article}

    \usepackage[breakable]{tcolorbox}
    \usepackage{parskip} % Stop auto-indenting (to mimic markdown behaviour)
    

    % Basic figure setup, for now with no caption control since it's done
    % automatically by Pandoc (which extracts ![](path) syntax from Markdown).
    \usepackage{graphicx}
    % Maintain compatibility with old templates. Remove in nbconvert 6.0
    \let\Oldincludegraphics\includegraphics
    % Ensure that by default, figures have no caption (until we provide a
    % proper Figure object with a Caption API and a way to capture that
    % in the conversion process - todo).
    \usepackage{caption}
    \DeclareCaptionFormat{nocaption}{}
    \captionsetup{format=nocaption,aboveskip=0pt,belowskip=0pt}

    \usepackage{float}
    \floatplacement{figure}{H} % forces figures to be placed at the correct location
    \usepackage{xcolor} % Allow colors to be defined
    \usepackage{enumerate} % Needed for markdown enumerations to work
    \usepackage{geometry} % Used to adjust the document margins
    \usepackage{amsmath} % Equations
    \usepackage{amssymb} % Equations
    \usepackage{textcomp} % defines textquotesingle
    % Hack from http://tex.stackexchange.com/a/47451/13684:
    \AtBeginDocument{%
        \def\PYZsq{\textquotesingle}% Upright quotes in Pygmentized code
    }
    \usepackage{upquote} % Upright quotes for verbatim code
    \usepackage{eurosym} % defines \euro

    \usepackage{iftex}
    \ifPDFTeX
        \usepackage[T1]{fontenc}
        \IfFileExists{alphabeta.sty}{
              \usepackage{alphabeta}
          }{
              \usepackage[mathletters]{ucs}
              \usepackage[utf8x]{inputenc}
          }
    \else
        \usepackage{fontspec}
        \usepackage{unicode-math}
    \fi

    \usepackage{fancyvrb} % verbatim replacement that allows latex
    \usepackage{grffile} % extends the file name processing of package graphics
                         % to support a larger range
    \makeatletter % fix for old versions of grffile with XeLaTeX
    \@ifpackagelater{grffile}{2019/11/01}
    {
      % Do nothing on new versions
    }
    {
      \def\Gread@@xetex#1{%
        \IfFileExists{"\Gin@base".bb}%
        {\Gread@eps{\Gin@base.bb}}%
        {\Gread@@xetex@aux#1}%
      }
    }
    \makeatother
    \usepackage[Export]{adjustbox} % Used to constrain images to a maximum size
    \adjustboxset{max size={0.9\linewidth}{0.9\paperheight}}

    % The hyperref package gives us a pdf with properly built
    % internal navigation ('pdf bookmarks' for the table of contents,
    % internal cross-reference links, web links for URLs, etc.)
    \usepackage{hyperref}
    % The default LaTeX title has an obnoxious amount of whitespace. By default,
    % titling removes some of it. It also provides customization options.
    \usepackage{titling}
    \usepackage{longtable} % longtable support required by pandoc >1.10
    \usepackage{booktabs}  % table support for pandoc > 1.12.2
    \usepackage{array}     % table support for pandoc >= 2.11.3
    \usepackage{calc}      % table minipage width calculation for pandoc >= 2.11.1
    \usepackage[inline]{enumitem} % IRkernel/repr support (it uses the enumerate* environment)
    \usepackage[normalem]{ulem} % ulem is needed to support strikethroughs (\sout)
                                % normalem makes italics be italics, not underlines
    \usepackage{mathrsfs}
    

    
    % Colors for the hyperref package
    \definecolor{urlcolor}{rgb}{0,.145,.698}
    \definecolor{linkcolor}{rgb}{.71,0.21,0.01}
    \definecolor{citecolor}{rgb}{.12,.54,.11}

    % ANSI colors
    \definecolor{ansi-black}{HTML}{3E424D}
    \definecolor{ansi-black-intense}{HTML}{282C36}
    \definecolor{ansi-red}{HTML}{E75C58}
    \definecolor{ansi-red-intense}{HTML}{B22B31}
    \definecolor{ansi-green}{HTML}{00A250}
    \definecolor{ansi-green-intense}{HTML}{007427}
    \definecolor{ansi-yellow}{HTML}{DDB62B}
    \definecolor{ansi-yellow-intense}{HTML}{B27D12}
    \definecolor{ansi-blue}{HTML}{208FFB}
    \definecolor{ansi-blue-intense}{HTML}{0065CA}
    \definecolor{ansi-magenta}{HTML}{D160C4}
    \definecolor{ansi-magenta-intense}{HTML}{A03196}
    \definecolor{ansi-cyan}{HTML}{60C6C8}
    \definecolor{ansi-cyan-intense}{HTML}{258F8F}
    \definecolor{ansi-white}{HTML}{C5C1B4}
    \definecolor{ansi-white-intense}{HTML}{A1A6B2}
    \definecolor{ansi-default-inverse-fg}{HTML}{FFFFFF}
    \definecolor{ansi-default-inverse-bg}{HTML}{000000}

    % common color for the border for error outputs.
    \definecolor{outerrorbackground}{HTML}{FFDFDF}

    % commands and environments needed by pandoc snippets
    % extracted from the output of `pandoc -s`
    \providecommand{\tightlist}{%
      \setlength{\itemsep}{0pt}\setlength{\parskip}{0pt}}
    \DefineVerbatimEnvironment{Highlighting}{Verbatim}{commandchars=\\\{\}}
    % Add ',fontsize=\small' for more characters per line
    \newenvironment{Shaded}{}{}
    \newcommand{\KeywordTok}[1]{\textcolor[rgb]{0.00,0.44,0.13}{\textbf{{#1}}}}
    \newcommand{\DataTypeTok}[1]{\textcolor[rgb]{0.56,0.13,0.00}{{#1}}}
    \newcommand{\DecValTok}[1]{\textcolor[rgb]{0.25,0.63,0.44}{{#1}}}
    \newcommand{\BaseNTok}[1]{\textcolor[rgb]{0.25,0.63,0.44}{{#1}}}
    \newcommand{\FloatTok}[1]{\textcolor[rgb]{0.25,0.63,0.44}{{#1}}}
    \newcommand{\CharTok}[1]{\textcolor[rgb]{0.25,0.44,0.63}{{#1}}}
    \newcommand{\StringTok}[1]{\textcolor[rgb]{0.25,0.44,0.63}{{#1}}}
    \newcommand{\CommentTok}[1]{\textcolor[rgb]{0.38,0.63,0.69}{\textit{{#1}}}}
    \newcommand{\OtherTok}[1]{\textcolor[rgb]{0.00,0.44,0.13}{{#1}}}
    \newcommand{\AlertTok}[1]{\textcolor[rgb]{1.00,0.00,0.00}{\textbf{{#1}}}}
    \newcommand{\FunctionTok}[1]{\textcolor[rgb]{0.02,0.16,0.49}{{#1}}}
    \newcommand{\RegionMarkerTok}[1]{{#1}}
    \newcommand{\ErrorTok}[1]{\textcolor[rgb]{1.00,0.00,0.00}{\textbf{{#1}}}}
    \newcommand{\NormalTok}[1]{{#1}}

    % Additional commands for more recent versions of Pandoc
    \newcommand{\ConstantTok}[1]{\textcolor[rgb]{0.53,0.00,0.00}{{#1}}}
    \newcommand{\SpecialCharTok}[1]{\textcolor[rgb]{0.25,0.44,0.63}{{#1}}}
    \newcommand{\VerbatimStringTok}[1]{\textcolor[rgb]{0.25,0.44,0.63}{{#1}}}
    \newcommand{\SpecialStringTok}[1]{\textcolor[rgb]{0.73,0.40,0.53}{{#1}}}
    \newcommand{\ImportTok}[1]{{#1}}
    \newcommand{\DocumentationTok}[1]{\textcolor[rgb]{0.73,0.13,0.13}{\textit{{#1}}}}
    \newcommand{\AnnotationTok}[1]{\textcolor[rgb]{0.38,0.63,0.69}{\textbf{\textit{{#1}}}}}
    \newcommand{\CommentVarTok}[1]{\textcolor[rgb]{0.38,0.63,0.69}{\textbf{\textit{{#1}}}}}
    \newcommand{\VariableTok}[1]{\textcolor[rgb]{0.10,0.09,0.49}{{#1}}}
    \newcommand{\ControlFlowTok}[1]{\textcolor[rgb]{0.00,0.44,0.13}{\textbf{{#1}}}}
    \newcommand{\OperatorTok}[1]{\textcolor[rgb]{0.40,0.40,0.40}{{#1}}}
    \newcommand{\BuiltInTok}[1]{{#1}}
    \newcommand{\ExtensionTok}[1]{{#1}}
    \newcommand{\PreprocessorTok}[1]{\textcolor[rgb]{0.74,0.48,0.00}{{#1}}}
    \newcommand{\AttributeTok}[1]{\textcolor[rgb]{0.49,0.56,0.16}{{#1}}}
    \newcommand{\InformationTok}[1]{\textcolor[rgb]{0.38,0.63,0.69}{\textbf{\textit{{#1}}}}}
    \newcommand{\WarningTok}[1]{\textcolor[rgb]{0.38,0.63,0.69}{\textbf{\textit{{#1}}}}}


    % Define a nice break command that doesn't care if a line doesn't already
    % exist.
    \def\br{\hspace*{\fill} \\* }
    % Math Jax compatibility definitions
    \def\gt{>}
    \def\lt{<}
    \let\Oldtex\TeX
    \let\Oldlatex\LaTeX
    \renewcommand{\TeX}{\textrm{\Oldtex}}
    \renewcommand{\LaTeX}{\textrm{\Oldlatex}}
    % Document parameters
    % Document title
    \title{main}
    
    
    
    
    
    
    
% Pygments definitions
\makeatletter
\def\PY@reset{\let\PY@it=\relax \let\PY@bf=\relax%
    \let\PY@ul=\relax \let\PY@tc=\relax%
    \let\PY@bc=\relax \let\PY@ff=\relax}
\def\PY@tok#1{\csname PY@tok@#1\endcsname}
\def\PY@toks#1+{\ifx\relax#1\empty\else%
    \PY@tok{#1}\expandafter\PY@toks\fi}
\def\PY@do#1{\PY@bc{\PY@tc{\PY@ul{%
    \PY@it{\PY@bf{\PY@ff{#1}}}}}}}
\def\PY#1#2{\PY@reset\PY@toks#1+\relax+\PY@do{#2}}

\@namedef{PY@tok@w}{\def\PY@tc##1{\textcolor[rgb]{0.73,0.73,0.73}{##1}}}
\@namedef{PY@tok@c}{\let\PY@it=\textit\def\PY@tc##1{\textcolor[rgb]{0.24,0.48,0.48}{##1}}}
\@namedef{PY@tok@cp}{\def\PY@tc##1{\textcolor[rgb]{0.61,0.40,0.00}{##1}}}
\@namedef{PY@tok@k}{\let\PY@bf=\textbf\def\PY@tc##1{\textcolor[rgb]{0.00,0.50,0.00}{##1}}}
\@namedef{PY@tok@kp}{\def\PY@tc##1{\textcolor[rgb]{0.00,0.50,0.00}{##1}}}
\@namedef{PY@tok@kt}{\def\PY@tc##1{\textcolor[rgb]{0.69,0.00,0.25}{##1}}}
\@namedef{PY@tok@o}{\def\PY@tc##1{\textcolor[rgb]{0.40,0.40,0.40}{##1}}}
\@namedef{PY@tok@ow}{\let\PY@bf=\textbf\def\PY@tc##1{\textcolor[rgb]{0.67,0.13,1.00}{##1}}}
\@namedef{PY@tok@nb}{\def\PY@tc##1{\textcolor[rgb]{0.00,0.50,0.00}{##1}}}
\@namedef{PY@tok@nf}{\def\PY@tc##1{\textcolor[rgb]{0.00,0.00,1.00}{##1}}}
\@namedef{PY@tok@nc}{\let\PY@bf=\textbf\def\PY@tc##1{\textcolor[rgb]{0.00,0.00,1.00}{##1}}}
\@namedef{PY@tok@nn}{\let\PY@bf=\textbf\def\PY@tc##1{\textcolor[rgb]{0.00,0.00,1.00}{##1}}}
\@namedef{PY@tok@ne}{\let\PY@bf=\textbf\def\PY@tc##1{\textcolor[rgb]{0.80,0.25,0.22}{##1}}}
\@namedef{PY@tok@nv}{\def\PY@tc##1{\textcolor[rgb]{0.10,0.09,0.49}{##1}}}
\@namedef{PY@tok@no}{\def\PY@tc##1{\textcolor[rgb]{0.53,0.00,0.00}{##1}}}
\@namedef{PY@tok@nl}{\def\PY@tc##1{\textcolor[rgb]{0.46,0.46,0.00}{##1}}}
\@namedef{PY@tok@ni}{\let\PY@bf=\textbf\def\PY@tc##1{\textcolor[rgb]{0.44,0.44,0.44}{##1}}}
\@namedef{PY@tok@na}{\def\PY@tc##1{\textcolor[rgb]{0.41,0.47,0.13}{##1}}}
\@namedef{PY@tok@nt}{\let\PY@bf=\textbf\def\PY@tc##1{\textcolor[rgb]{0.00,0.50,0.00}{##1}}}
\@namedef{PY@tok@nd}{\def\PY@tc##1{\textcolor[rgb]{0.67,0.13,1.00}{##1}}}
\@namedef{PY@tok@s}{\def\PY@tc##1{\textcolor[rgb]{0.73,0.13,0.13}{##1}}}
\@namedef{PY@tok@sd}{\let\PY@it=\textit\def\PY@tc##1{\textcolor[rgb]{0.73,0.13,0.13}{##1}}}
\@namedef{PY@tok@si}{\let\PY@bf=\textbf\def\PY@tc##1{\textcolor[rgb]{0.64,0.35,0.47}{##1}}}
\@namedef{PY@tok@se}{\let\PY@bf=\textbf\def\PY@tc##1{\textcolor[rgb]{0.67,0.36,0.12}{##1}}}
\@namedef{PY@tok@sr}{\def\PY@tc##1{\textcolor[rgb]{0.64,0.35,0.47}{##1}}}
\@namedef{PY@tok@ss}{\def\PY@tc##1{\textcolor[rgb]{0.10,0.09,0.49}{##1}}}
\@namedef{PY@tok@sx}{\def\PY@tc##1{\textcolor[rgb]{0.00,0.50,0.00}{##1}}}
\@namedef{PY@tok@m}{\def\PY@tc##1{\textcolor[rgb]{0.40,0.40,0.40}{##1}}}
\@namedef{PY@tok@gh}{\let\PY@bf=\textbf\def\PY@tc##1{\textcolor[rgb]{0.00,0.00,0.50}{##1}}}
\@namedef{PY@tok@gu}{\let\PY@bf=\textbf\def\PY@tc##1{\textcolor[rgb]{0.50,0.00,0.50}{##1}}}
\@namedef{PY@tok@gd}{\def\PY@tc##1{\textcolor[rgb]{0.63,0.00,0.00}{##1}}}
\@namedef{PY@tok@gi}{\def\PY@tc##1{\textcolor[rgb]{0.00,0.52,0.00}{##1}}}
\@namedef{PY@tok@gr}{\def\PY@tc##1{\textcolor[rgb]{0.89,0.00,0.00}{##1}}}
\@namedef{PY@tok@ge}{\let\PY@it=\textit}
\@namedef{PY@tok@gs}{\let\PY@bf=\textbf}
\@namedef{PY@tok@gp}{\let\PY@bf=\textbf\def\PY@tc##1{\textcolor[rgb]{0.00,0.00,0.50}{##1}}}
\@namedef{PY@tok@go}{\def\PY@tc##1{\textcolor[rgb]{0.44,0.44,0.44}{##1}}}
\@namedef{PY@tok@gt}{\def\PY@tc##1{\textcolor[rgb]{0.00,0.27,0.87}{##1}}}
\@namedef{PY@tok@err}{\def\PY@bc##1{{\setlength{\fboxsep}{\string -\fboxrule}\fcolorbox[rgb]{1.00,0.00,0.00}{1,1,1}{\strut ##1}}}}
\@namedef{PY@tok@kc}{\let\PY@bf=\textbf\def\PY@tc##1{\textcolor[rgb]{0.00,0.50,0.00}{##1}}}
\@namedef{PY@tok@kd}{\let\PY@bf=\textbf\def\PY@tc##1{\textcolor[rgb]{0.00,0.50,0.00}{##1}}}
\@namedef{PY@tok@kn}{\let\PY@bf=\textbf\def\PY@tc##1{\textcolor[rgb]{0.00,0.50,0.00}{##1}}}
\@namedef{PY@tok@kr}{\let\PY@bf=\textbf\def\PY@tc##1{\textcolor[rgb]{0.00,0.50,0.00}{##1}}}
\@namedef{PY@tok@bp}{\def\PY@tc##1{\textcolor[rgb]{0.00,0.50,0.00}{##1}}}
\@namedef{PY@tok@fm}{\def\PY@tc##1{\textcolor[rgb]{0.00,0.00,1.00}{##1}}}
\@namedef{PY@tok@vc}{\def\PY@tc##1{\textcolor[rgb]{0.10,0.09,0.49}{##1}}}
\@namedef{PY@tok@vg}{\def\PY@tc##1{\textcolor[rgb]{0.10,0.09,0.49}{##1}}}
\@namedef{PY@tok@vi}{\def\PY@tc##1{\textcolor[rgb]{0.10,0.09,0.49}{##1}}}
\@namedef{PY@tok@vm}{\def\PY@tc##1{\textcolor[rgb]{0.10,0.09,0.49}{##1}}}
\@namedef{PY@tok@sa}{\def\PY@tc##1{\textcolor[rgb]{0.73,0.13,0.13}{##1}}}
\@namedef{PY@tok@sb}{\def\PY@tc##1{\textcolor[rgb]{0.73,0.13,0.13}{##1}}}
\@namedef{PY@tok@sc}{\def\PY@tc##1{\textcolor[rgb]{0.73,0.13,0.13}{##1}}}
\@namedef{PY@tok@dl}{\def\PY@tc##1{\textcolor[rgb]{0.73,0.13,0.13}{##1}}}
\@namedef{PY@tok@s2}{\def\PY@tc##1{\textcolor[rgb]{0.73,0.13,0.13}{##1}}}
\@namedef{PY@tok@sh}{\def\PY@tc##1{\textcolor[rgb]{0.73,0.13,0.13}{##1}}}
\@namedef{PY@tok@s1}{\def\PY@tc##1{\textcolor[rgb]{0.73,0.13,0.13}{##1}}}
\@namedef{PY@tok@mb}{\def\PY@tc##1{\textcolor[rgb]{0.40,0.40,0.40}{##1}}}
\@namedef{PY@tok@mf}{\def\PY@tc##1{\textcolor[rgb]{0.40,0.40,0.40}{##1}}}
\@namedef{PY@tok@mh}{\def\PY@tc##1{\textcolor[rgb]{0.40,0.40,0.40}{##1}}}
\@namedef{PY@tok@mi}{\def\PY@tc##1{\textcolor[rgb]{0.40,0.40,0.40}{##1}}}
\@namedef{PY@tok@il}{\def\PY@tc##1{\textcolor[rgb]{0.40,0.40,0.40}{##1}}}
\@namedef{PY@tok@mo}{\def\PY@tc##1{\textcolor[rgb]{0.40,0.40,0.40}{##1}}}
\@namedef{PY@tok@ch}{\let\PY@it=\textit\def\PY@tc##1{\textcolor[rgb]{0.24,0.48,0.48}{##1}}}
\@namedef{PY@tok@cm}{\let\PY@it=\textit\def\PY@tc##1{\textcolor[rgb]{0.24,0.48,0.48}{##1}}}
\@namedef{PY@tok@cpf}{\let\PY@it=\textit\def\PY@tc##1{\textcolor[rgb]{0.24,0.48,0.48}{##1}}}
\@namedef{PY@tok@c1}{\let\PY@it=\textit\def\PY@tc##1{\textcolor[rgb]{0.24,0.48,0.48}{##1}}}
\@namedef{PY@tok@cs}{\let\PY@it=\textit\def\PY@tc##1{\textcolor[rgb]{0.24,0.48,0.48}{##1}}}

\def\PYZbs{\char`\\}
\def\PYZus{\char`\_}
\def\PYZob{\char`\{}
\def\PYZcb{\char`\}}
\def\PYZca{\char`\^}
\def\PYZam{\char`\&}
\def\PYZlt{\char`\<}
\def\PYZgt{\char`\>}
\def\PYZsh{\char`\#}
\def\PYZpc{\char`\%}
\def\PYZdl{\char`\$}
\def\PYZhy{\char`\-}
\def\PYZsq{\char`\'}
\def\PYZdq{\char`\"}
\def\PYZti{\char`\~}
% for compatibility with earlier versions
\def\PYZat{@}
\def\PYZlb{[}
\def\PYZrb{]}
\makeatother


    % For linebreaks inside Verbatim environment from package fancyvrb.
    \makeatletter
        \newbox\Wrappedcontinuationbox
        \newbox\Wrappedvisiblespacebox
        \newcommand*\Wrappedvisiblespace {\textcolor{red}{\textvisiblespace}}
        \newcommand*\Wrappedcontinuationsymbol {\textcolor{red}{\llap{\tiny$\m@th\hookrightarrow$}}}
        \newcommand*\Wrappedcontinuationindent {3ex }
        \newcommand*\Wrappedafterbreak {\kern\Wrappedcontinuationindent\copy\Wrappedcontinuationbox}
        % Take advantage of the already applied Pygments mark-up to insert
        % potential linebreaks for TeX processing.
        %        {, <, #, %, $, ' and ": go to next line.
        %        _, }, ^, &, >, - and ~: stay at end of broken line.
        % Use of \textquotesingle for straight quote.
        \newcommand*\Wrappedbreaksatspecials {%
            \def\PYGZus{\discretionary{\char`\_}{\Wrappedafterbreak}{\char`\_}}%
            \def\PYGZob{\discretionary{}{\Wrappedafterbreak\char`\{}{\char`\{}}%
            \def\PYGZcb{\discretionary{\char`\}}{\Wrappedafterbreak}{\char`\}}}%
            \def\PYGZca{\discretionary{\char`\^}{\Wrappedafterbreak}{\char`\^}}%
            \def\PYGZam{\discretionary{\char`\&}{\Wrappedafterbreak}{\char`\&}}%
            \def\PYGZlt{\discretionary{}{\Wrappedafterbreak\char`\<}{\char`\<}}%
            \def\PYGZgt{\discretionary{\char`\>}{\Wrappedafterbreak}{\char`\>}}%
            \def\PYGZsh{\discretionary{}{\Wrappedafterbreak\char`\#}{\char`\#}}%
            \def\PYGZpc{\discretionary{}{\Wrappedafterbreak\char`\%}{\char`\%}}%
            \def\PYGZdl{\discretionary{}{\Wrappedafterbreak\char`\$}{\char`\$}}%
            \def\PYGZhy{\discretionary{\char`\-}{\Wrappedafterbreak}{\char`\-}}%
            \def\PYGZsq{\discretionary{}{\Wrappedafterbreak\textquotesingle}{\textquotesingle}}%
            \def\PYGZdq{\discretionary{}{\Wrappedafterbreak\char`\"}{\char`\"}}%
            \def\PYGZti{\discretionary{\char`\~}{\Wrappedafterbreak}{\char`\~}}%
        }
        % Some characters . , ; ? ! / are not pygmentized.
        % This macro makes them "active" and they will insert potential linebreaks
        \newcommand*\Wrappedbreaksatpunct {%
            \lccode`\~`\.\lowercase{\def~}{\discretionary{\hbox{\char`\.}}{\Wrappedafterbreak}{\hbox{\char`\.}}}%
            \lccode`\~`\,\lowercase{\def~}{\discretionary{\hbox{\char`\,}}{\Wrappedafterbreak}{\hbox{\char`\,}}}%
            \lccode`\~`\;\lowercase{\def~}{\discretionary{\hbox{\char`\;}}{\Wrappedafterbreak}{\hbox{\char`\;}}}%
            \lccode`\~`\:\lowercase{\def~}{\discretionary{\hbox{\char`\:}}{\Wrappedafterbreak}{\hbox{\char`\:}}}%
            \lccode`\~`\?\lowercase{\def~}{\discretionary{\hbox{\char`\?}}{\Wrappedafterbreak}{\hbox{\char`\?}}}%
            \lccode`\~`\!\lowercase{\def~}{\discretionary{\hbox{\char`\!}}{\Wrappedafterbreak}{\hbox{\char`\!}}}%
            \lccode`\~`\/\lowercase{\def~}{\discretionary{\hbox{\char`\/}}{\Wrappedafterbreak}{\hbox{\char`\/}}}%
            \catcode`\.\active
            \catcode`\,\active
            \catcode`\;\active
            \catcode`\:\active
            \catcode`\?\active
            \catcode`\!\active
            \catcode`\/\active
            \lccode`\~`\~
        }
    \makeatother

    \let\OriginalVerbatim=\Verbatim
    \makeatletter
    \renewcommand{\Verbatim}[1][1]{%
        %\parskip\z@skip
        \sbox\Wrappedcontinuationbox {\Wrappedcontinuationsymbol}%
        \sbox\Wrappedvisiblespacebox {\FV@SetupFont\Wrappedvisiblespace}%
        \def\FancyVerbFormatLine ##1{\hsize\linewidth
            \vtop{\raggedright\hyphenpenalty\z@\exhyphenpenalty\z@
                \doublehyphendemerits\z@\finalhyphendemerits\z@
                \strut ##1\strut}%
        }%
        % If the linebreak is at a space, the latter will be displayed as visible
        % space at end of first line, and a continuation symbol starts next line.
        % Stretch/shrink are however usually zero for typewriter font.
        \def\FV@Space {%
            \nobreak\hskip\z@ plus\fontdimen3\font minus\fontdimen4\font
            \discretionary{\copy\Wrappedvisiblespacebox}{\Wrappedafterbreak}
            {\kern\fontdimen2\font}%
        }%

        % Allow breaks at special characters using \PYG... macros.
        \Wrappedbreaksatspecials
        % Breaks at punctuation characters . , ; ? ! and / need catcode=\active
        \OriginalVerbatim[#1,codes*=\Wrappedbreaksatpunct]%
    }
    \makeatother

    % Exact colors from NB
    \definecolor{incolor}{HTML}{303F9F}
    \definecolor{outcolor}{HTML}{D84315}
    \definecolor{cellborder}{HTML}{CFCFCF}
    \definecolor{cellbackground}{HTML}{F7F7F7}

    % prompt
    \makeatletter
    \newcommand{\boxspacing}{\kern\kvtcb@left@rule\kern\kvtcb@boxsep}
    \makeatother
    \newcommand{\prompt}[4]{
        {\ttfamily\llap{{\color{#2}[#3]:\hspace{3pt}#4}}\vspace{-\baselineskip}}
    }
    

    
    % Prevent overflowing lines due to hard-to-break entities
    \sloppy
    % Setup hyperref package
    \hypersetup{
      breaklinks=true,  % so long urls are correctly broken across lines
      colorlinks=true,
      urlcolor=urlcolor,
      linkcolor=linkcolor,
      citecolor=citecolor,
      }
    % Slightly bigger margins than the latex defaults
    
    \geometry{verbose,tmargin=1in,bmargin=1in,lmargin=1in,rmargin=1in}
    
    

\begin{document}
    
    \maketitle
    
    

    
    \hypertarget{symulacja-ciux105guxf3w-pseudolosowych-ich-filtracja-i-analiza}{%
\section{Symulacja ciągów pseudolosowych, ich filtracja i
analiza}\label{symulacja-ciux105guxf3w-pseudolosowych-ich-filtracja-i-analiza}}

\begin{enumerate}
\def\labelenumi{\arabic{enumi}.}
\tightlist
\item
  Symulować szum biały o rozkładzie normlanym N(5, 0.1).
\item
  Na podstawie otrzymanego ciągu obliczyć gęstość prawdopodobieństwa,
  dystrybuantę, a też wartość oczekiwaną, wariancję i funkcję
  kowariancyjną.
\item
  Przeprowadzić filtrację danych z wykorzystaniem filtru
  dolnoprzepustowego FIR (SOJ) o różnych parametrach.
\item
  Obliczyć gęstość prawdopodobieństwa, dystrybuantę, a też wartość
  oczekiwaną, wariancję i funkcję kowariancyjną sygnału wyjściowego.
  Porównać wyniki z p 2. Wyniki przedstawiać w postaci tablic oraz
  wykresów
\end{enumerate}

    \hypertarget{literatura}{%
\section{Literatura}\label{literatura}}

\begin{enumerate}
\def\labelenumi{\arabic{enumi}.}
\tightlist
\item
  Snopkowski R. Symulacja stochastyczna AGH, Kraków, 2007.
\item
  Niemiro W. Symulacje stochastyczne i metody Monte Carlo, Uniw.
  Warszawski, 2013.
\item
  Cacho K., Bily M., Bukowski J. Random processs, analysis and
  simulation, 1988
\item
  Othes R.K., Enochson Analiza numeryczna szeregów czasowych, WNT,
  Warszawa, 1988
\end{enumerate}

    \begin{center}\rule{0.5\linewidth}{0.5pt}\end{center}

    \hypertarget{importowanie-potrzebnych-bibliotek}{%
\subsection{Importowanie potrzebnych
bibliotek}\label{importowanie-potrzebnych-bibliotek}}

    \begin{tcolorbox}[breakable, size=fbox, boxrule=1pt, pad at break*=1mm,colback=cellbackground, colframe=cellborder]
\prompt{In}{incolor}{1}{\boxspacing}
\begin{Verbatim}[commandchars=\\\{\}]
\PY{k+kn}{import} \PY{n+nn}{numpy} \PY{k}{as} \PY{n+nn}{np}
\PY{k+kn}{import} \PY{n+nn}{matplotlib}\PY{n+nn}{.}\PY{n+nn}{pyplot} \PY{k}{as} \PY{n+nn}{plt}
\PY{k+kn}{from} \PY{n+nn}{scipy}\PY{n+nn}{.}\PY{n+nn}{stats} \PY{k+kn}{import} \PY{n}{norm}\PY{p}{,} \PY{n}{describe}
\PY{k+kn}{import} \PY{n+nn}{pandas}  \PY{k}{as} \PY{n+nn}{pd}
\end{Verbatim}
\end{tcolorbox}

    \hypertarget{symulacja-szumu-biaux142ego-o-rozkux142adzie-normlanym-n5-0.1.}{%
\subsection{1. Symulacja szumu białego o rozkładzie normlanym N(5,
0.1).}\label{symulacja-szumu-biaux142ego-o-rozkux142adzie-normlanym-n5-0.1.}}

\hypertarget{wytworzenie-szumu-biaux142ego-o-rozkux142adzie-gassowskim}{%
\subsubsection{Wytworzenie szumu białego o rozkładzie
Gassowskim}\label{wytworzenie-szumu-biaux142ego-o-rozkux142adzie-gassowskim}}

    \begin{tcolorbox}[breakable, size=fbox, boxrule=1pt, pad at break*=1mm,colback=cellbackground, colframe=cellborder]
\prompt{In}{incolor}{2}{\boxspacing}
\begin{Verbatim}[commandchars=\\\{\}]
\PY{c+c1}{\PYZsh{}Default parameters to describe white noise}
\PY{n}{mu} \PY{o}{=} \PY{l+m+mi}{5}
\PY{n}{sigma} \PY{o}{=} \PY{l+m+mf}{0.1}
\PY{n}{k\PYZus{}1} \PY{o}{=} \PY{n+nb}{pow}\PY{p}{(}\PY{l+m+mi}{10}\PY{p}{,}\PY{l+m+mi}{3}\PY{p}{)}     \PY{c+c1}{\PYZsh{}Amount of samples to generate white noise}
\PY{n}{k\PYZus{}2} \PY{o}{=} \PY{n+nb}{pow}\PY{p}{(}\PY{l+m+mi}{10}\PY{p}{,}\PY{l+m+mi}{4}\PY{p}{)}
\PY{n}{k\PYZus{}3} \PY{o}{=} \PY{n+nb}{pow}\PY{p}{(}\PY{l+m+mi}{10}\PY{p}{,}\PY{l+m+mi}{5}\PY{p}{)}
\PY{n}{k\PYZus{}4} \PY{o}{=} \PY{n+nb}{pow}\PY{p}{(}\PY{l+m+mi}{10}\PY{p}{,}\PY{l+m+mi}{6}\PY{p}{)}
\PY{n}{k\PYZus{}5} \PY{o}{=} \PY{n+nb}{pow}\PY{p}{(}\PY{l+m+mi}{10}\PY{p}{,}\PY{l+m+mi}{7}\PY{p}{)}

\PY{n}{samples\PYZus{}k\PYZus{}1} \PY{o}{=} \PY{n}{np}\PY{o}{.}\PY{n}{random}\PY{o}{.}\PY{n}{normal}\PY{p}{(}\PY{n}{mu}\PY{p}{,} \PY{n}{sigma}\PY{p}{,} \PY{n}{size}\PY{o}{=}\PY{n}{k\PYZus{}1}\PY{p}{)}
\PY{n}{samples\PYZus{}k\PYZus{}2} \PY{o}{=} \PY{n}{np}\PY{o}{.}\PY{n}{random}\PY{o}{.}\PY{n}{normal}\PY{p}{(}\PY{n}{mu}\PY{p}{,} \PY{n}{sigma}\PY{p}{,} \PY{n}{size}\PY{o}{=}\PY{n}{k\PYZus{}2}\PY{p}{)}
\PY{n}{samples\PYZus{}k\PYZus{}3} \PY{o}{=} \PY{n}{np}\PY{o}{.}\PY{n}{random}\PY{o}{.}\PY{n}{normal}\PY{p}{(}\PY{n}{mu}\PY{p}{,} \PY{n}{sigma}\PY{p}{,} \PY{n}{size}\PY{o}{=}\PY{n}{k\PYZus{}3}\PY{p}{)}
\PY{n}{samples\PYZus{}k\PYZus{}4} \PY{o}{=} \PY{n}{np}\PY{o}{.}\PY{n}{random}\PY{o}{.}\PY{n}{normal}\PY{p}{(}\PY{n}{mu}\PY{p}{,} \PY{n}{sigma}\PY{p}{,} \PY{n}{size}\PY{o}{=}\PY{n}{k\PYZus{}4}\PY{p}{)}
\PY{n}{samples\PYZus{}k\PYZus{}5} \PY{o}{=} \PY{n}{np}\PY{o}{.}\PY{n}{random}\PY{o}{.}\PY{n}{normal}\PY{p}{(}\PY{n}{mu}\PY{p}{,} \PY{n}{sigma}\PY{p}{,} \PY{n}{size}\PY{o}{=}\PY{n}{k\PYZus{}5}\PY{p}{)}

\PY{n}{fig}\PY{p}{,} \PY{n}{axarr} \PY{o}{=} \PY{n}{plt}\PY{o}{.}\PY{n}{subplots}\PY{p}{(}\PY{l+m+mi}{3}\PY{p}{,} \PY{l+m+mi}{2}\PY{p}{)}
\PY{n}{fig}\PY{o}{.}\PY{n}{set\PYZus{}figheight}\PY{p}{(}\PY{l+m+mi}{12}\PY{p}{)}
\PY{n}{fig}\PY{o}{.}\PY{n}{set\PYZus{}figwidth}\PY{p}{(}\PY{l+m+mi}{18}\PY{p}{)}
\PY{n}{fig}\PY{o}{.}\PY{n}{suptitle}\PY{p}{(}\PY{l+s+s2}{\PYZdq{}}\PY{l+s+s2}{Szum biały o rozkładzie Gaussowskim z różnym parametrem k}\PY{l+s+s2}{\PYZdq{}}\PY{p}{,} \PY{n}{fontsize}\PY{o}{=}\PY{l+m+mi}{16}\PY{p}{)}

\PY{n}{axarr}\PY{p}{[}\PY{l+m+mi}{0}\PY{p}{,} \PY{l+m+mi}{0}\PY{p}{]}\PY{o}{.}\PY{n}{plot}\PY{p}{(}\PY{n}{samples\PYZus{}k\PYZus{}1}\PY{p}{)}
\PY{n}{axarr}\PY{p}{[}\PY{l+m+mi}{0}\PY{p}{,} \PY{l+m+mi}{0}\PY{p}{]}\PY{o}{.}\PY{n}{set\PYZus{}title}\PY{p}{(}\PY{l+s+s1}{\PYZsq{}}\PY{l+s+s1}{k=10\PYZca{}(3)}\PY{l+s+s1}{\PYZsq{}}\PY{p}{)}
\PY{n}{axarr}\PY{p}{[}\PY{l+m+mi}{0}\PY{p}{,} \PY{l+m+mi}{1}\PY{p}{]}\PY{o}{.}\PY{n}{plot}\PY{p}{(}\PY{n}{samples\PYZus{}k\PYZus{}2}\PY{p}{)}
\PY{n}{axarr}\PY{p}{[}\PY{l+m+mi}{0}\PY{p}{,} \PY{l+m+mi}{1}\PY{p}{]}\PY{o}{.}\PY{n}{set\PYZus{}title}\PY{p}{(}\PY{l+s+s1}{\PYZsq{}}\PY{l+s+s1}{k=10\PYZca{}(4)}\PY{l+s+s1}{\PYZsq{}}\PY{p}{)}
\PY{n}{axarr}\PY{p}{[}\PY{l+m+mi}{1}\PY{p}{,} \PY{l+m+mi}{0}\PY{p}{]}\PY{o}{.}\PY{n}{plot}\PY{p}{(}\PY{n}{samples\PYZus{}k\PYZus{}3}\PY{p}{)}
\PY{n}{axarr}\PY{p}{[}\PY{l+m+mi}{1}\PY{p}{,} \PY{l+m+mi}{0}\PY{p}{]}\PY{o}{.}\PY{n}{set\PYZus{}title}\PY{p}{(}\PY{l+s+s1}{\PYZsq{}}\PY{l+s+s1}{k=10\PYZca{}(5)}\PY{l+s+s1}{\PYZsq{}}\PY{p}{)}
\PY{n}{axarr}\PY{p}{[}\PY{l+m+mi}{1}\PY{p}{,} \PY{l+m+mi}{1}\PY{p}{]}\PY{o}{.}\PY{n}{plot}\PY{p}{(}\PY{n}{samples\PYZus{}k\PYZus{}4}\PY{p}{)}
\PY{n}{axarr}\PY{p}{[}\PY{l+m+mi}{1}\PY{p}{,} \PY{l+m+mi}{1}\PY{p}{]}\PY{o}{.}\PY{n}{set\PYZus{}title}\PY{p}{(}\PY{l+s+s1}{\PYZsq{}}\PY{l+s+s1}{k=10\PYZca{}(6)}\PY{l+s+s1}{\PYZsq{}}\PY{p}{)}
\PY{n}{axarr}\PY{p}{[}\PY{l+m+mi}{2}\PY{p}{,} \PY{l+m+mi}{0}\PY{p}{]}\PY{o}{.}\PY{n}{plot}\PY{p}{(}\PY{n}{samples\PYZus{}k\PYZus{}5}\PY{p}{)}
\PY{n}{axarr}\PY{p}{[}\PY{l+m+mi}{2}\PY{p}{,} \PY{l+m+mi}{0}\PY{p}{]}\PY{o}{.}\PY{n}{set\PYZus{}title}\PY{p}{(}\PY{l+s+s1}{\PYZsq{}}\PY{l+s+s1}{k=10\PYZca{}(7)}\PY{l+s+s1}{\PYZsq{}}\PY{p}{)}
\PY{n}{axarr}\PY{p}{[}\PY{l+m+mi}{2}\PY{p}{,} \PY{l+m+mi}{1}\PY{p}{]}\PY{o}{.}\PY{n}{set\PYZus{}visible}\PY{p}{(}\PY{k+kc}{False}\PY{p}{)}

\PY{c+c1}{\PYZsh{} Tight layout often produces nice results}
\PY{c+c1}{\PYZsh{} but requires the title to be spaced accordingly}
\PY{n}{fig}\PY{o}{.}\PY{n}{tight\PYZus{}layout}\PY{p}{(}\PY{p}{)}
\PY{n}{fig}\PY{o}{.}\PY{n}{subplots\PYZus{}adjust}\PY{p}{(}\PY{n}{top}\PY{o}{=}\PY{l+m+mf}{0.92}\PY{p}{)}

\PY{n}{plt}\PY{o}{.}\PY{n}{show}\PY{p}{(}\PY{p}{)}
\end{Verbatim}
\end{tcolorbox}

    \begin{center}
    \adjustimage{max size={0.9\linewidth}{0.9\paperheight}}{main_files/main_6_0.png}
    \end{center}
    { \hspace*{\fill} \\}
    
    Powyższy wygenerowany diagram prezentuje szum biały składającego się z
określonej liczby próbek zadeklarowanej w zmiennej samples\_k\{i\}. Szum
biały jest rodzajem szumu akustycznego, który posiada całkowicie płaskie
widmo. W procesie stochastycznym szum biały to ciąg nieskorelowanych
zmiennych losowych o zerowej wartości oczekiwanej i stałej
wariancji(czyli biały szum to proces kowariancyjnie stacjonarny) oraz w
sensie ścisłym to biały szum w którym nieskorelowanie wzmianiamy do
niezależności. Biały szum jest tak zwaną ,,cegiełką'' podczas
konstrukcji procesów stochastycznych.

    \hypertarget{wyux15bwietlenie-histogramu-szumu-biaux142ego-o-charakterze-gaussowskim-dla-ruxf3znych-k}{%
\subsubsection{Wyświetlenie histogramu szumu białego o charakterze
Gaussowskim dla róznych
k}\label{wyux15bwietlenie-histogramu-szumu-biaux142ego-o-charakterze-gaussowskim-dla-ruxf3znych-k}}

    \begin{tcolorbox}[breakable, size=fbox, boxrule=1pt, pad at break*=1mm,colback=cellbackground, colframe=cellborder]
\prompt{In}{incolor}{3}{\boxspacing}
\begin{Verbatim}[commandchars=\\\{\}]
\PY{k}{def} \PY{n+nf}{display\PYZus{}hist}\PY{p}{(}\PY{n}{samples}\PY{p}{,} \PY{n}{k}\PY{p}{)}\PY{p}{:}
    \PY{n}{delta\PYZus{}x\PYZus{}1}\PY{o}{=}\PY{l+m+mi}{1}
    \PY{n}{delta\PYZus{}x\PYZus{}2}\PY{o}{=}\PY{l+m+mf}{0.6}
    \PY{n}{delta\PYZus{}x\PYZus{}3}\PY{o}{=}\PY{l+m+mf}{0.3}
    \PY{n}{delta\PYZus{}x\PYZus{}4}\PY{o}{=}\PY{l+m+mf}{0.1}

    \PY{n}{fig}\PY{p}{,} \PY{n}{axarr} \PY{o}{=} \PY{n}{plt}\PY{o}{.}\PY{n}{subplots}\PY{p}{(}\PY{l+m+mi}{2}\PY{p}{,} \PY{l+m+mi}{2}\PY{p}{)}
    \PY{n}{fig}\PY{o}{.}\PY{n}{set\PYZus{}figheight}\PY{p}{(}\PY{l+m+mi}{12}\PY{p}{)}
    \PY{n}{fig}\PY{o}{.}\PY{n}{set\PYZus{}figwidth}\PY{p}{(}\PY{l+m+mi}{18}\PY{p}{)}
    \PY{n}{fig}\PY{o}{.}\PY{n}{suptitle}\PY{p}{(}\PY{l+s+s2}{\PYZdq{}}\PY{l+s+s2}{Histogram szumu białego z różną wartością Δx dla k=}\PY{l+s+si}{\PYZob{}\PYZcb{}}\PY{l+s+s2}{\PYZdq{}}\PY{o}{.}\PY{n}{format}\PY{p}{(}\PY{n}{k}\PY{p}{)}\PY{p}{,} \PY{n}{fontsize}\PY{o}{=}\PY{l+m+mi}{16}\PY{p}{)}

    \PY{n}{axarr}\PY{p}{[}\PY{l+m+mi}{0}\PY{p}{,} \PY{l+m+mi}{0}\PY{p}{]}\PY{o}{.}\PY{n}{hist}\PY{p}{(}\PY{n}{samples}\PY{p}{,} \PY{n}{bins}\PY{o}{=}\PY{n+nb}{int}\PY{p}{(}\PY{l+m+mi}{6}\PY{o}{/}\PY{n}{delta\PYZus{}x\PYZus{}1}\PY{p}{)}\PY{p}{)}
    \PY{n}{axarr}\PY{p}{[}\PY{l+m+mi}{0}\PY{p}{,} \PY{l+m+mi}{0}\PY{p}{]}\PY{o}{.}\PY{n}{set\PYZus{}title}\PY{p}{(}\PY{l+s+s1}{\PYZsq{}}\PY{l+s+s1}{Δx=1}\PY{l+s+s1}{\PYZsq{}}\PY{p}{)}
    \PY{n}{axarr}\PY{p}{[}\PY{l+m+mi}{0}\PY{p}{,} \PY{l+m+mi}{1}\PY{p}{]}\PY{o}{.}\PY{n}{hist}\PY{p}{(}\PY{n}{samples}\PY{p}{,} \PY{n}{bins}\PY{o}{=}\PY{n+nb}{int}\PY{p}{(}\PY{l+m+mi}{6}\PY{o}{/}\PY{n}{delta\PYZus{}x\PYZus{}2}\PY{p}{)}\PY{p}{)}
    \PY{n}{axarr}\PY{p}{[}\PY{l+m+mi}{0}\PY{p}{,} \PY{l+m+mi}{1}\PY{p}{]}\PY{o}{.}\PY{n}{set\PYZus{}title}\PY{p}{(}\PY{l+s+s1}{\PYZsq{}}\PY{l+s+s1}{Δx=0,6}\PY{l+s+s1}{\PYZsq{}}\PY{p}{)}
    \PY{n}{axarr}\PY{p}{[}\PY{l+m+mi}{1}\PY{p}{,} \PY{l+m+mi}{0}\PY{p}{]}\PY{o}{.}\PY{n}{hist}\PY{p}{(}\PY{n}{samples}\PY{p}{,} \PY{n}{bins}\PY{o}{=}\PY{n+nb}{int}\PY{p}{(}\PY{l+m+mi}{6}\PY{o}{/}\PY{n}{delta\PYZus{}x\PYZus{}3}\PY{p}{)}\PY{p}{)}
    \PY{n}{axarr}\PY{p}{[}\PY{l+m+mi}{1}\PY{p}{,} \PY{l+m+mi}{0}\PY{p}{]}\PY{o}{.}\PY{n}{set\PYZus{}title}\PY{p}{(}\PY{l+s+s1}{\PYZsq{}}\PY{l+s+s1}{Δx=0,3}\PY{l+s+s1}{\PYZsq{}}\PY{p}{)}
    \PY{n}{axarr}\PY{p}{[}\PY{l+m+mi}{1}\PY{p}{,} \PY{l+m+mi}{1}\PY{p}{]}\PY{o}{.}\PY{n}{hist}\PY{p}{(}\PY{n}{samples}\PY{p}{,} \PY{n}{bins}\PY{o}{=}\PY{n+nb}{int}\PY{p}{(}\PY{l+m+mi}{6}\PY{o}{/}\PY{n}{delta\PYZus{}x\PYZus{}4}\PY{p}{)}\PY{p}{)}
    \PY{n}{axarr}\PY{p}{[}\PY{l+m+mi}{1}\PY{p}{,} \PY{l+m+mi}{1}\PY{p}{]}\PY{o}{.}\PY{n}{set\PYZus{}title}\PY{p}{(}\PY{l+s+s1}{\PYZsq{}}\PY{l+s+s1}{Δx=0,1}\PY{l+s+s1}{\PYZsq{}}\PY{p}{)}

    \PY{c+c1}{\PYZsh{} Tight layout often produces nice results}
    \PY{c+c1}{\PYZsh{} but requires the title to be spaced accordingly}
    \PY{n}{fig}\PY{o}{.}\PY{n}{tight\PYZus{}layout}\PY{p}{(}\PY{p}{)}
    \PY{n}{fig}\PY{o}{.}\PY{n}{subplots\PYZus{}adjust}\PY{p}{(}\PY{n}{top}\PY{o}{=}\PY{l+m+mf}{0.92}\PY{p}{)}

    \PY{n}{plt}\PY{o}{.}\PY{n}{show}\PY{p}{(}\PY{p}{)}
\end{Verbatim}
\end{tcolorbox}

    \hypertarget{k103}{%
\paragraph{k=10\^{}(3)}\label{k103}}

    \begin{tcolorbox}[breakable, size=fbox, boxrule=1pt, pad at break*=1mm,colback=cellbackground, colframe=cellborder]
\prompt{In}{incolor}{4}{\boxspacing}
\begin{Verbatim}[commandchars=\\\{\}]
\PY{n}{display\PYZus{}hist}\PY{p}{(}\PY{n}{samples\PYZus{}k\PYZus{}1}\PY{p}{,} \PY{n}{k\PYZus{}1}\PY{p}{)}
\end{Verbatim}
\end{tcolorbox}

    \begin{center}
    \adjustimage{max size={0.9\linewidth}{0.9\paperheight}}{main_files/main_11_0.png}
    \end{center}
    { \hspace*{\fill} \\}
    
    \hypertarget{k104}{%
\paragraph{k=10\^{}(4)}\label{k104}}

    \begin{tcolorbox}[breakable, size=fbox, boxrule=1pt, pad at break*=1mm,colback=cellbackground, colframe=cellborder]
\prompt{In}{incolor}{5}{\boxspacing}
\begin{Verbatim}[commandchars=\\\{\}]
\PY{n}{display\PYZus{}hist}\PY{p}{(}\PY{n}{samples\PYZus{}k\PYZus{}2}\PY{p}{,} \PY{n}{k\PYZus{}2}\PY{p}{)}
\end{Verbatim}
\end{tcolorbox}

    \begin{center}
    \adjustimage{max size={0.9\linewidth}{0.9\paperheight}}{main_files/main_13_0.png}
    \end{center}
    { \hspace*{\fill} \\}
    
    \hypertarget{k105}{%
\paragraph{k=10\^{}(5)}\label{k105}}

    \begin{tcolorbox}[breakable, size=fbox, boxrule=1pt, pad at break*=1mm,colback=cellbackground, colframe=cellborder]
\prompt{In}{incolor}{6}{\boxspacing}
\begin{Verbatim}[commandchars=\\\{\}]
\PY{n}{display\PYZus{}hist}\PY{p}{(}\PY{n}{samples\PYZus{}k\PYZus{}3}\PY{p}{,} \PY{n}{k\PYZus{}3}\PY{p}{)}
\end{Verbatim}
\end{tcolorbox}

    \begin{center}
    \adjustimage{max size={0.9\linewidth}{0.9\paperheight}}{main_files/main_15_0.png}
    \end{center}
    { \hspace*{\fill} \\}
    
    \hypertarget{k106}{%
\paragraph{k=10\^{}(6)}\label{k106}}

    \begin{tcolorbox}[breakable, size=fbox, boxrule=1pt, pad at break*=1mm,colback=cellbackground, colframe=cellborder]
\prompt{In}{incolor}{7}{\boxspacing}
\begin{Verbatim}[commandchars=\\\{\}]
\PY{n}{display\PYZus{}hist}\PY{p}{(}\PY{n}{samples\PYZus{}k\PYZus{}4}\PY{p}{,} \PY{n}{k\PYZus{}4}\PY{p}{)}
\end{Verbatim}
\end{tcolorbox}

    \begin{center}
    \adjustimage{max size={0.9\linewidth}{0.9\paperheight}}{main_files/main_17_0.png}
    \end{center}
    { \hspace*{\fill} \\}
    
    \hypertarget{k107}{%
\paragraph{k=10\^{}(7)}\label{k107}}

    \begin{tcolorbox}[breakable, size=fbox, boxrule=1pt, pad at break*=1mm,colback=cellbackground, colframe=cellborder]
\prompt{In}{incolor}{8}{\boxspacing}
\begin{Verbatim}[commandchars=\\\{\}]
\PY{n}{display\PYZus{}hist}\PY{p}{(}\PY{n}{samples\PYZus{}k\PYZus{}5}\PY{p}{,} \PY{n}{k\PYZus{}5}\PY{p}{)}
\end{Verbatim}
\end{tcolorbox}

    \begin{center}
    \adjustimage{max size={0.9\linewidth}{0.9\paperheight}}{main_files/main_19_0.png}
    \end{center}
    { \hspace*{\fill} \\}
    
    \hypertarget{na-podstawie-otrzymanego-ciux105gu-obliczyux107-gux119stoux15bux107-prawdopodobieux144stwta-dystrybuantux119-a-teux17c-wartoux15bux107-oczekiwan-wariancjux119-i-funkcjux119-kowariancyjnux105.}{%
\subsection{2. Na podstawie otrzymanego ciągu obliczyć gęstość
prawdopodobieństwta, dystrybuantę, a też wartość oczekiwan, wariancję i
funkcję
kowariancyjną.}\label{na-podstawie-otrzymanego-ciux105gu-obliczyux107-gux119stoux15bux107-prawdopodobieux144stwta-dystrybuantux119-a-teux17c-wartoux15bux107-oczekiwan-wariancjux119-i-funkcjux119-kowariancyjnux105.}}

    \hypertarget{obliczenie-gux119stoux15bci-prawdopodobieux144stwa}{%
\subsubsection{Obliczenie gęstości
prawdopodobieństwa}\label{obliczenie-gux119stoux15bci-prawdopodobieux144stwa}}

Gęstość prawdopodobieństwa (ang. probability density function) to
funkcja, która opisuje rozkład prawdopodobieństwa zmiennej losowej X.
Gęstość prawdopodobieństwa może być używana do obliczenia
prawdopodobieństwa wystąpienia wartości zmiennej losowej w określonym
przedziale. W przeciwieństwie do dystrybuanty, gęstość
prawdopodobieństwa nie jest równa prawdopodobieństwu, lecz określa
szybkość zmian prawdopodobieństwa zmiennej losowej wokół danej wartości.

\[ f(x) = \frac{1}{\sqrt{2\pi\sigma^2}} e^{-\frac{(x-\mu)^2}{2\sigma^2}} \]

Uwaga aby wyświetlić gęstość prawdopodobieństwa należy posortować
najpierw tablicę. Wytłumaczenie w linku poniżej\\
https://stackoverflow.com/questions/71296986/how-to-draw-the-probability-density-function-pdf-plot-in-python

    \begin{tcolorbox}[breakable, size=fbox, boxrule=1pt, pad at break*=1mm,colback=cellbackground, colframe=cellborder]
\prompt{In}{incolor}{9}{\boxspacing}
\begin{Verbatim}[commandchars=\\\{\}]
\PY{k}{def} \PY{n+nf}{probability\PYZus{}pdf}\PY{p}{(}\PY{n}{samples}\PY{p}{)}\PY{p}{:}
    \PY{n}{sort\PYZus{}samples} \PY{o}{=} \PY{n}{np}\PY{o}{.}\PY{n}{sort}\PY{p}{(}\PY{n}{samples}\PY{p}{)}
    \PY{n}{probability\PYZus{}pdf} \PY{o}{=} \PY{n}{norm}\PY{o}{.}\PY{n}{pdf}\PY{p}{(}\PY{n}{sort\PYZus{}samples}\PY{p}{,} \PY{n}{mu}\PY{p}{,} \PY{n}{sigma}\PY{p}{)}
    \PY{k}{return} \PY{n}{probability\PYZus{}pdf}
\end{Verbatim}
\end{tcolorbox}

    \begin{tcolorbox}[breakable, size=fbox, boxrule=1pt, pad at break*=1mm,colback=cellbackground, colframe=cellborder]
\prompt{In}{incolor}{10}{\boxspacing}
\begin{Verbatim}[commandchars=\\\{\}]
\PY{n}{fig}\PY{p}{,} \PY{n}{axarr} \PY{o}{=} \PY{n}{plt}\PY{o}{.}\PY{n}{subplots}\PY{p}{(}\PY{l+m+mi}{2}\PY{p}{,} \PY{l+m+mi}{3}\PY{p}{)}
\PY{n}{fig}\PY{o}{.}\PY{n}{set\PYZus{}figheight}\PY{p}{(}\PY{l+m+mi}{12}\PY{p}{)}
\PY{n}{fig}\PY{o}{.}\PY{n}{set\PYZus{}figwidth}\PY{p}{(}\PY{l+m+mi}{20}\PY{p}{)}
\PY{n}{fig}\PY{o}{.}\PY{n}{suptitle}\PY{p}{(}\PY{l+s+s2}{\PYZdq{}}\PY{l+s+s2}{Gęstość prawdopodobieństwa dla różnego k}\PY{l+s+s2}{\PYZdq{}}\PY{p}{,} \PY{n}{fontsize}\PY{o}{=}\PY{l+m+mi}{16}\PY{p}{)}

\PY{n}{axarr}\PY{p}{[}\PY{l+m+mi}{0}\PY{p}{,} \PY{l+m+mi}{0}\PY{p}{]}\PY{o}{.}\PY{n}{plot}\PY{p}{(}\PY{n}{np}\PY{o}{.}\PY{n}{sort}\PY{p}{(}\PY{n}{samples\PYZus{}k\PYZus{}1}\PY{p}{)}\PY{p}{,} \PY{n}{probability\PYZus{}pdf}\PY{p}{(}\PY{n}{samples\PYZus{}k\PYZus{}1}\PY{p}{)}\PY{p}{)}
\PY{n}{axarr}\PY{p}{[}\PY{l+m+mi}{0}\PY{p}{,} \PY{l+m+mi}{0}\PY{p}{]}\PY{o}{.}\PY{n}{set\PYZus{}title}\PY{p}{(}\PY{l+s+s1}{\PYZsq{}}\PY{l+s+s1}{k=10\PYZca{}(3)}\PY{l+s+s1}{\PYZsq{}}\PY{p}{)}
\PY{n}{axarr}\PY{p}{[}\PY{l+m+mi}{0}\PY{p}{,} \PY{l+m+mi}{1}\PY{p}{]}\PY{o}{.}\PY{n}{plot}\PY{p}{(}\PY{n}{np}\PY{o}{.}\PY{n}{sort}\PY{p}{(}\PY{n}{samples\PYZus{}k\PYZus{}2}\PY{p}{)}\PY{p}{,} \PY{n}{probability\PYZus{}pdf}\PY{p}{(}\PY{n}{samples\PYZus{}k\PYZus{}2}\PY{p}{)}\PY{p}{)}
\PY{n}{axarr}\PY{p}{[}\PY{l+m+mi}{0}\PY{p}{,} \PY{l+m+mi}{1}\PY{p}{]}\PY{o}{.}\PY{n}{set\PYZus{}title}\PY{p}{(}\PY{l+s+s1}{\PYZsq{}}\PY{l+s+s1}{k=10\PYZca{}(4)}\PY{l+s+s1}{\PYZsq{}}\PY{p}{)}
\PY{n}{axarr}\PY{p}{[}\PY{l+m+mi}{0}\PY{p}{,} \PY{l+m+mi}{2}\PY{p}{]}\PY{o}{.}\PY{n}{plot}\PY{p}{(}\PY{n}{np}\PY{o}{.}\PY{n}{sort}\PY{p}{(}\PY{n}{samples\PYZus{}k\PYZus{}3}\PY{p}{)}\PY{p}{,} \PY{n}{probability\PYZus{}pdf}\PY{p}{(}\PY{n}{samples\PYZus{}k\PYZus{}3}\PY{p}{)}\PY{p}{)}
\PY{n}{axarr}\PY{p}{[}\PY{l+m+mi}{0}\PY{p}{,} \PY{l+m+mi}{2}\PY{p}{]}\PY{o}{.}\PY{n}{set\PYZus{}title}\PY{p}{(}\PY{l+s+s1}{\PYZsq{}}\PY{l+s+s1}{k=10\PYZca{}(5)}\PY{l+s+s1}{\PYZsq{}}\PY{p}{)}
\PY{n}{axarr}\PY{p}{[}\PY{l+m+mi}{1}\PY{p}{,} \PY{l+m+mi}{0}\PY{p}{]}\PY{o}{.}\PY{n}{plot}\PY{p}{(}\PY{n}{np}\PY{o}{.}\PY{n}{sort}\PY{p}{(}\PY{n}{samples\PYZus{}k\PYZus{}4}\PY{p}{)}\PY{p}{,} \PY{n}{probability\PYZus{}pdf}\PY{p}{(}\PY{n}{samples\PYZus{}k\PYZus{}4}\PY{p}{)}\PY{p}{)}
\PY{n}{axarr}\PY{p}{[}\PY{l+m+mi}{1}\PY{p}{,} \PY{l+m+mi}{0}\PY{p}{]}\PY{o}{.}\PY{n}{set\PYZus{}title}\PY{p}{(}\PY{l+s+s1}{\PYZsq{}}\PY{l+s+s1}{k=10\PYZca{}(6)}\PY{l+s+s1}{\PYZsq{}}\PY{p}{)}
\PY{n}{axarr}\PY{p}{[}\PY{l+m+mi}{1}\PY{p}{,} \PY{l+m+mi}{1}\PY{p}{]}\PY{o}{.}\PY{n}{plot}\PY{p}{(}\PY{n}{np}\PY{o}{.}\PY{n}{sort}\PY{p}{(}\PY{n}{samples\PYZus{}k\PYZus{}5}\PY{p}{)}\PY{p}{,} \PY{n}{probability\PYZus{}pdf}\PY{p}{(}\PY{n}{samples\PYZus{}k\PYZus{}5}\PY{p}{)}\PY{p}{)}
\PY{n}{axarr}\PY{p}{[}\PY{l+m+mi}{1}\PY{p}{,} \PY{l+m+mi}{1}\PY{p}{]}\PY{o}{.}\PY{n}{set\PYZus{}title}\PY{p}{(}\PY{l+s+s1}{\PYZsq{}}\PY{l+s+s1}{k=10\PYZca{}(7)}\PY{l+s+s1}{\PYZsq{}}\PY{p}{)}
\PY{n}{axarr}\PY{p}{[}\PY{l+m+mi}{1}\PY{p}{,} \PY{l+m+mi}{2}\PY{p}{]}\PY{o}{.}\PY{n}{set\PYZus{}visible}\PY{p}{(}\PY{k+kc}{False}\PY{p}{)}

\PY{c+c1}{\PYZsh{} Tight layout often produces nice results}
\PY{c+c1}{\PYZsh{} but requires the title to be spaced accordingly}
\PY{n}{fig}\PY{o}{.}\PY{n}{tight\PYZus{}layout}\PY{p}{(}\PY{p}{)}
\PY{n}{fig}\PY{o}{.}\PY{n}{subplots\PYZus{}adjust}\PY{p}{(}\PY{n}{top}\PY{o}{=}\PY{l+m+mf}{0.92}\PY{p}{)}

\PY{n}{plt}\PY{o}{.}\PY{n}{show}\PY{p}{(}\PY{p}{)}
\end{Verbatim}
\end{tcolorbox}

    \begin{center}
    \adjustimage{max size={0.9\linewidth}{0.9\paperheight}}{main_files/main_23_0.png}
    \end{center}
    { \hspace*{\fill} \\}
    
    \hypertarget{obliczenie-dystrybuanty}{%
\subsubsection{Obliczenie dystrybuanty}\label{obliczenie-dystrybuanty}}

Dystrybuanta - (ang. cumulative distribution function) to funkcja
matematyczna, która określa prawdopodobieństwo, że losowo wybrana
zmienna losowa X jest mniejsza lub równa danej wartości x, tzn. F(x) =
P(X ≤ x). Dystrybuanta może być użyta do określenia takich wartości jak
kwantyle (np. mediana) oraz do badania asymetrii i ogona rozkładu
zmiennej losowej.

\[ F(x) = \frac{1}{2}\left[1 + \operatorname{erf}\left(\frac{x-\mu}{\sigma\sqrt{2}}\right)\right] \]

Podobnie jak dla obliczenia gęstości prawdopodobieństwa najpierw
należało skorzystać z posortowanych wcześniej danych w tablicy samples
https://stackoverflow.com/questions/24788200/calculate-the-cumulative-distribution-function-cdf-in-python

    \begin{tcolorbox}[breakable, size=fbox, boxrule=1pt, pad at break*=1mm,colback=cellbackground, colframe=cellborder]
\prompt{In}{incolor}{11}{\boxspacing}
\begin{Verbatim}[commandchars=\\\{\}]
\PY{k}{def} \PY{n+nf}{norm\PYZus{}cdf}\PY{p}{(}\PY{n}{samples}\PY{p}{)}\PY{p}{:}
    \PY{n}{sort\PYZus{}samples} \PY{o}{=} \PY{n}{np}\PY{o}{.}\PY{n}{sort}\PY{p}{(}\PY{n}{samples}\PY{p}{)}
    \PY{n}{norm\PYZus{}cdf} \PY{o}{=} \PY{n}{norm}\PY{o}{.}\PY{n}{cdf}\PY{p}{(}\PY{n}{sort\PYZus{}samples}\PY{p}{,} \PY{n}{mu}\PY{p}{,} \PY{n}{sigma}\PY{p}{)}
    \PY{k}{return} \PY{n}{norm\PYZus{}cdf}
\end{Verbatim}
\end{tcolorbox}

    \begin{tcolorbox}[breakable, size=fbox, boxrule=1pt, pad at break*=1mm,colback=cellbackground, colframe=cellborder]
\prompt{In}{incolor}{12}{\boxspacing}
\begin{Verbatim}[commandchars=\\\{\}]
\PY{n}{fig}\PY{p}{,} \PY{n}{axarr} \PY{o}{=} \PY{n}{plt}\PY{o}{.}\PY{n}{subplots}\PY{p}{(}\PY{l+m+mi}{2}\PY{p}{,} \PY{l+m+mi}{3}\PY{p}{)}
\PY{n}{fig}\PY{o}{.}\PY{n}{set\PYZus{}figheight}\PY{p}{(}\PY{l+m+mi}{12}\PY{p}{)}
\PY{n}{fig}\PY{o}{.}\PY{n}{set\PYZus{}figwidth}\PY{p}{(}\PY{l+m+mi}{20}\PY{p}{)}
\PY{n}{fig}\PY{o}{.}\PY{n}{suptitle}\PY{p}{(}\PY{l+s+s2}{\PYZdq{}}\PY{l+s+s2}{Gęstość prawdopodobieństwa dla różnego k}\PY{l+s+s2}{\PYZdq{}}\PY{p}{,} \PY{n}{fontsize}\PY{o}{=}\PY{l+m+mi}{16}\PY{p}{)}

\PY{n}{axarr}\PY{p}{[}\PY{l+m+mi}{0}\PY{p}{,} \PY{l+m+mi}{0}\PY{p}{]}\PY{o}{.}\PY{n}{plot}\PY{p}{(}\PY{n}{np}\PY{o}{.}\PY{n}{sort}\PY{p}{(}\PY{n}{samples\PYZus{}k\PYZus{}1}\PY{p}{)}\PY{p}{,} \PY{n}{norm\PYZus{}cdf}\PY{p}{(}\PY{n}{samples\PYZus{}k\PYZus{}1}\PY{p}{)}\PY{p}{)}
\PY{n}{axarr}\PY{p}{[}\PY{l+m+mi}{0}\PY{p}{,} \PY{l+m+mi}{0}\PY{p}{]}\PY{o}{.}\PY{n}{set\PYZus{}title}\PY{p}{(}\PY{l+s+s1}{\PYZsq{}}\PY{l+s+s1}{k=10\PYZca{}(3)}\PY{l+s+s1}{\PYZsq{}}\PY{p}{)}
\PY{n}{axarr}\PY{p}{[}\PY{l+m+mi}{0}\PY{p}{,} \PY{l+m+mi}{1}\PY{p}{]}\PY{o}{.}\PY{n}{plot}\PY{p}{(}\PY{n}{np}\PY{o}{.}\PY{n}{sort}\PY{p}{(}\PY{n}{samples\PYZus{}k\PYZus{}2}\PY{p}{)}\PY{p}{,} \PY{n}{norm\PYZus{}cdf}\PY{p}{(}\PY{n}{samples\PYZus{}k\PYZus{}2}\PY{p}{)}\PY{p}{)}
\PY{n}{axarr}\PY{p}{[}\PY{l+m+mi}{0}\PY{p}{,} \PY{l+m+mi}{1}\PY{p}{]}\PY{o}{.}\PY{n}{set\PYZus{}title}\PY{p}{(}\PY{l+s+s1}{\PYZsq{}}\PY{l+s+s1}{k=10\PYZca{}(4)}\PY{l+s+s1}{\PYZsq{}}\PY{p}{)}
\PY{n}{axarr}\PY{p}{[}\PY{l+m+mi}{0}\PY{p}{,} \PY{l+m+mi}{2}\PY{p}{]}\PY{o}{.}\PY{n}{plot}\PY{p}{(}\PY{n}{np}\PY{o}{.}\PY{n}{sort}\PY{p}{(}\PY{n}{samples\PYZus{}k\PYZus{}3}\PY{p}{)}\PY{p}{,} \PY{n}{norm\PYZus{}cdf}\PY{p}{(}\PY{n}{samples\PYZus{}k\PYZus{}3}\PY{p}{)}\PY{p}{)}
\PY{n}{axarr}\PY{p}{[}\PY{l+m+mi}{0}\PY{p}{,} \PY{l+m+mi}{2}\PY{p}{]}\PY{o}{.}\PY{n}{set\PYZus{}title}\PY{p}{(}\PY{l+s+s1}{\PYZsq{}}\PY{l+s+s1}{k=10\PYZca{}(5)}\PY{l+s+s1}{\PYZsq{}}\PY{p}{)}
\PY{n}{axarr}\PY{p}{[}\PY{l+m+mi}{1}\PY{p}{,} \PY{l+m+mi}{0}\PY{p}{]}\PY{o}{.}\PY{n}{plot}\PY{p}{(}\PY{n}{np}\PY{o}{.}\PY{n}{sort}\PY{p}{(}\PY{n}{samples\PYZus{}k\PYZus{}4}\PY{p}{)}\PY{p}{,} \PY{n}{norm\PYZus{}cdf}\PY{p}{(}\PY{n}{samples\PYZus{}k\PYZus{}4}\PY{p}{)}\PY{p}{)}
\PY{n}{axarr}\PY{p}{[}\PY{l+m+mi}{1}\PY{p}{,} \PY{l+m+mi}{0}\PY{p}{]}\PY{o}{.}\PY{n}{set\PYZus{}title}\PY{p}{(}\PY{l+s+s1}{\PYZsq{}}\PY{l+s+s1}{k=10\PYZca{}(6)}\PY{l+s+s1}{\PYZsq{}}\PY{p}{)}
\PY{n}{axarr}\PY{p}{[}\PY{l+m+mi}{1}\PY{p}{,} \PY{l+m+mi}{1}\PY{p}{]}\PY{o}{.}\PY{n}{plot}\PY{p}{(}\PY{n}{np}\PY{o}{.}\PY{n}{sort}\PY{p}{(}\PY{n}{samples\PYZus{}k\PYZus{}5}\PY{p}{)}\PY{p}{,} \PY{n}{norm\PYZus{}cdf}\PY{p}{(}\PY{n}{samples\PYZus{}k\PYZus{}5}\PY{p}{)}\PY{p}{)}
\PY{n}{axarr}\PY{p}{[}\PY{l+m+mi}{1}\PY{p}{,} \PY{l+m+mi}{1}\PY{p}{]}\PY{o}{.}\PY{n}{set\PYZus{}title}\PY{p}{(}\PY{l+s+s1}{\PYZsq{}}\PY{l+s+s1}{k=10\PYZca{}(7)}\PY{l+s+s1}{\PYZsq{}}\PY{p}{)}
\PY{n}{axarr}\PY{p}{[}\PY{l+m+mi}{1}\PY{p}{,} \PY{l+m+mi}{2}\PY{p}{]}\PY{o}{.}\PY{n}{set\PYZus{}visible}\PY{p}{(}\PY{k+kc}{False}\PY{p}{)}

\PY{c+c1}{\PYZsh{} Tight layout often produces nice results}
\PY{c+c1}{\PYZsh{} but requires the title to be spaced accordingly}
\PY{n}{fig}\PY{o}{.}\PY{n}{tight\PYZus{}layout}\PY{p}{(}\PY{p}{)}
\PY{n}{fig}\PY{o}{.}\PY{n}{subplots\PYZus{}adjust}\PY{p}{(}\PY{n}{top}\PY{o}{=}\PY{l+m+mf}{0.92}\PY{p}{)}

\PY{n}{plt}\PY{o}{.}\PY{n}{show}\PY{p}{(}\PY{p}{)}
\end{Verbatim}
\end{tcolorbox}

    \begin{center}
    \adjustimage{max size={0.9\linewidth}{0.9\paperheight}}{main_files/main_26_0.png}
    \end{center}
    { \hspace*{\fill} \\}
    
    \hypertarget{obliczenie-wartoux15bci-oczekiwanej}{%
\subsubsection{Obliczenie wartości
oczekiwanej}\label{obliczenie-wartoux15bci-oczekiwanej}}

Wartość oczekiwana - (ang. expected value) to średnia wartość zmiennej
losowej X. Można ją wyznaczyć przez pomnożenie każdej wartości zmiennej
losowej przez jej prawdopodobieństwo i zsumowanie tych wartości. Wartość
oczekiwana jest jednym z najważniejszych parametrów opisujących zmienną
losową, ponieważ pozwala na określenie, jakiej wartości można oczekiwać,
gdy zmienna losowa zostanie wielokrotnie pomierzona.

\[ E(X) = \int_{-\infty}^{\infty} x f(x) dx \]

Wytłumaczone na filmiku jak w pythonie obliczać wartość oczekiwaną oraz
wariancję https://www.youtube.com/watch?v=ikcUBqELZVU

    \begin{tcolorbox}[breakable, size=fbox, boxrule=1pt, pad at break*=1mm,colback=cellbackground, colframe=cellborder]
\prompt{In}{incolor}{13}{\boxspacing}
\begin{Verbatim}[commandchars=\\\{\}]
\PY{c+c1}{\PYZsh{}Dla k=10\PYZca{}(3)}
\PY{n}{expected\PYZus{}value\PYZus{}k1} \PY{o}{=} \PY{n}{describe}\PY{p}{(}\PY{n}{samples\PYZus{}k\PYZus{}1}\PY{p}{)}\PY{o}{.}\PY{n}{mean}
\PY{c+c1}{\PYZsh{}Dla k=10\PYZca{}(4)}
\PY{n}{expected\PYZus{}value\PYZus{}k2} \PY{o}{=} \PY{n}{describe}\PY{p}{(}\PY{n}{samples\PYZus{}k\PYZus{}2}\PY{p}{)}\PY{o}{.}\PY{n}{mean}
\PY{c+c1}{\PYZsh{}Dla k=10\PYZca{}(5)}
\PY{n}{expected\PYZus{}value\PYZus{}k3} \PY{o}{=} \PY{n}{describe}\PY{p}{(}\PY{n}{samples\PYZus{}k\PYZus{}3}\PY{p}{)}\PY{o}{.}\PY{n}{mean}
\PY{c+c1}{\PYZsh{}Dla k=10\PYZca{}(6)}
\PY{n}{expected\PYZus{}value\PYZus{}k4} \PY{o}{=} \PY{n}{describe}\PY{p}{(}\PY{n}{samples\PYZus{}k\PYZus{}4}\PY{p}{)}\PY{o}{.}\PY{n}{mean}
\PY{c+c1}{\PYZsh{}Dla k=10\PYZca{}(7)}
\PY{n}{expected\PYZus{}value\PYZus{}k5} \PY{o}{=} \PY{n}{describe}\PY{p}{(}\PY{n}{samples\PYZus{}k\PYZus{}5}\PY{p}{)}\PY{o}{.}\PY{n}{mean}
\end{Verbatim}
\end{tcolorbox}

    \hypertarget{obliczenie-wariancji}{%
\subsubsection{Obliczenie wariancji}\label{obliczenie-wariancji}}

Wariancja - w procesach stochastycznych to miara zmienności losowej
zmiennej w czasie. Jest to średnia arytmetyczna kwadratów odchyleń
wartości losowej zmiennej od jej wartości oczekiwanej w ciągu
określonego czasu. Innymi słowy, wariancja procesu stochastycznego
mierzy, jak bardzo zmieniają się wartości zmiennej losowej w czasie, i
określa, jak bardzo trajektoria procesu różni się od średniej
trajektorii. Im większa wariancja, tym większa zmienność w trajektorii
procesu, a tym samym większa szansa na wystąpienie dużych odchyleń od
wartości oczekiwanej. Wariancja jest jednym z podstawowych parametrów
charakteryzujących proces stochastyczny i jest istotnym narzędziem w
analizie i modelowaniu procesów losowych. W praktyce, często używa się
także odchylenia standardowego, które jest pierwiastkiem kwadratowym z
wariancji i wyraża się w tych samych jednostkach co zmienna losowa.

\[ \operatorname{Var}(X) = E\left[(X - E(X))^2\right] = \int_{-\infty}^{\infty} (x - E(X))^2 f(x) dx \]

    \begin{tcolorbox}[breakable, size=fbox, boxrule=1pt, pad at break*=1mm,colback=cellbackground, colframe=cellborder]
\prompt{In}{incolor}{14}{\boxspacing}
\begin{Verbatim}[commandchars=\\\{\}]
\PY{c+c1}{\PYZsh{}Dla k=10\PYZca{}(3)}
\PY{n}{variance\PYZus{}k1} \PY{o}{=} \PY{n}{describe}\PY{p}{(}\PY{n}{samples\PYZus{}k\PYZus{}1}\PY{p}{)}\PY{o}{.}\PY{n}{variance}
\PY{c+c1}{\PYZsh{}Dla k=10\PYZca{}(4)}
\PY{n}{variance\PYZus{}k2} \PY{o}{=} \PY{n}{describe}\PY{p}{(}\PY{n}{samples\PYZus{}k\PYZus{}2}\PY{p}{)}\PY{o}{.}\PY{n}{variance}
\PY{c+c1}{\PYZsh{}Dla k=10\PYZca{}(5)}
\PY{n}{variance\PYZus{}k3} \PY{o}{=} \PY{n}{describe}\PY{p}{(}\PY{n}{samples\PYZus{}k\PYZus{}3}\PY{p}{)}\PY{o}{.}\PY{n}{variance}
\PY{c+c1}{\PYZsh{}Dla k=10\PYZca{}(6)}
\PY{n}{variance\PYZus{}k4} \PY{o}{=} \PY{n}{describe}\PY{p}{(}\PY{n}{samples\PYZus{}k\PYZus{}4}\PY{p}{)}\PY{o}{.}\PY{n}{variance}
\PY{c+c1}{\PYZsh{}Dla k=10\PYZca{}(7)}
\PY{n}{variance\PYZus{}k5} \PY{o}{=} \PY{n}{describe}\PY{p}{(}\PY{n}{samples\PYZus{}k\PYZus{}5}\PY{p}{)}\PY{o}{.}\PY{n}{variance}
\end{Verbatim}
\end{tcolorbox}

    Podsumowanie obliczeń wartości oczekiwanej oraz kowariancji względem
różnych k

    \begin{tcolorbox}[breakable, size=fbox, boxrule=1pt, pad at break*=1mm,colback=cellbackground, colframe=cellborder]
\prompt{In}{incolor}{15}{\boxspacing}
\begin{Verbatim}[commandchars=\\\{\}]
\PY{n+nb}{dict} \PY{o}{=} \PY{p}{\PYZob{}}\PY{l+s+s1}{\PYZsq{}}\PY{l+s+s1}{k}\PY{l+s+s1}{\PYZsq{}} \PY{p}{:} \PY{p}{[}\PY{l+s+s1}{\PYZsq{}}\PY{l+s+s1}{k\PYZca{}(3)}\PY{l+s+s1}{\PYZsq{}}\PY{p}{,} \PY{l+s+s1}{\PYZsq{}}\PY{l+s+s1}{k\PYZca{}(4)}\PY{l+s+s1}{\PYZsq{}}\PY{p}{,} \PY{l+s+s1}{\PYZsq{}}\PY{l+s+s1}{k\PYZca{}(5)}\PY{l+s+s1}{\PYZsq{}}\PY{p}{,} \PY{l+s+s1}{\PYZsq{}}\PY{l+s+s1}{k\PYZca{}(6)}\PY{l+s+s1}{\PYZsq{}}\PY{p}{,} \PY{l+s+s1}{\PYZsq{}}\PY{l+s+s1}{k\PYZca{}(7)}\PY{l+s+s1}{\PYZsq{}}\PY{p}{]}\PY{p}{,}
        \PY{l+s+s1}{\PYZsq{}}\PY{l+s+s1}{wartość oczekiwana}\PY{l+s+s1}{\PYZsq{}} \PY{p}{:} \PY{p}{[}\PY{n}{expected\PYZus{}value\PYZus{}k1}\PY{p}{,} \PY{n}{expected\PYZus{}value\PYZus{}k2}\PY{p}{,} \PY{n}{expected\PYZus{}value\PYZus{}k3}\PY{p}{,} \PY{n}{expected\PYZus{}value\PYZus{}k4}\PY{p}{,} \PY{n}{expected\PYZus{}value\PYZus{}k5}\PY{p}{]}\PY{p}{,}
        \PY{l+s+s1}{\PYZsq{}}\PY{l+s+s1}{wariancja}\PY{l+s+s1}{\PYZsq{}} \PY{p}{:} \PY{p}{[}\PY{n}{variance\PYZus{}k1}\PY{p}{,} \PY{n}{variance\PYZus{}k2}\PY{p}{,} \PY{n}{variance\PYZus{}k3}\PY{p}{,} \PY{n}{variance\PYZus{}k4}\PY{p}{,} \PY{n}{variance\PYZus{}k5}\PY{p}{]}\PY{p}{\PYZcb{}}
\PY{n}{df} \PY{o}{=} \PY{n}{pd}\PY{o}{.}\PY{n}{DataFrame}\PY{p}{(}\PY{n+nb}{dict}\PY{p}{)}

\PY{n}{df}\PY{o}{.}\PY{n}{style}
\end{Verbatim}
\end{tcolorbox}

            \begin{tcolorbox}[breakable, size=fbox, boxrule=.5pt, pad at break*=1mm, opacityfill=0]
\prompt{Out}{outcolor}{15}{\boxspacing}
\begin{Verbatim}[commandchars=\\\{\}]
<pandas.io.formats.style.Styler at 0x2b48bbf8940>
\end{Verbatim}
\end{tcolorbox}
        
    \hypertarget{obliczenie-funkcji-kowariancyjnej}{%
\subsubsection{Obliczenie funkcji
kowariancyjnej}\label{obliczenie-funkcji-kowariancyjnej}}

Funkcja kowariancyjna - (ang. covariance function) to funkcja, która
opisuje zależność między dwiema zmiennymi losowymi X i Y. Funkcja
kowariancyjna określa, czy zmienne losowe X i Y są skorelowane (czy
zmieniają się razem) lub niezależne (czy zmieniają się niezależnie od
siebie). W przypadku zmiennych losowych niezależnych funkcja
kowariancyjna wynosi 0, a w przypadku zmiennych losowych skorelowanych
funkcja kowariancyjna może być dodatnia lub ujemna.

\[ \operatorname{Cov}(X_i, X_j) = \begin{cases} \sigma^2 & i = j \\ 0 & i \neq j \end{cases} \]

gdzie X\_i i X\_j to próbki szumu białego, a σ\^{}2 to wariancja szumu
białego.

    \begin{tcolorbox}[breakable, size=fbox, boxrule=1pt, pad at break*=1mm,colback=cellbackground, colframe=cellborder]
\prompt{In}{incolor}{16}{\boxspacing}
\begin{Verbatim}[commandchars=\\\{\}]
\PY{k}{def} \PY{n+nf}{covariance}\PY{p}{(}\PY{n}{samples}\PY{p}{,} \PY{n}{k}\PY{p}{)}\PY{p}{:}
    \PY{c+c1}{\PYZsh{} Subtitute DC component noise signal from samples}
    \PY{n}{subtitute\PYZus{}mu\PYZus{}value} \PY{o}{=} \PY{n}{samples} \PY{o}{\PYZhy{}} \PY{n}{mu}
    \PY{c+c1}{\PYZsh{} Compute covariance function}
    \PY{n}{cov} \PY{o}{=} \PY{n}{np}\PY{o}{.}\PY{n}{correlate}\PY{p}{(}\PY{n}{subtitute\PYZus{}mu\PYZus{}value}\PY{p}{,} \PY{n}{subtitute\PYZus{}mu\PYZus{}value}\PY{p}{,} \PY{n}{mode}\PY{o}{=}\PY{l+s+s1}{\PYZsq{}}\PY{l+s+s1}{full}\PY{l+s+s1}{\PYZsq{}}\PY{p}{)} \PY{o}{/} \PY{n}{k}
    \PY{k}{return} \PY{n}{cov}
\end{Verbatim}
\end{tcolorbox}

    \begin{tcolorbox}[breakable, size=fbox, boxrule=1pt, pad at break*=1mm,colback=cellbackground, colframe=cellborder]
\prompt{In}{incolor}{28}{\boxspacing}
\begin{Verbatim}[commandchars=\\\{\}]
\PY{n}{fig}\PY{p}{,} \PY{n}{axarr} \PY{o}{=} \PY{n}{plt}\PY{o}{.}\PY{n}{subplots}\PY{p}{(}\PY{l+m+mi}{1}\PY{p}{,} \PY{l+m+mi}{2}\PY{p}{)}
\PY{n}{fig}\PY{o}{.}\PY{n}{set\PYZus{}figheight}\PY{p}{(}\PY{l+m+mi}{12}\PY{p}{)}
\PY{n}{fig}\PY{o}{.}\PY{n}{set\PYZus{}figwidth}\PY{p}{(}\PY{l+m+mi}{22}\PY{p}{)}
\PY{n}{fig}\PY{o}{.}\PY{n}{suptitle}\PY{p}{(}\PY{l+s+s2}{\PYZdq{}}\PY{l+s+s2}{Funkcja kowariancyjna dla różnego k}\PY{l+s+s2}{\PYZdq{}}\PY{p}{,} \PY{n}{fontsize}\PY{o}{=}\PY{l+m+mi}{16}\PY{p}{)}

\PY{n}{axarr}\PY{p}{[}\PY{l+m+mi}{0}\PY{p}{]}\PY{o}{.}\PY{n}{plot}\PY{p}{(}\PY{n}{covariance}\PY{p}{(}\PY{n}{samples\PYZus{}k\PYZus{}1}\PY{p}{,} \PY{n}{k\PYZus{}1}\PY{p}{)}\PY{p}{)}
\PY{n}{axarr}\PY{p}{[}\PY{l+m+mi}{0}\PY{p}{]}\PY{o}{.}\PY{n}{set\PYZus{}title}\PY{p}{(}\PY{l+s+s1}{\PYZsq{}}\PY{l+s+s1}{k=10\PYZca{}(3)}\PY{l+s+s1}{\PYZsq{}}\PY{p}{)}
\PY{n}{axarr}\PY{p}{[}\PY{l+m+mi}{1}\PY{p}{]}\PY{o}{.}\PY{n}{plot}\PY{p}{(}\PY{n}{covariance}\PY{p}{(}\PY{n}{samples\PYZus{}k\PYZus{}2}\PY{p}{,} \PY{n}{k\PYZus{}2}\PY{p}{)}\PY{p}{)}
\PY{n}{axarr}\PY{p}{[}\PY{l+m+mi}{1}\PY{p}{]}\PY{o}{.}\PY{n}{set\PYZus{}title}\PY{p}{(}\PY{l+s+s1}{\PYZsq{}}\PY{l+s+s1}{k=10\PYZca{}(4)}\PY{l+s+s1}{\PYZsq{}}\PY{p}{)}

\PY{c+c1}{\PYZsh{} Tight layout often produces nice results}
\PY{c+c1}{\PYZsh{} but requires the title to be spaced accordingly}
\PY{n}{fig}\PY{o}{.}\PY{n}{tight\PYZus{}layout}\PY{p}{(}\PY{p}{)}
\PY{n}{fig}\PY{o}{.}\PY{n}{subplots\PYZus{}adjust}\PY{p}{(}\PY{n}{top}\PY{o}{=}\PY{l+m+mf}{0.92}\PY{p}{)}

\PY{n}{plt}\PY{o}{.}\PY{n}{show}\PY{p}{(}\PY{p}{)}
\end{Verbatim}
\end{tcolorbox}

    \begin{center}
    \adjustimage{max size={0.9\linewidth}{0.9\paperheight}}{main_files/main_35_0.png}
    \end{center}
    { \hspace*{\fill} \\}
    

    % Add a bibliography block to the postdoc
    
    
    
\end{document}
