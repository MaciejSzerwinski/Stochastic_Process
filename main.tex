\documentclass[11pt]{article}

    \usepackage[breakable]{tcolorbox}
    \usepackage{parskip} % Stop auto-indenting (to mimic markdown behaviour)
    

    % Basic figure setup, for now with no caption control since it's done
    % automatically by Pandoc (which extracts ![](path) syntax from Markdown).
    \usepackage{graphicx}
    % Maintain compatibility with old templates. Remove in nbconvert 6.0
    \let\Oldincludegraphics\includegraphics
    % Ensure that by default, figures have no caption (until we provide a
    % proper Figure object with a Caption API and a way to capture that
    % in the conversion process - todo).
    \usepackage{caption}
    \DeclareCaptionFormat{nocaption}{}
    \captionsetup{format=nocaption,aboveskip=0pt,belowskip=0pt}

    \usepackage{float}
    \floatplacement{figure}{H} % forces figures to be placed at the correct location
    \usepackage{xcolor} % Allow colors to be defined
    \usepackage{enumerate} % Needed for markdown enumerations to work
    \usepackage{geometry} % Used to adjust the document margins
    \usepackage{amsmath} % Equations
    \usepackage{amssymb} % Equations
    \usepackage{textcomp} % defines textquotesingle
    % Hack from http://tex.stackexchange.com/a/47451/13684:
    \AtBeginDocument{%
        \def\PYZsq{\textquotesingle}% Upright quotes in Pygmentized code
    }
    \usepackage{upquote} % Upright quotes for verbatim code
    \usepackage{eurosym} % defines \euro

    \usepackage{iftex}
    \ifPDFTeX
        \usepackage[T1]{fontenc}
        \IfFileExists{alphabeta.sty}{
              \usepackage{alphabeta}
          }{
              \usepackage[mathletters]{ucs}
              \usepackage[utf8x]{inputenc}
          }
    \else
        \usepackage{fontspec}
        \usepackage{unicode-math}
    \fi

    \usepackage{fancyvrb} % verbatim replacement that allows latex
    \usepackage{grffile} % extends the file name processing of package graphics
                         % to support a larger range
    \makeatletter % fix for old versions of grffile with XeLaTeX
    \@ifpackagelater{grffile}{2019/11/01}
    {
      % Do nothing on new versions
    }
    {
      \def\Gread@@xetex#1{%
        \IfFileExists{"\Gin@base".bb}%
        {\Gread@eps{\Gin@base.bb}}%
        {\Gread@@xetex@aux#1}%
      }
    }
    \makeatother
    \usepackage[Export]{adjustbox} % Used to constrain images to a maximum size
    \adjustboxset{max size={0.9\linewidth}{0.9\paperheight}}

    % The hyperref package gives us a pdf with properly built
    % internal navigation ('pdf bookmarks' for the table of contents,
    % internal cross-reference links, web links for URLs, etc.)
    \usepackage{hyperref}
    % The default LaTeX title has an obnoxious amount of whitespace. By default,
    % titling removes some of it. It also provides customization options.
    \usepackage{titling}
    \usepackage{longtable} % longtable support required by pandoc >1.10
    \usepackage{booktabs}  % table support for pandoc > 1.12.2
    \usepackage{array}     % table support for pandoc >= 2.11.3
    \usepackage{calc}      % table minipage width calculation for pandoc >= 2.11.1
    \usepackage[inline]{enumitem} % IRkernel/repr support (it uses the enumerate* environment)
    \usepackage[normalem]{ulem} % ulem is needed to support strikethroughs (\sout)
                                % normalem makes italics be italics, not underlines
    \usepackage{mathrsfs}
    

    
    % Colors for the hyperref package
    \definecolor{urlcolor}{rgb}{0,.145,.698}
    \definecolor{linkcolor}{rgb}{.71,0.21,0.01}
    \definecolor{citecolor}{rgb}{.12,.54,.11}

    % ANSI colors
    \definecolor{ansi-black}{HTML}{3E424D}
    \definecolor{ansi-black-intense}{HTML}{282C36}
    \definecolor{ansi-red}{HTML}{E75C58}
    \definecolor{ansi-red-intense}{HTML}{B22B31}
    \definecolor{ansi-green}{HTML}{00A250}
    \definecolor{ansi-green-intense}{HTML}{007427}
    \definecolor{ansi-yellow}{HTML}{DDB62B}
    \definecolor{ansi-yellow-intense}{HTML}{B27D12}
    \definecolor{ansi-blue}{HTML}{208FFB}
    \definecolor{ansi-blue-intense}{HTML}{0065CA}
    \definecolor{ansi-magenta}{HTML}{D160C4}
    \definecolor{ansi-magenta-intense}{HTML}{A03196}
    \definecolor{ansi-cyan}{HTML}{60C6C8}
    \definecolor{ansi-cyan-intense}{HTML}{258F8F}
    \definecolor{ansi-white}{HTML}{C5C1B4}
    \definecolor{ansi-white-intense}{HTML}{A1A6B2}
    \definecolor{ansi-default-inverse-fg}{HTML}{FFFFFF}
    \definecolor{ansi-default-inverse-bg}{HTML}{000000}

    % common color for the border for error outputs.
    \definecolor{outerrorbackground}{HTML}{FFDFDF}

    % commands and environments needed by pandoc snippets
    % extracted from the output of `pandoc -s`
    \providecommand{\tightlist}{%
      \setlength{\itemsep}{0pt}\setlength{\parskip}{0pt}}
    \DefineVerbatimEnvironment{Highlighting}{Verbatim}{commandchars=\\\{\}}
    % Add ',fontsize=\small' for more characters per line
    \newenvironment{Shaded}{}{}
    \newcommand{\KeywordTok}[1]{\textcolor[rgb]{0.00,0.44,0.13}{\textbf{{#1}}}}
    \newcommand{\DataTypeTok}[1]{\textcolor[rgb]{0.56,0.13,0.00}{{#1}}}
    \newcommand{\DecValTok}[1]{\textcolor[rgb]{0.25,0.63,0.44}{{#1}}}
    \newcommand{\BaseNTok}[1]{\textcolor[rgb]{0.25,0.63,0.44}{{#1}}}
    \newcommand{\FloatTok}[1]{\textcolor[rgb]{0.25,0.63,0.44}{{#1}}}
    \newcommand{\CharTok}[1]{\textcolor[rgb]{0.25,0.44,0.63}{{#1}}}
    \newcommand{\StringTok}[1]{\textcolor[rgb]{0.25,0.44,0.63}{{#1}}}
    \newcommand{\CommentTok}[1]{\textcolor[rgb]{0.38,0.63,0.69}{\textit{{#1}}}}
    \newcommand{\OtherTok}[1]{\textcolor[rgb]{0.00,0.44,0.13}{{#1}}}
    \newcommand{\AlertTok}[1]{\textcolor[rgb]{1.00,0.00,0.00}{\textbf{{#1}}}}
    \newcommand{\FunctionTok}[1]{\textcolor[rgb]{0.02,0.16,0.49}{{#1}}}
    \newcommand{\RegionMarkerTok}[1]{{#1}}
    \newcommand{\ErrorTok}[1]{\textcolor[rgb]{1.00,0.00,0.00}{\textbf{{#1}}}}
    \newcommand{\NormalTok}[1]{{#1}}

    % Additional commands for more recent versions of Pandoc
    \newcommand{\ConstantTok}[1]{\textcolor[rgb]{0.53,0.00,0.00}{{#1}}}
    \newcommand{\SpecialCharTok}[1]{\textcolor[rgb]{0.25,0.44,0.63}{{#1}}}
    \newcommand{\VerbatimStringTok}[1]{\textcolor[rgb]{0.25,0.44,0.63}{{#1}}}
    \newcommand{\SpecialStringTok}[1]{\textcolor[rgb]{0.73,0.40,0.53}{{#1}}}
    \newcommand{\ImportTok}[1]{{#1}}
    \newcommand{\DocumentationTok}[1]{\textcolor[rgb]{0.73,0.13,0.13}{\textit{{#1}}}}
    \newcommand{\AnnotationTok}[1]{\textcolor[rgb]{0.38,0.63,0.69}{\textbf{\textit{{#1}}}}}
    \newcommand{\CommentVarTok}[1]{\textcolor[rgb]{0.38,0.63,0.69}{\textbf{\textit{{#1}}}}}
    \newcommand{\VariableTok}[1]{\textcolor[rgb]{0.10,0.09,0.49}{{#1}}}
    \newcommand{\ControlFlowTok}[1]{\textcolor[rgb]{0.00,0.44,0.13}{\textbf{{#1}}}}
    \newcommand{\OperatorTok}[1]{\textcolor[rgb]{0.40,0.40,0.40}{{#1}}}
    \newcommand{\BuiltInTok}[1]{{#1}}
    \newcommand{\ExtensionTok}[1]{{#1}}
    \newcommand{\PreprocessorTok}[1]{\textcolor[rgb]{0.74,0.48,0.00}{{#1}}}
    \newcommand{\AttributeTok}[1]{\textcolor[rgb]{0.49,0.56,0.16}{{#1}}}
    \newcommand{\InformationTok}[1]{\textcolor[rgb]{0.38,0.63,0.69}{\textbf{\textit{{#1}}}}}
    \newcommand{\WarningTok}[1]{\textcolor[rgb]{0.38,0.63,0.69}{\textbf{\textit{{#1}}}}}


    % Define a nice break command that doesn't care if a line doesn't already
    % exist.
    \def\br{\hspace*{\fill} \\* }
    % Math Jax compatibility definitions
    \def\gt{>}
    \def\lt{<}
    \let\Oldtex\TeX
    \let\Oldlatex\LaTeX
    \renewcommand{\TeX}{\textrm{\Oldtex}}
    \renewcommand{\LaTeX}{\textrm{\Oldlatex}}
    % Document parameters
    % Document title
    \title{main}
    
    
    
    
    
    
    
% Pygments definitions
\makeatletter
\def\PY@reset{\let\PY@it=\relax \let\PY@bf=\relax%
    \let\PY@ul=\relax \let\PY@tc=\relax%
    \let\PY@bc=\relax \let\PY@ff=\relax}
\def\PY@tok#1{\csname PY@tok@#1\endcsname}
\def\PY@toks#1+{\ifx\relax#1\empty\else%
    \PY@tok{#1}\expandafter\PY@toks\fi}
\def\PY@do#1{\PY@bc{\PY@tc{\PY@ul{%
    \PY@it{\PY@bf{\PY@ff{#1}}}}}}}
\def\PY#1#2{\PY@reset\PY@toks#1+\relax+\PY@do{#2}}

\@namedef{PY@tok@w}{\def\PY@tc##1{\textcolor[rgb]{0.73,0.73,0.73}{##1}}}
\@namedef{PY@tok@c}{\let\PY@it=\textit\def\PY@tc##1{\textcolor[rgb]{0.24,0.48,0.48}{##1}}}
\@namedef{PY@tok@cp}{\def\PY@tc##1{\textcolor[rgb]{0.61,0.40,0.00}{##1}}}
\@namedef{PY@tok@k}{\let\PY@bf=\textbf\def\PY@tc##1{\textcolor[rgb]{0.00,0.50,0.00}{##1}}}
\@namedef{PY@tok@kp}{\def\PY@tc##1{\textcolor[rgb]{0.00,0.50,0.00}{##1}}}
\@namedef{PY@tok@kt}{\def\PY@tc##1{\textcolor[rgb]{0.69,0.00,0.25}{##1}}}
\@namedef{PY@tok@o}{\def\PY@tc##1{\textcolor[rgb]{0.40,0.40,0.40}{##1}}}
\@namedef{PY@tok@ow}{\let\PY@bf=\textbf\def\PY@tc##1{\textcolor[rgb]{0.67,0.13,1.00}{##1}}}
\@namedef{PY@tok@nb}{\def\PY@tc##1{\textcolor[rgb]{0.00,0.50,0.00}{##1}}}
\@namedef{PY@tok@nf}{\def\PY@tc##1{\textcolor[rgb]{0.00,0.00,1.00}{##1}}}
\@namedef{PY@tok@nc}{\let\PY@bf=\textbf\def\PY@tc##1{\textcolor[rgb]{0.00,0.00,1.00}{##1}}}
\@namedef{PY@tok@nn}{\let\PY@bf=\textbf\def\PY@tc##1{\textcolor[rgb]{0.00,0.00,1.00}{##1}}}
\@namedef{PY@tok@ne}{\let\PY@bf=\textbf\def\PY@tc##1{\textcolor[rgb]{0.80,0.25,0.22}{##1}}}
\@namedef{PY@tok@nv}{\def\PY@tc##1{\textcolor[rgb]{0.10,0.09,0.49}{##1}}}
\@namedef{PY@tok@no}{\def\PY@tc##1{\textcolor[rgb]{0.53,0.00,0.00}{##1}}}
\@namedef{PY@tok@nl}{\def\PY@tc##1{\textcolor[rgb]{0.46,0.46,0.00}{##1}}}
\@namedef{PY@tok@ni}{\let\PY@bf=\textbf\def\PY@tc##1{\textcolor[rgb]{0.44,0.44,0.44}{##1}}}
\@namedef{PY@tok@na}{\def\PY@tc##1{\textcolor[rgb]{0.41,0.47,0.13}{##1}}}
\@namedef{PY@tok@nt}{\let\PY@bf=\textbf\def\PY@tc##1{\textcolor[rgb]{0.00,0.50,0.00}{##1}}}
\@namedef{PY@tok@nd}{\def\PY@tc##1{\textcolor[rgb]{0.67,0.13,1.00}{##1}}}
\@namedef{PY@tok@s}{\def\PY@tc##1{\textcolor[rgb]{0.73,0.13,0.13}{##1}}}
\@namedef{PY@tok@sd}{\let\PY@it=\textit\def\PY@tc##1{\textcolor[rgb]{0.73,0.13,0.13}{##1}}}
\@namedef{PY@tok@si}{\let\PY@bf=\textbf\def\PY@tc##1{\textcolor[rgb]{0.64,0.35,0.47}{##1}}}
\@namedef{PY@tok@se}{\let\PY@bf=\textbf\def\PY@tc##1{\textcolor[rgb]{0.67,0.36,0.12}{##1}}}
\@namedef{PY@tok@sr}{\def\PY@tc##1{\textcolor[rgb]{0.64,0.35,0.47}{##1}}}
\@namedef{PY@tok@ss}{\def\PY@tc##1{\textcolor[rgb]{0.10,0.09,0.49}{##1}}}
\@namedef{PY@tok@sx}{\def\PY@tc##1{\textcolor[rgb]{0.00,0.50,0.00}{##1}}}
\@namedef{PY@tok@m}{\def\PY@tc##1{\textcolor[rgb]{0.40,0.40,0.40}{##1}}}
\@namedef{PY@tok@gh}{\let\PY@bf=\textbf\def\PY@tc##1{\textcolor[rgb]{0.00,0.00,0.50}{##1}}}
\@namedef{PY@tok@gu}{\let\PY@bf=\textbf\def\PY@tc##1{\textcolor[rgb]{0.50,0.00,0.50}{##1}}}
\@namedef{PY@tok@gd}{\def\PY@tc##1{\textcolor[rgb]{0.63,0.00,0.00}{##1}}}
\@namedef{PY@tok@gi}{\def\PY@tc##1{\textcolor[rgb]{0.00,0.52,0.00}{##1}}}
\@namedef{PY@tok@gr}{\def\PY@tc##1{\textcolor[rgb]{0.89,0.00,0.00}{##1}}}
\@namedef{PY@tok@ge}{\let\PY@it=\textit}
\@namedef{PY@tok@gs}{\let\PY@bf=\textbf}
\@namedef{PY@tok@gp}{\let\PY@bf=\textbf\def\PY@tc##1{\textcolor[rgb]{0.00,0.00,0.50}{##1}}}
\@namedef{PY@tok@go}{\def\PY@tc##1{\textcolor[rgb]{0.44,0.44,0.44}{##1}}}
\@namedef{PY@tok@gt}{\def\PY@tc##1{\textcolor[rgb]{0.00,0.27,0.87}{##1}}}
\@namedef{PY@tok@err}{\def\PY@bc##1{{\setlength{\fboxsep}{\string -\fboxrule}\fcolorbox[rgb]{1.00,0.00,0.00}{1,1,1}{\strut ##1}}}}
\@namedef{PY@tok@kc}{\let\PY@bf=\textbf\def\PY@tc##1{\textcolor[rgb]{0.00,0.50,0.00}{##1}}}
\@namedef{PY@tok@kd}{\let\PY@bf=\textbf\def\PY@tc##1{\textcolor[rgb]{0.00,0.50,0.00}{##1}}}
\@namedef{PY@tok@kn}{\let\PY@bf=\textbf\def\PY@tc##1{\textcolor[rgb]{0.00,0.50,0.00}{##1}}}
\@namedef{PY@tok@kr}{\let\PY@bf=\textbf\def\PY@tc##1{\textcolor[rgb]{0.00,0.50,0.00}{##1}}}
\@namedef{PY@tok@bp}{\def\PY@tc##1{\textcolor[rgb]{0.00,0.50,0.00}{##1}}}
\@namedef{PY@tok@fm}{\def\PY@tc##1{\textcolor[rgb]{0.00,0.00,1.00}{##1}}}
\@namedef{PY@tok@vc}{\def\PY@tc##1{\textcolor[rgb]{0.10,0.09,0.49}{##1}}}
\@namedef{PY@tok@vg}{\def\PY@tc##1{\textcolor[rgb]{0.10,0.09,0.49}{##1}}}
\@namedef{PY@tok@vi}{\def\PY@tc##1{\textcolor[rgb]{0.10,0.09,0.49}{##1}}}
\@namedef{PY@tok@vm}{\def\PY@tc##1{\textcolor[rgb]{0.10,0.09,0.49}{##1}}}
\@namedef{PY@tok@sa}{\def\PY@tc##1{\textcolor[rgb]{0.73,0.13,0.13}{##1}}}
\@namedef{PY@tok@sb}{\def\PY@tc##1{\textcolor[rgb]{0.73,0.13,0.13}{##1}}}
\@namedef{PY@tok@sc}{\def\PY@tc##1{\textcolor[rgb]{0.73,0.13,0.13}{##1}}}
\@namedef{PY@tok@dl}{\def\PY@tc##1{\textcolor[rgb]{0.73,0.13,0.13}{##1}}}
\@namedef{PY@tok@s2}{\def\PY@tc##1{\textcolor[rgb]{0.73,0.13,0.13}{##1}}}
\@namedef{PY@tok@sh}{\def\PY@tc##1{\textcolor[rgb]{0.73,0.13,0.13}{##1}}}
\@namedef{PY@tok@s1}{\def\PY@tc##1{\textcolor[rgb]{0.73,0.13,0.13}{##1}}}
\@namedef{PY@tok@mb}{\def\PY@tc##1{\textcolor[rgb]{0.40,0.40,0.40}{##1}}}
\@namedef{PY@tok@mf}{\def\PY@tc##1{\textcolor[rgb]{0.40,0.40,0.40}{##1}}}
\@namedef{PY@tok@mh}{\def\PY@tc##1{\textcolor[rgb]{0.40,0.40,0.40}{##1}}}
\@namedef{PY@tok@mi}{\def\PY@tc##1{\textcolor[rgb]{0.40,0.40,0.40}{##1}}}
\@namedef{PY@tok@il}{\def\PY@tc##1{\textcolor[rgb]{0.40,0.40,0.40}{##1}}}
\@namedef{PY@tok@mo}{\def\PY@tc##1{\textcolor[rgb]{0.40,0.40,0.40}{##1}}}
\@namedef{PY@tok@ch}{\let\PY@it=\textit\def\PY@tc##1{\textcolor[rgb]{0.24,0.48,0.48}{##1}}}
\@namedef{PY@tok@cm}{\let\PY@it=\textit\def\PY@tc##1{\textcolor[rgb]{0.24,0.48,0.48}{##1}}}
\@namedef{PY@tok@cpf}{\let\PY@it=\textit\def\PY@tc##1{\textcolor[rgb]{0.24,0.48,0.48}{##1}}}
\@namedef{PY@tok@c1}{\let\PY@it=\textit\def\PY@tc##1{\textcolor[rgb]{0.24,0.48,0.48}{##1}}}
\@namedef{PY@tok@cs}{\let\PY@it=\textit\def\PY@tc##1{\textcolor[rgb]{0.24,0.48,0.48}{##1}}}

\def\PYZbs{\char`\\}
\def\PYZus{\char`\_}
\def\PYZob{\char`\{}
\def\PYZcb{\char`\}}
\def\PYZca{\char`\^}
\def\PYZam{\char`\&}
\def\PYZlt{\char`\<}
\def\PYZgt{\char`\>}
\def\PYZsh{\char`\#}
\def\PYZpc{\char`\%}
\def\PYZdl{\char`\$}
\def\PYZhy{\char`\-}
\def\PYZsq{\char`\'}
\def\PYZdq{\char`\"}
\def\PYZti{\char`\~}
% for compatibility with earlier versions
\def\PYZat{@}
\def\PYZlb{[}
\def\PYZrb{]}
\makeatother


    % For linebreaks inside Verbatim environment from package fancyvrb.
    \makeatletter
        \newbox\Wrappedcontinuationbox
        \newbox\Wrappedvisiblespacebox
        \newcommand*\Wrappedvisiblespace {\textcolor{red}{\textvisiblespace}}
        \newcommand*\Wrappedcontinuationsymbol {\textcolor{red}{\llap{\tiny$\m@th\hookrightarrow$}}}
        \newcommand*\Wrappedcontinuationindent {3ex }
        \newcommand*\Wrappedafterbreak {\kern\Wrappedcontinuationindent\copy\Wrappedcontinuationbox}
        % Take advantage of the already applied Pygments mark-up to insert
        % potential linebreaks for TeX processing.
        %        {, <, #, %, $, ' and ": go to next line.
        %        _, }, ^, &, >, - and ~: stay at end of broken line.
        % Use of \textquotesingle for straight quote.
        \newcommand*\Wrappedbreaksatspecials {%
            \def\PYGZus{\discretionary{\char`\_}{\Wrappedafterbreak}{\char`\_}}%
            \def\PYGZob{\discretionary{}{\Wrappedafterbreak\char`\{}{\char`\{}}%
            \def\PYGZcb{\discretionary{\char`\}}{\Wrappedafterbreak}{\char`\}}}%
            \def\PYGZca{\discretionary{\char`\^}{\Wrappedafterbreak}{\char`\^}}%
            \def\PYGZam{\discretionary{\char`\&}{\Wrappedafterbreak}{\char`\&}}%
            \def\PYGZlt{\discretionary{}{\Wrappedafterbreak\char`\<}{\char`\<}}%
            \def\PYGZgt{\discretionary{\char`\>}{\Wrappedafterbreak}{\char`\>}}%
            \def\PYGZsh{\discretionary{}{\Wrappedafterbreak\char`\#}{\char`\#}}%
            \def\PYGZpc{\discretionary{}{\Wrappedafterbreak\char`\%}{\char`\%}}%
            \def\PYGZdl{\discretionary{}{\Wrappedafterbreak\char`\$}{\char`\$}}%
            \def\PYGZhy{\discretionary{\char`\-}{\Wrappedafterbreak}{\char`\-}}%
            \def\PYGZsq{\discretionary{}{\Wrappedafterbreak\textquotesingle}{\textquotesingle}}%
            \def\PYGZdq{\discretionary{}{\Wrappedafterbreak\char`\"}{\char`\"}}%
            \def\PYGZti{\discretionary{\char`\~}{\Wrappedafterbreak}{\char`\~}}%
        }
        % Some characters . , ; ? ! / are not pygmentized.
        % This macro makes them "active" and they will insert potential linebreaks
        \newcommand*\Wrappedbreaksatpunct {%
            \lccode`\~`\.\lowercase{\def~}{\discretionary{\hbox{\char`\.}}{\Wrappedafterbreak}{\hbox{\char`\.}}}%
            \lccode`\~`\,\lowercase{\def~}{\discretionary{\hbox{\char`\,}}{\Wrappedafterbreak}{\hbox{\char`\,}}}%
            \lccode`\~`\;\lowercase{\def~}{\discretionary{\hbox{\char`\;}}{\Wrappedafterbreak}{\hbox{\char`\;}}}%
            \lccode`\~`\:\lowercase{\def~}{\discretionary{\hbox{\char`\:}}{\Wrappedafterbreak}{\hbox{\char`\:}}}%
            \lccode`\~`\?\lowercase{\def~}{\discretionary{\hbox{\char`\?}}{\Wrappedafterbreak}{\hbox{\char`\?}}}%
            \lccode`\~`\!\lowercase{\def~}{\discretionary{\hbox{\char`\!}}{\Wrappedafterbreak}{\hbox{\char`\!}}}%
            \lccode`\~`\/\lowercase{\def~}{\discretionary{\hbox{\char`\/}}{\Wrappedafterbreak}{\hbox{\char`\/}}}%
            \catcode`\.\active
            \catcode`\,\active
            \catcode`\;\active
            \catcode`\:\active
            \catcode`\?\active
            \catcode`\!\active
            \catcode`\/\active
            \lccode`\~`\~
        }
    \makeatother

    \let\OriginalVerbatim=\Verbatim
    \makeatletter
    \renewcommand{\Verbatim}[1][1]{%
        %\parskip\z@skip
        \sbox\Wrappedcontinuationbox {\Wrappedcontinuationsymbol}%
        \sbox\Wrappedvisiblespacebox {\FV@SetupFont\Wrappedvisiblespace}%
        \def\FancyVerbFormatLine ##1{\hsize\linewidth
            \vtop{\raggedright\hyphenpenalty\z@\exhyphenpenalty\z@
                \doublehyphendemerits\z@\finalhyphendemerits\z@
                \strut ##1\strut}%
        }%
        % If the linebreak is at a space, the latter will be displayed as visible
        % space at end of first line, and a continuation symbol starts next line.
        % Stretch/shrink are however usually zero for typewriter font.
        \def\FV@Space {%
            \nobreak\hskip\z@ plus\fontdimen3\font minus\fontdimen4\font
            \discretionary{\copy\Wrappedvisiblespacebox}{\Wrappedafterbreak}
            {\kern\fontdimen2\font}%
        }%

        % Allow breaks at special characters using \PYG... macros.
        \Wrappedbreaksatspecials
        % Breaks at punctuation characters . , ; ? ! and / need catcode=\active
        \OriginalVerbatim[#1,codes*=\Wrappedbreaksatpunct]%
    }
    \makeatother

    % Exact colors from NB
    \definecolor{incolor}{HTML}{303F9F}
    \definecolor{outcolor}{HTML}{D84315}
    \definecolor{cellborder}{HTML}{CFCFCF}
    \definecolor{cellbackground}{HTML}{F7F7F7}

    % prompt
    \makeatletter
    \newcommand{\boxspacing}{\kern\kvtcb@left@rule\kern\kvtcb@boxsep}
    \makeatother
    \newcommand{\prompt}[4]{
        {\ttfamily\llap{{\color{#2}[#3]:\hspace{3pt}#4}}\vspace{-\baselineskip}}
    }
    

    
    % Prevent overflowing lines due to hard-to-break entities
    \sloppy
    % Setup hyperref package
    \hypersetup{
      breaklinks=true,  % so long urls are correctly broken across lines
      colorlinks=true,
      urlcolor=urlcolor,
      linkcolor=linkcolor,
      citecolor=citecolor,
      }
    % Slightly bigger margins than the latex defaults
    
    \geometry{verbose,tmargin=1in,bmargin=1in,lmargin=1in,rmargin=1in}
    
    

\begin{document}
    
    \maketitle
    
    

    
    \hypertarget{symulacja-ciux105guxf3w-pseudolosowych-ich-filtracja-i-analiza}{%
\section{Symulacja ciągów pseudolosowych, ich filtracja i
analiza}\label{symulacja-ciux105guxf3w-pseudolosowych-ich-filtracja-i-analiza}}

\begin{enumerate}
\def\labelenumi{\arabic{enumi}.}
\tightlist
\item
  Symulować szum biały o rozkładzie normlanym N(5, 0.1).
\item
  Na podstawie otrzymanego ciągu obliczyć gęstość prawdopodobieństwa,
  dystrybuantę, a też wartość oczekiwaną, wariancję i funkcję
  kowariancyjną.
\item
  Przeprowadzić filtrację danych z wykorzystaniem filtru
  dolnoprzepustowego FIR (SOJ) o różnych parametrach.
\item
  Obliczyć gęstość prawdopodobieństwa, dystrybuantę, a też wartość
  oczekiwaną, wariancję i funkcję kowariancyjną sygnału wyjściowego.
  Porównać wyniki z p 2. Wyniki przedstawiać w postaci tablic oraz
  wykresów
\end{enumerate}

    \hypertarget{literatura}{%
\section{Literatura}\label{literatura}}

\begin{enumerate}
\def\labelenumi{\arabic{enumi}.}
\tightlist
\item
  Snopkowski R. Symulacja stochastyczna AGH, Kraków, 2007.
\item
  Niemiro W. Symulacje stochastyczne i metody Monte Carlo, Uniw.
  Warszawski, 2013.
\item
  Cacho K., Bily M., Bukowski J. Random processs, analysis and
  simulation, 1988
\item
  Othes R.K., Enochson Analiza numeryczna szeregów czasowych, WNT,
  Warszawa, 1988
\end{enumerate}

    \hypertarget{wstux119p}{%
\section{Wstęp}\label{wstux119p}}

W dzisiejszych czasach analiza sygnałów odgrywa kluczową rolę w wielu
dziedzinach, takich jak telekomunikacja, akustyka, elektronika,
inżynieria dźwięku, systemy komputerowe i wiele innych. Jednym z
podstawowych rodzajów sygnałów jest tzw. biały szum, który jest szeroko
stosowany jako model losowych fluktuacji lub zakłóceń.

Biały szum jest sygnałem o losowej amplitudzie, w którym wartości próbek
są niezależne od siebie i mają jednakowe rozkłady prawdopodobieństwa.
Jest to sygnał o szerokim pasmie częstotliwości, co oznacza, że zawiera
komponenty we wszystkich dostępnych częstotliwościach w zakresie od zera
do nieskończoności. Nazwa ``biały'' wynika stąd, że taki sygnał zawiera
``równomierny'' rozkład energii w widmie częstotliwości, podobnie jak
białe światło zawiera wszystkie kolory widma światła widzialnego.

W kontekście analizy sygnałów, biały szum jest często modelowany jako
szereg próbek generowanych z rozkładu normalnego, znane jako rozkład
Gaussowski. Rozkład Gaussowski (lub normalny) jest jednym z
najważniejszych i najczęściej występujących rozkładów
prawdopodobieństwa. Charakteryzuje się on symetrią wokół średniej
wartości i opisany jest dwoma parametrami: średnią (wartością
oczekiwaną) i odchyleniem standardowym.

W niniejszym sprawozdaniu skupimy się na analizie białego szumu o
rozkładzie Gaussowskim N(5,0.1), co oznacza, że średnia wartość szumu
wynosi 5, a odchylenie standardowe wynosi 0.1. Przeprowadzimy różne
analizy tego sygnału, aby zbadać jego właściwości statystyczne.

Naszymi głównymi celami będą:

\begin{enumerate}
\def\labelenumi{\arabic{enumi}.}
\item
  Symulacja białego szumu o rozkładzie Gaussowskim: Wygenerujemy szereg
  próbek białego szumu o zadanych parametrach (średnia 5, odchylenie
  standardowe 0.1) w celu uzyskania odpowiedniego zestawu danych do
  analizy.
\item
  Obliczenie gęstości prawdopodobieństwa, dystrybuanty oraz wartości
  oczekiwanej, wariancji i funkcji kowariancyjnej: Przeanalizujemy
  otrzymany ciąg próbek białego szumu i obliczymy jego gęstość
  prawdopodobieństwa oraz dystrybuantę, które pozwolą nam zrozumieć
  rozkład wartości sygnału. Dodatkowo, obliczymy wartość oczekiwaną,
  wariancję i funkcję kowariancyjną, które dostarczą nam informacji na
  temat charakterystyk statystycznych tego sygnału.
\item
  Filtracja danych za pomocą filtru dolnoprzepustowego FIR (SOJ) o
  różnych parametrach: Przeprowadzimy filtrację otrzymanego białego
  szumu za pomocą filtru dolnoprzepustowego FIR (SOJ), który jest
  powszechnie stosowany do usuwania szumów wysokoczęstotliwościowych.
  Zbadamy wpływ różnych parametrów filtru na sygnał wyjściowy.
\item
  Porównanie statystycznych charakterystyk sygnału przed i po filtracji:
  Obliczymy gęstość prawdopodobieństwa, dystrybuantę, wartość
  oczekiwaną, wariancję i funkcję kowariancyjną sygnału wyjściowego po
  filtracji. Następnie porównamy wyniki z wynikami uzyskanymi w punkcie
  2, aby ocenić, w jaki sposób filtracja wpływa na statystyczne cechy
  sygnału.
\end{enumerate}

Wyniki naszej analizy przedstawimy w postaci tabel i wykresów, co
pozwoli nam na lepsze zrozumienie i porównanie różnych parametrów oraz
efektów filtracji. Poprzez przeprowadzenie tej kompleksowej analizy,
będziemy mogli pogłębić naszą wiedzę na temat białego szumu o rozkładzie
Gaussowskim i zrozumieć, jak filtry mogą wpływać na statystyczne
właściwości sygnału.

    \begin{center}\rule{0.5\linewidth}{0.5pt}\end{center}

    \hypertarget{importowanie-potrzebnych-bibliotek}{%
\subsection{Importowanie potrzebnych
bibliotek}\label{importowanie-potrzebnych-bibliotek}}

    \begin{tcolorbox}[breakable, size=fbox, boxrule=1pt, pad at break*=1mm,colback=cellbackground, colframe=cellborder]
\prompt{In}{incolor}{56}{\boxspacing}
\begin{Verbatim}[commandchars=\\\{\}]
\PY{k+kn}{import} \PY{n+nn}{numpy} \PY{k}{as} \PY{n+nn}{np}
\PY{k+kn}{import} \PY{n+nn}{matplotlib}\PY{n+nn}{.}\PY{n+nn}{pyplot} \PY{k}{as} \PY{n+nn}{plt}
\PY{k+kn}{from} \PY{n+nn}{scipy}\PY{n+nn}{.}\PY{n+nn}{stats} \PY{k+kn}{import} \PY{n}{norm}\PY{p}{,} \PY{n}{describe}
\PY{k+kn}{from} \PY{n+nn}{scipy} \PY{k+kn}{import} \PY{n}{signal}
\PY{k+kn}{import} \PY{n+nn}{pandas}  \PY{k}{as} \PY{n+nn}{pd}
\PY{k+kn}{from} \PY{n+nn}{IPython}\PY{n+nn}{.}\PY{n+nn}{display} \PY{k+kn}{import} \PY{n}{display}\PY{p}{,} \PY{n}{Latex}
\PY{k+kn}{from} \PY{n+nn}{tabulate} \PY{k+kn}{import} \PY{n}{tabulate}
\PY{k+kn}{from} \PY{n+nn}{matplotlib}\PY{n+nn}{.}\PY{n+nn}{gridspec} \PY{k+kn}{import} \PY{n}{GridSpec}
\PY{k+kn}{from} \PY{n+nn}{scipy}\PY{n+nn}{.}\PY{n+nn}{signal} \PY{k+kn}{import} \PY{n}{butter}\PY{p}{,} \PY{n}{filtfilt}
\end{Verbatim}
\end{tcolorbox}

    \hypertarget{symulacja-szumu-biaux142ego-o-rozkux142adzie-normlanym-n5-0.1.}{%
\subsection{1. Symulacja szumu białego o rozkładzie normlanym N(5,
0.1).}\label{symulacja-szumu-biaux142ego-o-rozkux142adzie-normlanym-n5-0.1.}}

\hypertarget{wytworzenie-szumu-biaux142ego-o-rozkux142adzie-gassowskim}{%
\subsubsection{Wytworzenie szumu białego o rozkładzie
Gassowskim}\label{wytworzenie-szumu-biaux142ego-o-rozkux142adzie-gassowskim}}

    W poniższym kodzie przeprowadzamy generowanie próbek białego szumu o
rozkładzie Gaussowskim z różnymi parametrami. Biały szum jest
fundamentalnym sygnałem w teorii sygnałów i jest szeroko stosowany w
różnych dziedzinach nauki i technologii.

Domyślne parametry do opisu białego szumu zostały zdefiniowane jako
średnia (mu) równa 5 oraz odchylenie standardowe (sigma) równa 0.1. Te
parametry określają rozkład normalny (Gaussowski) generowanych próbek
białego szumu.

Następnie, dla różnych wartości parametru k (ilość próbek), generujemy
próbki białego szumu o rozkładzie Gaussowskim za pomocą funkcji
np.random.normal z biblioteki NumPy. Wygenerowane próbki białego szumu
są przechowywane w zmiennych samples\_k\_3, samples\_k\_4,
samples\_k\_5, samples\_k\_6 i samples\_k\_7, które reprezentują kolejno
próbki dla k=10\^{}3, k=10\^{}4, k=10\^{}5, k=10\^{}6 oraz k=10\^{}7.

Następnie, za pomocą biblioteki Matplotlib, tworzymy wykresy dla każdego
zestawu próbek białego szumu. Wykresy są przedstawione na siatce 3x2,
gdzie w każdym wierszu znajdują się dwa wykresy dla różnych wartości k.
Szerokość i wysokość figury zostały dostosowane, a tytuł główny ``Szum
biały o rozkładzie Gaussowskim z różnym parametrem k'' został dodany dla
lepszej identyfikacji.

Wizualizacja tych próbek białego szumu o różnych wartościach k pozwala
na obserwację zmian w ich charakterze w zależności od liczby próbek.
Pozwala to lepiej zrozumieć, jak różne wartości k wpływają na wygląd
białego szumu oraz jego losowość.

Wyniki tych generacji próbek białego szumu będą stanowić punkt wyjścia
do dalszej analizy statystycznej w kolejnych częściach sprawozdania.

    \begin{tcolorbox}[breakable, size=fbox, boxrule=1pt, pad at break*=1mm,colback=cellbackground, colframe=cellborder]
\prompt{In}{incolor}{57}{\boxspacing}
\begin{Verbatim}[commandchars=\\\{\}]
\PY{c+c1}{\PYZsh{}Default parameters to describe white noise}
\PY{n}{mu} \PY{o}{=} \PY{l+m+mi}{5}
\PY{n}{sigma} \PY{o}{=} \PY{l+m+mf}{0.1}

\PY{n}{k\PYZus{}3} \PY{o}{=} \PY{n+nb}{pow}\PY{p}{(}\PY{l+m+mi}{10}\PY{p}{,}\PY{l+m+mi}{3}\PY{p}{)}     \PY{c+c1}{\PYZsh{}Amount of samples to generate white noise}
\PY{n}{k\PYZus{}4} \PY{o}{=} \PY{n+nb}{pow}\PY{p}{(}\PY{l+m+mi}{10}\PY{p}{,}\PY{l+m+mi}{4}\PY{p}{)}
\PY{n}{k\PYZus{}5} \PY{o}{=} \PY{n+nb}{pow}\PY{p}{(}\PY{l+m+mi}{10}\PY{p}{,}\PY{l+m+mi}{5}\PY{p}{)}
\PY{n}{k\PYZus{}6} \PY{o}{=} \PY{n+nb}{pow}\PY{p}{(}\PY{l+m+mi}{10}\PY{p}{,}\PY{l+m+mi}{6}\PY{p}{)}
\PY{n}{k\PYZus{}7} \PY{o}{=} \PY{n+nb}{pow}\PY{p}{(}\PY{l+m+mi}{10}\PY{p}{,}\PY{l+m+mi}{7}\PY{p}{)}

\PY{n}{samples\PYZus{}k\PYZus{}3} \PY{o}{=} \PY{n}{np}\PY{o}{.}\PY{n}{random}\PY{o}{.}\PY{n}{normal}\PY{p}{(}\PY{n}{mu}\PY{p}{,} \PY{n}{sigma}\PY{p}{,} \PY{n}{size}\PY{o}{=}\PY{n}{k\PYZus{}3}\PY{p}{)}
\PY{n}{samples\PYZus{}k\PYZus{}4} \PY{o}{=} \PY{n}{np}\PY{o}{.}\PY{n}{random}\PY{o}{.}\PY{n}{normal}\PY{p}{(}\PY{n}{mu}\PY{p}{,} \PY{n}{sigma}\PY{p}{,} \PY{n}{size}\PY{o}{=}\PY{n}{k\PYZus{}4}\PY{p}{)}
\PY{n}{samples\PYZus{}k\PYZus{}5} \PY{o}{=} \PY{n}{np}\PY{o}{.}\PY{n}{random}\PY{o}{.}\PY{n}{normal}\PY{p}{(}\PY{n}{mu}\PY{p}{,} \PY{n}{sigma}\PY{p}{,} \PY{n}{size}\PY{o}{=}\PY{n}{k\PYZus{}5}\PY{p}{)}
\PY{n}{samples\PYZus{}k\PYZus{}6} \PY{o}{=} \PY{n}{np}\PY{o}{.}\PY{n}{random}\PY{o}{.}\PY{n}{normal}\PY{p}{(}\PY{n}{mu}\PY{p}{,} \PY{n}{sigma}\PY{p}{,} \PY{n}{size}\PY{o}{=}\PY{n}{k\PYZus{}6}\PY{p}{)}
\PY{n}{samples\PYZus{}k\PYZus{}7} \PY{o}{=} \PY{n}{np}\PY{o}{.}\PY{n}{random}\PY{o}{.}\PY{n}{normal}\PY{p}{(}\PY{n}{mu}\PY{p}{,} \PY{n}{sigma}\PY{p}{,} \PY{n}{size}\PY{o}{=}\PY{n}{k\PYZus{}7}\PY{p}{)}

\PY{n}{fig}\PY{p}{,} \PY{n}{axarr} \PY{o}{=} \PY{n}{plt}\PY{o}{.}\PY{n}{subplots}\PY{p}{(}\PY{l+m+mi}{3}\PY{p}{,} \PY{l+m+mi}{2}\PY{p}{)}
\PY{n}{fig}\PY{o}{.}\PY{n}{set\PYZus{}figheight}\PY{p}{(}\PY{l+m+mi}{12}\PY{p}{)}
\PY{n}{fig}\PY{o}{.}\PY{n}{set\PYZus{}figwidth}\PY{p}{(}\PY{l+m+mi}{18}\PY{p}{)}
\PY{n}{fig}\PY{o}{.}\PY{n}{suptitle}\PY{p}{(}\PY{l+s+s2}{\PYZdq{}}\PY{l+s+s2}{Szum biały o rozkładzie Gaussowskim z różnym parametrem k}\PY{l+s+s2}{\PYZdq{}}\PY{p}{,} \PY{n}{fontsize}\PY{o}{=}\PY{l+m+mi}{16}\PY{p}{)}

\PY{n}{axarr}\PY{p}{[}\PY{l+m+mi}{0}\PY{p}{,} \PY{l+m+mi}{0}\PY{p}{]}\PY{o}{.}\PY{n}{plot}\PY{p}{(}\PY{n}{samples\PYZus{}k\PYZus{}3}\PY{p}{)}
\PY{n}{axarr}\PY{p}{[}\PY{l+m+mi}{0}\PY{p}{,} \PY{l+m+mi}{0}\PY{p}{]}\PY{o}{.}\PY{n}{set\PYZus{}title}\PY{p}{(}\PY{l+s+s1}{\PYZsq{}}\PY{l+s+s1}{k=10\PYZca{}(3)}\PY{l+s+s1}{\PYZsq{}}\PY{p}{)}
\PY{n}{axarr}\PY{p}{[}\PY{l+m+mi}{0}\PY{p}{,} \PY{l+m+mi}{1}\PY{p}{]}\PY{o}{.}\PY{n}{plot}\PY{p}{(}\PY{n}{samples\PYZus{}k\PYZus{}4}\PY{p}{)}
\PY{n}{axarr}\PY{p}{[}\PY{l+m+mi}{0}\PY{p}{,} \PY{l+m+mi}{1}\PY{p}{]}\PY{o}{.}\PY{n}{set\PYZus{}title}\PY{p}{(}\PY{l+s+s1}{\PYZsq{}}\PY{l+s+s1}{k=10\PYZca{}(4)}\PY{l+s+s1}{\PYZsq{}}\PY{p}{)}
\PY{n}{axarr}\PY{p}{[}\PY{l+m+mi}{1}\PY{p}{,} \PY{l+m+mi}{0}\PY{p}{]}\PY{o}{.}\PY{n}{plot}\PY{p}{(}\PY{n}{samples\PYZus{}k\PYZus{}5}\PY{p}{)}
\PY{n}{axarr}\PY{p}{[}\PY{l+m+mi}{1}\PY{p}{,} \PY{l+m+mi}{0}\PY{p}{]}\PY{o}{.}\PY{n}{set\PYZus{}title}\PY{p}{(}\PY{l+s+s1}{\PYZsq{}}\PY{l+s+s1}{k=10\PYZca{}(5)}\PY{l+s+s1}{\PYZsq{}}\PY{p}{)}
\PY{n}{axarr}\PY{p}{[}\PY{l+m+mi}{1}\PY{p}{,} \PY{l+m+mi}{1}\PY{p}{]}\PY{o}{.}\PY{n}{plot}\PY{p}{(}\PY{n}{samples\PYZus{}k\PYZus{}6}\PY{p}{)}
\PY{n}{axarr}\PY{p}{[}\PY{l+m+mi}{1}\PY{p}{,} \PY{l+m+mi}{1}\PY{p}{]}\PY{o}{.}\PY{n}{set\PYZus{}title}\PY{p}{(}\PY{l+s+s1}{\PYZsq{}}\PY{l+s+s1}{k=10\PYZca{}(6)}\PY{l+s+s1}{\PYZsq{}}\PY{p}{)}
\PY{n}{axarr}\PY{p}{[}\PY{l+m+mi}{2}\PY{p}{,} \PY{l+m+mi}{0}\PY{p}{]}\PY{o}{.}\PY{n}{plot}\PY{p}{(}\PY{n}{samples\PYZus{}k\PYZus{}7}\PY{p}{)}
\PY{n}{axarr}\PY{p}{[}\PY{l+m+mi}{2}\PY{p}{,} \PY{l+m+mi}{0}\PY{p}{]}\PY{o}{.}\PY{n}{set\PYZus{}title}\PY{p}{(}\PY{l+s+s1}{\PYZsq{}}\PY{l+s+s1}{k=10\PYZca{}(7)}\PY{l+s+s1}{\PYZsq{}}\PY{p}{)}
\PY{n}{axarr}\PY{p}{[}\PY{l+m+mi}{2}\PY{p}{,} \PY{l+m+mi}{1}\PY{p}{]}\PY{o}{.}\PY{n}{set\PYZus{}visible}\PY{p}{(}\PY{k+kc}{False}\PY{p}{)}

\PY{c+c1}{\PYZsh{} Tight layout often produces nice results}
\PY{c+c1}{\PYZsh{} but requires the title to be spaced accordingly}
\PY{n}{fig}\PY{o}{.}\PY{n}{tight\PYZus{}layout}\PY{p}{(}\PY{p}{)}
\PY{n}{fig}\PY{o}{.}\PY{n}{subplots\PYZus{}adjust}\PY{p}{(}\PY{n}{top}\PY{o}{=}\PY{l+m+mf}{0.92}\PY{p}{)}

\PY{n}{plt}\PY{o}{.}\PY{n}{show}\PY{p}{(}\PY{p}{)}
\end{Verbatim}
\end{tcolorbox}

    \begin{center}
    \adjustimage{max size={0.9\linewidth}{0.9\paperheight}}{main_files/main_8_0.png}
    \end{center}
    { \hspace*{\fill} \\}
    
    Powyższy wygenerowany diagram prezentuje szum biały składającego się z
określonej liczby próbek zadeklarowanej w zmiennej samples\_k\{i\}. Szum
biały jest rodzajem szumu akustycznego, który posiada całkowicie płaskie
widmo. W procesie stochastycznym szum biały to ciąg nieskorelowanych
zmiennych losowych o zerowej wartości oczekiwanej i stałej
wariancji(czyli biały szum to proces kowariancyjnie stacjonarny) oraz w
sensie ścisłym to biały szum w którym nieskorelowanie wzmianiamy do
niezależności. Biały szum jest tak zwaną ,,cegiełką'' podczas
konstrukcji procesów stochastycznych.

    \hypertarget{wyux15bwietlenie-histogramu-szumu-biaux142ego-o-charakterze-gaussowskim-dla-ruxf3znych-k}{%
\subsubsection{Wyświetlenie histogramu szumu białego o charakterze
Gaussowskim dla róznych
k}\label{wyux15bwietlenie-histogramu-szumu-biaux142ego-o-charakterze-gaussowskim-dla-ruxf3znych-k}}

    \begin{tcolorbox}[breakable, size=fbox, boxrule=1pt, pad at break*=1mm,colback=cellbackground, colframe=cellborder]
\prompt{In}{incolor}{58}{\boxspacing}
\begin{Verbatim}[commandchars=\\\{\}]
\PY{k}{def} \PY{n+nf}{display\PYZus{}hist}\PY{p}{(}\PY{n}{samples}\PY{p}{,} \PY{n}{k}\PY{p}{)}\PY{p}{:}
    \PY{n}{delta\PYZus{}x\PYZus{}1}\PY{o}{=}\PY{l+m+mi}{1}
    \PY{n}{delta\PYZus{}x\PYZus{}2}\PY{o}{=}\PY{l+m+mf}{0.6}
    \PY{n}{delta\PYZus{}x\PYZus{}3}\PY{o}{=}\PY{l+m+mf}{0.3}
    \PY{n}{delta\PYZus{}x\PYZus{}4}\PY{o}{=}\PY{l+m+mf}{0.1}

    \PY{n}{fig}\PY{p}{,} \PY{n}{axarr} \PY{o}{=} \PY{n}{plt}\PY{o}{.}\PY{n}{subplots}\PY{p}{(}\PY{l+m+mi}{2}\PY{p}{,} \PY{l+m+mi}{2}\PY{p}{)}
    \PY{n}{fig}\PY{o}{.}\PY{n}{set\PYZus{}figheight}\PY{p}{(}\PY{l+m+mi}{10}\PY{p}{)}
    \PY{n}{fig}\PY{o}{.}\PY{n}{set\PYZus{}figwidth}\PY{p}{(}\PY{l+m+mi}{12}\PY{p}{)}
    \PY{n}{fig}\PY{o}{.}\PY{n}{suptitle}\PY{p}{(}\PY{l+s+s2}{\PYZdq{}}\PY{l+s+s2}{Histogram szumu białego z różną wartością Δx dla k=}\PY{l+s+si}{\PYZob{}\PYZcb{}}\PY{l+s+s2}{\PYZdq{}}\PY{o}{.}\PY{n}{format}\PY{p}{(}\PY{n}{k}\PY{p}{)}\PY{p}{,} \PY{n}{fontsize}\PY{o}{=}\PY{l+m+mi}{16}\PY{p}{)}

    \PY{n}{axarr}\PY{p}{[}\PY{l+m+mi}{0}\PY{p}{,} \PY{l+m+mi}{0}\PY{p}{]}\PY{o}{.}\PY{n}{hist}\PY{p}{(}\PY{n}{samples}\PY{p}{,} \PY{n}{bins}\PY{o}{=}\PY{n+nb}{int}\PY{p}{(}\PY{l+m+mi}{6}\PY{o}{/}\PY{n}{delta\PYZus{}x\PYZus{}1}\PY{p}{)}\PY{p}{)}
    \PY{n}{axarr}\PY{p}{[}\PY{l+m+mi}{0}\PY{p}{,} \PY{l+m+mi}{0}\PY{p}{]}\PY{o}{.}\PY{n}{set\PYZus{}title}\PY{p}{(}\PY{l+s+s1}{\PYZsq{}}\PY{l+s+s1}{Δx=1}\PY{l+s+s1}{\PYZsq{}}\PY{p}{)}
    \PY{n}{axarr}\PY{p}{[}\PY{l+m+mi}{0}\PY{p}{,} \PY{l+m+mi}{1}\PY{p}{]}\PY{o}{.}\PY{n}{hist}\PY{p}{(}\PY{n}{samples}\PY{p}{,} \PY{n}{bins}\PY{o}{=}\PY{n+nb}{int}\PY{p}{(}\PY{l+m+mi}{6}\PY{o}{/}\PY{n}{delta\PYZus{}x\PYZus{}2}\PY{p}{)}\PY{p}{)}
    \PY{n}{axarr}\PY{p}{[}\PY{l+m+mi}{0}\PY{p}{,} \PY{l+m+mi}{1}\PY{p}{]}\PY{o}{.}\PY{n}{set\PYZus{}title}\PY{p}{(}\PY{l+s+s1}{\PYZsq{}}\PY{l+s+s1}{Δx=0,6}\PY{l+s+s1}{\PYZsq{}}\PY{p}{)}
    \PY{n}{axarr}\PY{p}{[}\PY{l+m+mi}{1}\PY{p}{,} \PY{l+m+mi}{0}\PY{p}{]}\PY{o}{.}\PY{n}{hist}\PY{p}{(}\PY{n}{samples}\PY{p}{,} \PY{n}{bins}\PY{o}{=}\PY{n+nb}{int}\PY{p}{(}\PY{l+m+mi}{6}\PY{o}{/}\PY{n}{delta\PYZus{}x\PYZus{}3}\PY{p}{)}\PY{p}{)}
    \PY{n}{axarr}\PY{p}{[}\PY{l+m+mi}{1}\PY{p}{,} \PY{l+m+mi}{0}\PY{p}{]}\PY{o}{.}\PY{n}{set\PYZus{}title}\PY{p}{(}\PY{l+s+s1}{\PYZsq{}}\PY{l+s+s1}{Δx=0,3}\PY{l+s+s1}{\PYZsq{}}\PY{p}{)}
    \PY{n}{axarr}\PY{p}{[}\PY{l+m+mi}{1}\PY{p}{,} \PY{l+m+mi}{1}\PY{p}{]}\PY{o}{.}\PY{n}{hist}\PY{p}{(}\PY{n}{samples}\PY{p}{,} \PY{n}{bins}\PY{o}{=}\PY{n+nb}{int}\PY{p}{(}\PY{l+m+mi}{6}\PY{o}{/}\PY{n}{delta\PYZus{}x\PYZus{}4}\PY{p}{)}\PY{p}{)}
    \PY{n}{axarr}\PY{p}{[}\PY{l+m+mi}{1}\PY{p}{,} \PY{l+m+mi}{1}\PY{p}{]}\PY{o}{.}\PY{n}{set\PYZus{}title}\PY{p}{(}\PY{l+s+s1}{\PYZsq{}}\PY{l+s+s1}{Δx=0,1}\PY{l+s+s1}{\PYZsq{}}\PY{p}{)}

    \PY{c+c1}{\PYZsh{} Tight layout often produces nice results}
    \PY{c+c1}{\PYZsh{} but requires the title to be spaced accordingly}
    \PY{n}{fig}\PY{o}{.}\PY{n}{tight\PYZus{}layout}\PY{p}{(}\PY{p}{)}
    \PY{n}{fig}\PY{o}{.}\PY{n}{subplots\PYZus{}adjust}\PY{p}{(}\PY{n}{top}\PY{o}{=}\PY{l+m+mf}{0.92}\PY{p}{)}

    \PY{n}{plt}\PY{o}{.}\PY{n}{show}\PY{p}{(}\PY{p}{)}
\end{Verbatim}
\end{tcolorbox}

    \hypertarget{k103}{%
\paragraph{k=10\^{}(3)}\label{k103}}

    \begin{tcolorbox}[breakable, size=fbox, boxrule=1pt, pad at break*=1mm,colback=cellbackground, colframe=cellborder]
\prompt{In}{incolor}{59}{\boxspacing}
\begin{Verbatim}[commandchars=\\\{\}]
\PY{n}{display\PYZus{}hist}\PY{p}{(}\PY{n}{samples\PYZus{}k\PYZus{}3}\PY{p}{,} \PY{n}{k\PYZus{}3}\PY{p}{)}
\end{Verbatim}
\end{tcolorbox}

    \begin{center}
    \adjustimage{max size={0.9\linewidth}{0.9\paperheight}}{main_files/main_13_0.png}
    \end{center}
    { \hspace*{\fill} \\}
    
    Histogram ten przedstawia rozkład próbek białego szumu dla szerokości
binów równych 1. Możemy zauważyć, że histogram jest gładki i symetryczny
wokół wartości oczekiwanej, która wynosi 5. Większość próbek skupia się
wokół tej wartości, co jest zgodne z rozkładem Gaussowskim o średniej 5.
Histogram ten sugeruje, że próbki białego szumu są równomiernie
rozłożone wokół wartości oczekiwanej.

    \hypertarget{k104}{%
\paragraph{k=10\^{}(4)}\label{k104}}

    \begin{tcolorbox}[breakable, size=fbox, boxrule=1pt, pad at break*=1mm,colback=cellbackground, colframe=cellborder]
\prompt{In}{incolor}{60}{\boxspacing}
\begin{Verbatim}[commandchars=\\\{\}]
\PY{n}{display\PYZus{}hist}\PY{p}{(}\PY{n}{samples\PYZus{}k\PYZus{}4}\PY{p}{,} \PY{n}{k\PYZus{}4}\PY{p}{)}
\end{Verbatim}
\end{tcolorbox}

    \begin{center}
    \adjustimage{max size={0.9\linewidth}{0.9\paperheight}}{main_files/main_16_0.png}
    \end{center}
    { \hspace*{\fill} \\}
    
    Histogram ten reprezentuje rozkład próbek białego szumu dla szerokości
binów równych 0,6. Zmniejszenie szerokości binów powoduje większą liczbę
binów i bardziej szczegółowy rozkład próbek na wykresie. Wartości próbek
wciąż są skupione wokół wartości oczekiwanej 5, co jest zgodne z
rozkładem Gaussowskim. Widzimy, że histogram jest symetryczny, a rozkład
próbek wokół wartości oczekiwanej jest bardziej wyraźny niż w przypadku
Δx=1.

    \hypertarget{k105}{%
\paragraph{k=10\^{}(5)}\label{k105}}

    \begin{tcolorbox}[breakable, size=fbox, boxrule=1pt, pad at break*=1mm,colback=cellbackground, colframe=cellborder]
\prompt{In}{incolor}{61}{\boxspacing}
\begin{Verbatim}[commandchars=\\\{\}]
\PY{n}{display\PYZus{}hist}\PY{p}{(}\PY{n}{samples\PYZus{}k\PYZus{}5}\PY{p}{,} \PY{n}{k\PYZus{}5}\PY{p}{)}
\end{Verbatim}
\end{tcolorbox}

    \begin{center}
    \adjustimage{max size={0.9\linewidth}{0.9\paperheight}}{main_files/main_19_0.png}
    \end{center}
    { \hspace*{\fill} \\}
    
    Histogram ten pokazuje rozkład próbek białego szumu dla szerokości binów
wynoszącej 0,3. Zmniejszenie szerokości binów prowadzi do jeszcze
większej liczby binów i jeszcze większej szczegółowości na wykresie.
Rozkład próbek nadal jest symetryczny wokół wartości oczekiwanej 5.
Obserwujemy, że histogram wykazuje lepszą separację między
poszczególnymi wartościami próbek białego szumu, co pozwala nam
zidentyfikować bardziej subtelne niuanse w rozkładzie.

    \hypertarget{k106}{%
\paragraph{k=10\^{}(6)}\label{k106}}

    \begin{tcolorbox}[breakable, size=fbox, boxrule=1pt, pad at break*=1mm,colback=cellbackground, colframe=cellborder]
\prompt{In}{incolor}{62}{\boxspacing}
\begin{Verbatim}[commandchars=\\\{\}]
\PY{n}{display\PYZus{}hist}\PY{p}{(}\PY{n}{samples\PYZus{}k\PYZus{}6}\PY{p}{,} \PY{n}{k\PYZus{}6}\PY{p}{)}
\end{Verbatim}
\end{tcolorbox}

    \begin{center}
    \adjustimage{max size={0.9\linewidth}{0.9\paperheight}}{main_files/main_22_0.png}
    \end{center}
    { \hspace*{\fill} \\}
    
    Ten histogram przedstawia rozkład próbek białego szumu dla najmniejszej
szerokości binów wynoszącej 0,1. Widzimy, że histogram składa się z
bardzo dużej liczby binów, co daje nam bardzo szczegółowy obraz rozkładu
próbek. Mimo drobnych nieregularności w rozkładzie, wciąż możemy
zaobserwować symetryczną strukturę wokół wartości oczekiwanej 5.
Szczegółowość histogramu pozwala nam dostrzec bardziej subtelną
strukturę i różnice w wartościach próbek białego szumu.

    \hypertarget{k107}{%
\paragraph{k=10\^{}(7)}\label{k107}}

    \begin{tcolorbox}[breakable, size=fbox, boxrule=1pt, pad at break*=1mm,colback=cellbackground, colframe=cellborder]
\prompt{In}{incolor}{63}{\boxspacing}
\begin{Verbatim}[commandchars=\\\{\}]
\PY{n}{display\PYZus{}hist}\PY{p}{(}\PY{n}{samples\PYZus{}k\PYZus{}7}\PY{p}{,} \PY{n}{k\PYZus{}7}\PY{p}{)}
\end{Verbatim}
\end{tcolorbox}

    \begin{center}
    \adjustimage{max size={0.9\linewidth}{0.9\paperheight}}{main_files/main_25_0.png}
    \end{center}
    { \hspace*{\fill} \\}
    
    Histogram ten przedstawia rozkład próbek białego szumu dla najmniejszej
szerokości binów wynoszącej 0,1. Obserwujemy bardzo dużą liczbę binów na
wykresie, co daje nam największą szczegółowość i najbardziej dokładne
odzwierciedlenie rozkładu próbek. Podobnie jak wcześniej, pomimo
drobnych nieregularności w rozkładzie, wciąż możemy zaobserwować
symetryczny rozkład wokół wartości oczekiwanej 5. Szczegółowość tego
histogramu pozwala nam jeszcze lepiej zrozumieć różnice i subtelne
niuanse w próbkach białego szumu.

Wnioski ogólne: * Zmniejszanie szerokości binów (większa liczba binów)
prowadzi do bardziej szczegółowego rozkładu próbek białego szumu. *
Pomimo zwiększonej szczegółowości, obserwujemy zachowanie symetrycznego
rozkładu wokół wartości oczekiwanej. * Szerokość binów ma istotny wpływ
na reprezentację rozkładu próbek białego szumu, a odpowiedni wybór
wartości Δx może być istotny w analizie i interpretacji tego rodzaju
szumu.

    \hypertarget{na-podstawie-otrzymanego-ciux105gu-obliczyux107-gux119stoux15bux107-prawdopodobieux144stwta-dystrybuantux119-a-teux17c-wartoux15bux107-oczekiwan-wariancjux119-i-funkcjux119-kowariancyjnux105.}{%
\subsection{2. Na podstawie otrzymanego ciągu obliczyć gęstość
prawdopodobieństwta, dystrybuantę, a też wartość oczekiwan, wariancję i
funkcję
kowariancyjną.}\label{na-podstawie-otrzymanego-ciux105gu-obliczyux107-gux119stoux15bux107-prawdopodobieux144stwta-dystrybuantux119-a-teux17c-wartoux15bux107-oczekiwan-wariancjux119-i-funkcjux119-kowariancyjnux105.}}

    \hypertarget{obliczenie-gux119stoux15bci-prawdopodobieux144stwa}{%
\subsubsection{Obliczenie gęstości
prawdopodobieństwa}\label{obliczenie-gux119stoux15bci-prawdopodobieux144stwa}}

Gęstość prawdopodobieństwa (ang. probability density function) to
funkcja, która opisuje rozkład prawdopodobieństwa zmiennej losowej X.
Gęstość prawdopodobieństwa może być używana do obliczenia
prawdopodobieństwa wystąpienia wartości zmiennej losowej w określonym
przedziale. W przeciwieństwie do dystrybuanty, gęstość
prawdopodobieństwa nie jest równa prawdopodobieństwu, lecz określa
szybkość zmian prawdopodobieństwa zmiennej losowej wokół danej wartości.

\[ f(x) = \frac{1}{\sqrt{2\pi\sigma^2}} e^{-\frac{(x-\mu)^2}{2\sigma^2}} \]

Uwaga aby wyświetlić gęstość prawdopodobieństwa należy posortować
najpierw tablicę. Wytłumaczenie w linku poniżej\\
https://stackoverflow.com/questions/71296986/how-to-draw-the-probability-density-function-pdf-plot-in-python

    \begin{tcolorbox}[breakable, size=fbox, boxrule=1pt, pad at break*=1mm,colback=cellbackground, colframe=cellborder]
\prompt{In}{incolor}{64}{\boxspacing}
\begin{Verbatim}[commandchars=\\\{\}]
\PY{k}{def} \PY{n+nf}{probability\PYZus{}pdf}\PY{p}{(}\PY{n}{samples}\PY{p}{)}\PY{p}{:}
    \PY{n}{sort\PYZus{}samples} \PY{o}{=} \PY{n}{np}\PY{o}{.}\PY{n}{sort}\PY{p}{(}\PY{n}{samples}\PY{p}{)}
    \PY{n}{probability\PYZus{}pdf} \PY{o}{=} \PY{n}{norm}\PY{o}{.}\PY{n}{pdf}\PY{p}{(}\PY{n}{sort\PYZus{}samples}\PY{p}{,} \PY{n}{mu}\PY{p}{,} \PY{n}{sigma}\PY{p}{)}
    \PY{k}{return} \PY{n}{probability\PYZus{}pdf}
\end{Verbatim}
\end{tcolorbox}

    \begin{tcolorbox}[breakable, size=fbox, boxrule=1pt, pad at break*=1mm,colback=cellbackground, colframe=cellborder]
\prompt{In}{incolor}{65}{\boxspacing}
\begin{Verbatim}[commandchars=\\\{\}]
\PY{n}{fig}\PY{p}{,} \PY{n}{axarr} \PY{o}{=} \PY{n}{plt}\PY{o}{.}\PY{n}{subplots}\PY{p}{(}\PY{l+m+mi}{2}\PY{p}{,} \PY{l+m+mi}{3}\PY{p}{)}
\PY{n}{fig}\PY{o}{.}\PY{n}{set\PYZus{}figheight}\PY{p}{(}\PY{l+m+mi}{12}\PY{p}{)}
\PY{n}{fig}\PY{o}{.}\PY{n}{set\PYZus{}figwidth}\PY{p}{(}\PY{l+m+mi}{20}\PY{p}{)}
\PY{n}{fig}\PY{o}{.}\PY{n}{suptitle}\PY{p}{(}\PY{l+s+s2}{\PYZdq{}}\PY{l+s+s2}{Gęstość prawdopodobieństwa dla różnego k}\PY{l+s+s2}{\PYZdq{}}\PY{p}{,} \PY{n}{fontsize}\PY{o}{=}\PY{l+m+mi}{16}\PY{p}{)}

\PY{n}{axarr}\PY{p}{[}\PY{l+m+mi}{0}\PY{p}{,} \PY{l+m+mi}{0}\PY{p}{]}\PY{o}{.}\PY{n}{plot}\PY{p}{(}\PY{n}{np}\PY{o}{.}\PY{n}{sort}\PY{p}{(}\PY{n}{samples\PYZus{}k\PYZus{}3}\PY{p}{)}\PY{p}{,} \PY{n}{probability\PYZus{}pdf}\PY{p}{(}\PY{n}{samples\PYZus{}k\PYZus{}3}\PY{p}{)}\PY{p}{)}
\PY{n}{axarr}\PY{p}{[}\PY{l+m+mi}{0}\PY{p}{,} \PY{l+m+mi}{0}\PY{p}{]}\PY{o}{.}\PY{n}{set\PYZus{}title}\PY{p}{(}\PY{l+s+s1}{\PYZsq{}}\PY{l+s+s1}{k=10\PYZca{}(3)}\PY{l+s+s1}{\PYZsq{}}\PY{p}{)}
\PY{n}{axarr}\PY{p}{[}\PY{l+m+mi}{0}\PY{p}{,} \PY{l+m+mi}{1}\PY{p}{]}\PY{o}{.}\PY{n}{plot}\PY{p}{(}\PY{n}{np}\PY{o}{.}\PY{n}{sort}\PY{p}{(}\PY{n}{samples\PYZus{}k\PYZus{}4}\PY{p}{)}\PY{p}{,} \PY{n}{probability\PYZus{}pdf}\PY{p}{(}\PY{n}{samples\PYZus{}k\PYZus{}4}\PY{p}{)}\PY{p}{)}
\PY{n}{axarr}\PY{p}{[}\PY{l+m+mi}{0}\PY{p}{,} \PY{l+m+mi}{1}\PY{p}{]}\PY{o}{.}\PY{n}{set\PYZus{}title}\PY{p}{(}\PY{l+s+s1}{\PYZsq{}}\PY{l+s+s1}{k=10\PYZca{}(4)}\PY{l+s+s1}{\PYZsq{}}\PY{p}{)}
\PY{n}{axarr}\PY{p}{[}\PY{l+m+mi}{0}\PY{p}{,} \PY{l+m+mi}{2}\PY{p}{]}\PY{o}{.}\PY{n}{plot}\PY{p}{(}\PY{n}{np}\PY{o}{.}\PY{n}{sort}\PY{p}{(}\PY{n}{samples\PYZus{}k\PYZus{}5}\PY{p}{)}\PY{p}{,} \PY{n}{probability\PYZus{}pdf}\PY{p}{(}\PY{n}{samples\PYZus{}k\PYZus{}5}\PY{p}{)}\PY{p}{)}
\PY{n}{axarr}\PY{p}{[}\PY{l+m+mi}{0}\PY{p}{,} \PY{l+m+mi}{2}\PY{p}{]}\PY{o}{.}\PY{n}{set\PYZus{}title}\PY{p}{(}\PY{l+s+s1}{\PYZsq{}}\PY{l+s+s1}{k=10\PYZca{}(5)}\PY{l+s+s1}{\PYZsq{}}\PY{p}{)}
\PY{n}{axarr}\PY{p}{[}\PY{l+m+mi}{1}\PY{p}{,} \PY{l+m+mi}{0}\PY{p}{]}\PY{o}{.}\PY{n}{plot}\PY{p}{(}\PY{n}{np}\PY{o}{.}\PY{n}{sort}\PY{p}{(}\PY{n}{samples\PYZus{}k\PYZus{}6}\PY{p}{)}\PY{p}{,} \PY{n}{probability\PYZus{}pdf}\PY{p}{(}\PY{n}{samples\PYZus{}k\PYZus{}6}\PY{p}{)}\PY{p}{)}
\PY{n}{axarr}\PY{p}{[}\PY{l+m+mi}{1}\PY{p}{,} \PY{l+m+mi}{0}\PY{p}{]}\PY{o}{.}\PY{n}{set\PYZus{}title}\PY{p}{(}\PY{l+s+s1}{\PYZsq{}}\PY{l+s+s1}{k=10\PYZca{}(6)}\PY{l+s+s1}{\PYZsq{}}\PY{p}{)}
\PY{n}{axarr}\PY{p}{[}\PY{l+m+mi}{1}\PY{p}{,} \PY{l+m+mi}{1}\PY{p}{]}\PY{o}{.}\PY{n}{plot}\PY{p}{(}\PY{n}{np}\PY{o}{.}\PY{n}{sort}\PY{p}{(}\PY{n}{samples\PYZus{}k\PYZus{}7}\PY{p}{)}\PY{p}{,} \PY{n}{probability\PYZus{}pdf}\PY{p}{(}\PY{n}{samples\PYZus{}k\PYZus{}7}\PY{p}{)}\PY{p}{)}
\PY{n}{axarr}\PY{p}{[}\PY{l+m+mi}{1}\PY{p}{,} \PY{l+m+mi}{1}\PY{p}{]}\PY{o}{.}\PY{n}{set\PYZus{}title}\PY{p}{(}\PY{l+s+s1}{\PYZsq{}}\PY{l+s+s1}{k=10\PYZca{}(7)}\PY{l+s+s1}{\PYZsq{}}\PY{p}{)}
\PY{n}{axarr}\PY{p}{[}\PY{l+m+mi}{1}\PY{p}{,} \PY{l+m+mi}{2}\PY{p}{]}\PY{o}{.}\PY{n}{set\PYZus{}visible}\PY{p}{(}\PY{k+kc}{False}\PY{p}{)}

\PY{c+c1}{\PYZsh{} Tight layout often produces nice results}
\PY{c+c1}{\PYZsh{} but requires the title to be spaced accordingly}
\PY{n}{fig}\PY{o}{.}\PY{n}{tight\PYZus{}layout}\PY{p}{(}\PY{p}{)}
\PY{n}{fig}\PY{o}{.}\PY{n}{subplots\PYZus{}adjust}\PY{p}{(}\PY{n}{top}\PY{o}{=}\PY{l+m+mf}{0.92}\PY{p}{)}

\PY{n}{plt}\PY{o}{.}\PY{n}{show}\PY{p}{(}\PY{p}{)}
\end{Verbatim}
\end{tcolorbox}

    \begin{center}
    \adjustimage{max size={0.9\linewidth}{0.9\paperheight}}{main_files/main_30_0.png}
    \end{center}
    { \hspace*{\fill} \\}
    
    Dla każdej wartości k (k=10\^{}3, k=10\^{}4, k=10\^{}5, k=10\^{}6,
k=10\^{}7), zostały wygenerowane odpowiednie próbki białego szumu o
rozkładzie Gaussowskim N(5, 0.1). Następnie te próbki zostały
posortowane rosnąco.

Dla każdej wartości k, obliczona została gęstość prawdopodobieństwa dla
posortowanych próbek, korzystając z rozkładu normalnego o parametrach
mu=5 i sigma=0.1. Funkcja probability\_pdf zwraca wartości gęstości
prawdopodobieństwa dla poszczególnych próbek.

Na wykresie przedstawiono wyniki dla każdej wartości k. Na osi poziomej
znajdują się posortowane próbki, a na osi pionowej wartości gęstości
prawdopodobieństwa. Wykresy te pozwalają na wizualizację rozkładu próbek
białego szumu i porównanie różnych wartości k.

Analiza i wnioski: * W miarę zwiększania wartości k, czyli liczby
generowanych próbek, gęstość prawdopodobieństwa staje się bardziej
gładka i zbliża się do gęstości rozkładu Gaussowskiego o parametrach
mu=5 i sigma=0.1. * Dla k=10\^{}3 i k=10\^{}4, gęstość
prawdopodobieństwa wykazuje nieco większe fluktuacje wokół wartości
oczekiwanej. * Dla większych wartości k (k=10\^{}5, k=10\^{}6,
k=10\^{}7), gęstość prawdopodobieństwa staje się coraz bardziej zbliżona
do teoretycznej gęstości rozkładu Gaussowskiego. * Wartości gęstości
prawdopodobieństwa wokół wartości oczekiwanej są najwyższe, co jest
zgodne z oczekiwaniami, ponieważ wartość oczekiwana rozkładu
Gaussowskiego jest równa 5. * Analiza tych wykresów pozwala lepiej
zrozumieć, jak wartość k wpływa na gładkość i dokładność przybliżenia
rozkładu próbek białego szumu do rozkładu Gaussowskiego. Wnioski te
ilustrują, jak analiza gęstości prawdopodobieństwa może pomóc w lepszym
zrozumieniu i wizualizacji charakterystyk białego szumu oraz wpływu
parametru k na jakość aproksymacji rozkładu Gaussowskiego.

    \hypertarget{obliczenie-dystrybuanty}{%
\subsubsection{Obliczenie dystrybuanty}\label{obliczenie-dystrybuanty}}

Dystrybuanta - (ang. cumulative distribution function) to funkcja
matematyczna, która określa prawdopodobieństwo, że losowo wybrana
zmienna losowa X jest mniejsza lub równa danej wartości x, tzn. F(x) =
P(X \(\leq\) x). Dystrybuanta może być użyta do określenia takich
wartości jak kwantyle (np. mediana) oraz do badania asymetrii i ogona
rozkładu zmiennej losowej.

\[ F(x) = \frac{1}{2}\left[1 + \operatorname{erf}\left(\frac{x-\mu}{\sigma\sqrt{2}}\right)\right] \]

Podobnie jak dla obliczenia gęstości prawdopodobieństwa najpierw
należało skorzystać z posortowanych wcześniej danych w tablicy samples
https://stackoverflow.com/questions/24788200/calculate-the-cumulative-distribution-function-cdf-in-python

    \begin{tcolorbox}[breakable, size=fbox, boxrule=1pt, pad at break*=1mm,colback=cellbackground, colframe=cellborder]
\prompt{In}{incolor}{66}{\boxspacing}
\begin{Verbatim}[commandchars=\\\{\}]
\PY{k}{def} \PY{n+nf}{norm\PYZus{}cdf}\PY{p}{(}\PY{n}{samples}\PY{p}{)}\PY{p}{:}
    \PY{n}{sort\PYZus{}samples} \PY{o}{=} \PY{n}{np}\PY{o}{.}\PY{n}{sort}\PY{p}{(}\PY{n}{samples}\PY{p}{)}
    \PY{n}{norm\PYZus{}cdf} \PY{o}{=} \PY{n}{norm}\PY{o}{.}\PY{n}{cdf}\PY{p}{(}\PY{n}{sort\PYZus{}samples}\PY{p}{,} \PY{n}{mu}\PY{p}{,} \PY{n}{sigma}\PY{p}{)}
    \PY{k}{return} \PY{n}{norm\PYZus{}cdf}
\end{Verbatim}
\end{tcolorbox}

    \begin{tcolorbox}[breakable, size=fbox, boxrule=1pt, pad at break*=1mm,colback=cellbackground, colframe=cellborder]
\prompt{In}{incolor}{67}{\boxspacing}
\begin{Verbatim}[commandchars=\\\{\}]
\PY{n}{fig}\PY{p}{,} \PY{n}{axarr} \PY{o}{=} \PY{n}{plt}\PY{o}{.}\PY{n}{subplots}\PY{p}{(}\PY{l+m+mi}{2}\PY{p}{,} \PY{l+m+mi}{3}\PY{p}{)}
\PY{n}{fig}\PY{o}{.}\PY{n}{set\PYZus{}figheight}\PY{p}{(}\PY{l+m+mi}{12}\PY{p}{)}
\PY{n}{fig}\PY{o}{.}\PY{n}{set\PYZus{}figwidth}\PY{p}{(}\PY{l+m+mi}{20}\PY{p}{)}
\PY{n}{fig}\PY{o}{.}\PY{n}{suptitle}\PY{p}{(}\PY{l+s+s2}{\PYZdq{}}\PY{l+s+s2}{Gęstość prawdopodobieństwa dla różnego k}\PY{l+s+s2}{\PYZdq{}}\PY{p}{,} \PY{n}{fontsize}\PY{o}{=}\PY{l+m+mi}{16}\PY{p}{)}

\PY{n}{axarr}\PY{p}{[}\PY{l+m+mi}{0}\PY{p}{,} \PY{l+m+mi}{0}\PY{p}{]}\PY{o}{.}\PY{n}{plot}\PY{p}{(}\PY{n}{np}\PY{o}{.}\PY{n}{sort}\PY{p}{(}\PY{n}{samples\PYZus{}k\PYZus{}3}\PY{p}{)}\PY{p}{,} \PY{n}{norm\PYZus{}cdf}\PY{p}{(}\PY{n}{samples\PYZus{}k\PYZus{}3}\PY{p}{)}\PY{p}{)}
\PY{n}{axarr}\PY{p}{[}\PY{l+m+mi}{0}\PY{p}{,} \PY{l+m+mi}{0}\PY{p}{]}\PY{o}{.}\PY{n}{set\PYZus{}title}\PY{p}{(}\PY{l+s+s1}{\PYZsq{}}\PY{l+s+s1}{k=10\PYZca{}(3)}\PY{l+s+s1}{\PYZsq{}}\PY{p}{)}
\PY{n}{axarr}\PY{p}{[}\PY{l+m+mi}{0}\PY{p}{,} \PY{l+m+mi}{1}\PY{p}{]}\PY{o}{.}\PY{n}{plot}\PY{p}{(}\PY{n}{np}\PY{o}{.}\PY{n}{sort}\PY{p}{(}\PY{n}{samples\PYZus{}k\PYZus{}4}\PY{p}{)}\PY{p}{,} \PY{n}{norm\PYZus{}cdf}\PY{p}{(}\PY{n}{samples\PYZus{}k\PYZus{}4}\PY{p}{)}\PY{p}{)}
\PY{n}{axarr}\PY{p}{[}\PY{l+m+mi}{0}\PY{p}{,} \PY{l+m+mi}{1}\PY{p}{]}\PY{o}{.}\PY{n}{set\PYZus{}title}\PY{p}{(}\PY{l+s+s1}{\PYZsq{}}\PY{l+s+s1}{k=10\PYZca{}(4)}\PY{l+s+s1}{\PYZsq{}}\PY{p}{)}
\PY{n}{axarr}\PY{p}{[}\PY{l+m+mi}{0}\PY{p}{,} \PY{l+m+mi}{2}\PY{p}{]}\PY{o}{.}\PY{n}{plot}\PY{p}{(}\PY{n}{np}\PY{o}{.}\PY{n}{sort}\PY{p}{(}\PY{n}{samples\PYZus{}k\PYZus{}5}\PY{p}{)}\PY{p}{,} \PY{n}{norm\PYZus{}cdf}\PY{p}{(}\PY{n}{samples\PYZus{}k\PYZus{}5}\PY{p}{)}\PY{p}{)}
\PY{n}{axarr}\PY{p}{[}\PY{l+m+mi}{0}\PY{p}{,} \PY{l+m+mi}{2}\PY{p}{]}\PY{o}{.}\PY{n}{set\PYZus{}title}\PY{p}{(}\PY{l+s+s1}{\PYZsq{}}\PY{l+s+s1}{k=10\PYZca{}(5)}\PY{l+s+s1}{\PYZsq{}}\PY{p}{)}
\PY{n}{axarr}\PY{p}{[}\PY{l+m+mi}{1}\PY{p}{,} \PY{l+m+mi}{0}\PY{p}{]}\PY{o}{.}\PY{n}{plot}\PY{p}{(}\PY{n}{np}\PY{o}{.}\PY{n}{sort}\PY{p}{(}\PY{n}{samples\PYZus{}k\PYZus{}6}\PY{p}{)}\PY{p}{,} \PY{n}{norm\PYZus{}cdf}\PY{p}{(}\PY{n}{samples\PYZus{}k\PYZus{}6}\PY{p}{)}\PY{p}{)}
\PY{n}{axarr}\PY{p}{[}\PY{l+m+mi}{1}\PY{p}{,} \PY{l+m+mi}{0}\PY{p}{]}\PY{o}{.}\PY{n}{set\PYZus{}title}\PY{p}{(}\PY{l+s+s1}{\PYZsq{}}\PY{l+s+s1}{k=10\PYZca{}(6)}\PY{l+s+s1}{\PYZsq{}}\PY{p}{)}
\PY{n}{axarr}\PY{p}{[}\PY{l+m+mi}{1}\PY{p}{,} \PY{l+m+mi}{1}\PY{p}{]}\PY{o}{.}\PY{n}{plot}\PY{p}{(}\PY{n}{np}\PY{o}{.}\PY{n}{sort}\PY{p}{(}\PY{n}{samples\PYZus{}k\PYZus{}7}\PY{p}{)}\PY{p}{,} \PY{n}{norm\PYZus{}cdf}\PY{p}{(}\PY{n}{samples\PYZus{}k\PYZus{}7}\PY{p}{)}\PY{p}{)}
\PY{n}{axarr}\PY{p}{[}\PY{l+m+mi}{1}\PY{p}{,} \PY{l+m+mi}{1}\PY{p}{]}\PY{o}{.}\PY{n}{set\PYZus{}title}\PY{p}{(}\PY{l+s+s1}{\PYZsq{}}\PY{l+s+s1}{k=10\PYZca{}(7)}\PY{l+s+s1}{\PYZsq{}}\PY{p}{)}
\PY{n}{axarr}\PY{p}{[}\PY{l+m+mi}{1}\PY{p}{,} \PY{l+m+mi}{2}\PY{p}{]}\PY{o}{.}\PY{n}{set\PYZus{}visible}\PY{p}{(}\PY{k+kc}{False}\PY{p}{)}

\PY{c+c1}{\PYZsh{} Tight layout often produces nice results}
\PY{c+c1}{\PYZsh{} but requires the title to be spaced accordingly}
\PY{n}{fig}\PY{o}{.}\PY{n}{tight\PYZus{}layout}\PY{p}{(}\PY{p}{)}
\PY{n}{fig}\PY{o}{.}\PY{n}{subplots\PYZus{}adjust}\PY{p}{(}\PY{n}{top}\PY{o}{=}\PY{l+m+mf}{0.92}\PY{p}{)}

\PY{n}{plt}\PY{o}{.}\PY{n}{show}\PY{p}{(}\PY{p}{)}
\end{Verbatim}
\end{tcolorbox}

    \begin{center}
    \adjustimage{max size={0.9\linewidth}{0.9\paperheight}}{main_files/main_34_0.png}
    \end{center}
    { \hspace*{\fill} \\}
    
    Dla każdej wartości k (k=10\^{}3, k=10\^{}4, k=10\^{}5, k=10\^{}6,
k=10\^{}7), zostały wygenerowane odpowiednie próbki białego szumu o
rozkładzie Gaussowskim N(5, 0.1). Następnie te próbki zostały
posortowane rosnąco.

Dla każdej wartości k, obliczona została dystrybuanta dla posortowanych
próbek, korzystając z rozkładu normalnego o parametrach mu=5 i
sigma=0.1. Funkcja norm\_cdf zwraca wartości dystrybuanty dla
poszczególnych próbek.

Na wykresie przedstawiono wyniki dla każdej wartości k. Na osi poziomej
znajdują się posortowane próbki, a na osi pionowej wartości
dystrybuanty. Wykresy te pozwalają na wizualizację rozkładu próbek
białego szumu i porównanie różnych wartości k.

Analiza i wnioski: * W miarę zwiększania wartości k, dystrybuanta staje
się coraz bardziej zbliżona do dystrybuanty rozkładu Gaussowskiego o
parametrach mu=5 i sigma=0.1. * Dla małych wartości k (k=10\^{}3,
k=10\^{}4), dystrybuanta wykazuje większe fluktuacje wokół wartości
oczekiwanej. * Dla większych wartości k (k=10\^{}5, k=10\^{}6,
k=10\^{}7), dystrybuanta staje się coraz bardziej zbliżona do
teoretycznej dystrybuanty rozkładu Gaussowskiego. * Wartości
dystrybuanty w punkcie wartości oczekiwanej (mu) są coraz bliższe 0.5,
co jest zgodne z właściwościami dystrybuanty rozkładu Gaussowskiego. *
Analiza tych wykresów pozwala lepiej zrozumieć, jak wartość k wpływa na
dokładność aproksymacji dystrybuanty rozkładu Gaussowskiego dla próbek
białego szumu.

Wnioski te ilustrują, jak analiza dystrybuanty może pomóc w lepszym
zrozumieniu i wizualizacji charakterystyk białego szumu oraz wpływu
parametru k na jakość aproksymacji dystrybuanty rozkładu Gaussowskiego.

    \hypertarget{obliczenie-wartoux15bci-oczekiwanej}{%
\subsubsection{Obliczenie wartości
oczekiwanej}\label{obliczenie-wartoux15bci-oczekiwanej}}

Wartość oczekiwana - (ang. expected value) to średnia wartość zmiennej
losowej X. Można ją wyznaczyć przez pomnożenie każdej wartości zmiennej
losowej przez jej prawdopodobieństwo i zsumowanie tych wartości. Wartość
oczekiwana jest jednym z najważniejszych parametrów opisujących zmienną
losową, ponieważ pozwala na określenie, jakiej wartości można oczekiwać,
gdy zmienna losowa zostanie wielokrotnie pomierzona.

\[ E(X) = \int_{-\infty}^{\infty} x f(x) dx \]

Wytłumaczone na filmiku jak w pythonie obliczać wartość oczekiwaną oraz
wariancję https://www.youtube.com/watch?v=ikcUBqELZVU

    \begin{tcolorbox}[breakable, size=fbox, boxrule=1pt, pad at break*=1mm,colback=cellbackground, colframe=cellborder]
\prompt{In}{incolor}{68}{\boxspacing}
\begin{Verbatim}[commandchars=\\\{\}]
\PY{c+c1}{\PYZsh{}Dla k=10\PYZca{}(3)}
\PY{n}{expected\PYZus{}value\PYZus{}k1} \PY{o}{=} \PY{n}{describe}\PY{p}{(}\PY{n}{samples\PYZus{}k\PYZus{}3}\PY{p}{)}\PY{o}{.}\PY{n}{mean}
\PY{c+c1}{\PYZsh{}Dla k=10\PYZca{}(4)}
\PY{n}{expected\PYZus{}value\PYZus{}k2} \PY{o}{=} \PY{n}{describe}\PY{p}{(}\PY{n}{samples\PYZus{}k\PYZus{}4}\PY{p}{)}\PY{o}{.}\PY{n}{mean}
\PY{c+c1}{\PYZsh{}Dla k=10\PYZca{}(5)}
\PY{n}{expected\PYZus{}value\PYZus{}k3} \PY{o}{=} \PY{n}{describe}\PY{p}{(}\PY{n}{samples\PYZus{}k\PYZus{}5}\PY{p}{)}\PY{o}{.}\PY{n}{mean}
\PY{c+c1}{\PYZsh{}Dla k=10\PYZca{}(6)}
\PY{n}{expected\PYZus{}value\PYZus{}k4} \PY{o}{=} \PY{n}{describe}\PY{p}{(}\PY{n}{samples\PYZus{}k\PYZus{}6}\PY{p}{)}\PY{o}{.}\PY{n}{mean}
\PY{c+c1}{\PYZsh{}Dla k=10\PYZca{}(7)}
\PY{n}{expected\PYZus{}value\PYZus{}k5} \PY{o}{=} \PY{n}{describe}\PY{p}{(}\PY{n}{samples\PYZus{}k\PYZus{}7}\PY{p}{)}\PY{o}{.}\PY{n}{mean}
\end{Verbatim}
\end{tcolorbox}

    \hypertarget{obliczenie-wariancji}{%
\subsubsection{Obliczenie wariancji}\label{obliczenie-wariancji}}

Wariancja - w procesach stochastycznych to miara zmienności losowej
zmiennej w czasie. Jest to średnia arytmetyczna kwadratów odchyleń
wartości losowej zmiennej od jej wartości oczekiwanej w ciągu
określonego czasu. Innymi słowy, wariancja procesu stochastycznego
mierzy, jak bardzo zmieniają się wartości zmiennej losowej w czasie, i
określa, jak bardzo trajektoria procesu różni się od średniej
trajektorii. Im większa wariancja, tym większa zmienność w trajektorii
procesu, a tym samym większa szansa na wystąpienie dużych odchyleń od
wartości oczekiwanej. Wariancja jest jednym z podstawowych parametrów
charakteryzujących proces stochastyczny i jest istotnym narzędziem w
analizie i modelowaniu procesów losowych. W praktyce, często używa się
także odchylenia standardowego, które jest pierwiastkiem kwadratowym z
wariancji i wyraża się w tych samych jednostkach co zmienna losowa.

\[ \operatorname{Var}(X) = E\left[(X - E(X))^2\right] = \int_{-\infty}^{\infty} (x - E(X))^2 f(x) dx \]

    \begin{tcolorbox}[breakable, size=fbox, boxrule=1pt, pad at break*=1mm,colback=cellbackground, colframe=cellborder]
\prompt{In}{incolor}{69}{\boxspacing}
\begin{Verbatim}[commandchars=\\\{\}]
\PY{c+c1}{\PYZsh{}Dla k=10\PYZca{}(3)}
\PY{n}{variance\PYZus{}k1} \PY{o}{=} \PY{n}{describe}\PY{p}{(}\PY{n}{samples\PYZus{}k\PYZus{}3}\PY{p}{)}\PY{o}{.}\PY{n}{variance}
\PY{c+c1}{\PYZsh{}Dla k=10\PYZca{}(4)}
\PY{n}{variance\PYZus{}k2} \PY{o}{=} \PY{n}{describe}\PY{p}{(}\PY{n}{samples\PYZus{}k\PYZus{}4}\PY{p}{)}\PY{o}{.}\PY{n}{variance}
\PY{c+c1}{\PYZsh{}Dla k=10\PYZca{}(5)}
\PY{n}{variance\PYZus{}k3} \PY{o}{=} \PY{n}{describe}\PY{p}{(}\PY{n}{samples\PYZus{}k\PYZus{}5}\PY{p}{)}\PY{o}{.}\PY{n}{variance}
\PY{c+c1}{\PYZsh{}Dla k=10\PYZca{}(6)}
\PY{n}{variance\PYZus{}k4} \PY{o}{=} \PY{n}{describe}\PY{p}{(}\PY{n}{samples\PYZus{}k\PYZus{}6}\PY{p}{)}\PY{o}{.}\PY{n}{variance}
\PY{c+c1}{\PYZsh{}Dla k=10\PYZca{}(7)}
\PY{n}{variance\PYZus{}k5} \PY{o}{=} \PY{n}{describe}\PY{p}{(}\PY{n}{samples\PYZus{}k\PYZus{}7}\PY{p}{)}\PY{o}{.}\PY{n}{variance}
\end{Verbatim}
\end{tcolorbox}

    Podsumowanie obliczeń wartości oczekiwanej oraz kowariancji względem
różnych k

\begin{tabular}{llrr}
\toprule
{} &      k &  wartość oczekiwana &  wariancja \\
\midrule
0 &  k\textasciicircum (3) &            5.004410 &   0.010628 \\
1 &  k\textasciicircum (4) &            4.999147 &   0.009954 \\
2 &  k\textasciicircum (5) &            5.000310 &   0.010048 \\
3 &  k\textasciicircum (6) &            5.000073 &   0.009984 \\
4 &  k\textasciicircum (7) &            4.999993 &   0.009997 \\
\bottomrule
\end{tabular}


Analiza: 
\newline * Wartość oczekiwana (średnia) reprezentuje centralny punkt
rozkładu próbek i informuje nas o oczekiwanej wartości białego szumu. \newline
*Wariancja mierzy rozproszenie próbek wokół wartości oczekiwanej i
wskazuje na stopień zmienności białego szumu.

Wnioski: 
\newline * Dla każdej próbki białego szumu o rozkładzie Gaussowskim,
wartość oczekiwana (mean) jest bliska wartości parametru mu (5), co jest
zgodne z oczekiwaniami, ponieważ próbki zostały wygenerowane z rozkładu
N(5, 0.1). 
\newline * Wraz ze zwiększaniem wartości k (liczby próbek), wartość
oczekiwana staje się bardziej stabilna i zbliża się do wartości
parametru mu, co potwierdza centralne twierdzenie graniczne. 
\newline * Wariancja
próbek białego szumu maleje wraz ze zwiększaniem wartości k. Oznacza to,
że im większa liczba próbek, tym mniejsze jest rozproszenie i bardziej
stabilne jest zachowanie białego szumu. 
\newline * Dla każdej próbki, wariancja
jest bliska wartości parametru sigma (0.1), co jest zgodne z
oczekiwaniami, ponieważ próbki zostały wygenerowane z rozkładu N(5,
0.1).

Analiza wariancji i wartości oczekiwanej pomaga w zrozumieniu
charakterystyk białego szumu i ilustruje, jak zwiększanie liczby próbek
wpływa na dokładność aproksymacji wartości oczekiwanej i wariancji
rozkładu Gaussowskiego.

    \hypertarget{obliczenie-funkcji-kowariancyjnej}{%
\subsubsection{Obliczenie funkcji
kowariancyjnej}\label{obliczenie-funkcji-kowariancyjnej}}

Funkcja kowariancyjna - (ang. covariance function) to funkcja, która
opisuje zależność między dwiema zmiennymi losowymi X i Y. Funkcja
kowariancyjna określa, czy zmienne losowe X i Y są skorelowane (czy
zmieniają się razem) lub niezależne (czy zmieniają się niezależnie od
siebie). W przypadku zmiennych losowych niezależnych funkcja
kowariancyjna wynosi 0, a w przypadku zmiennych losowych skorelowanych
funkcja kowariancyjna może być dodatnia lub ujemna.

\[ \operatorname{Cov}(X_i, X_j) = \begin{cases} \sigma^2 & i = j \\ 0 & i \neq j \end{cases} \]

gdzie X\_i i X\_j to próbki szumu białego, a σ\^{}2 to wariancja szumu
białego.

    \begin{tcolorbox}[breakable, size=fbox, boxrule=1pt, pad at break*=1mm,colback=cellbackground, colframe=cellborder]
\prompt{In}{incolor}{71}{\boxspacing}
\begin{Verbatim}[commandchars=\\\{\}]
\PY{c+c1}{\PYZsh{} def covariance(samples, k):}
\PY{c+c1}{\PYZsh{}     \PYZsh{} Subtitute DC component noise signal from samples}
\PY{c+c1}{\PYZsh{}     subtitute\PYZus{}mu\PYZus{}value = samples \PYZhy{} mu}
\PY{c+c1}{\PYZsh{}     \PYZsh{} Compute covariance function}
\PY{c+c1}{\PYZsh{}     cov = np.correlate(subtitute\PYZus{}mu\PYZus{}value, subtitute\PYZus{}mu\PYZus{}value, mode=\PYZsq{}full\PYZsq{}) / k}
\PY{c+c1}{\PYZsh{}     return cov}

\PY{k}{def} \PY{n+nf}{covariance}\PY{p}{(}\PY{n}{samples}\PY{p}{,} \PY{n}{k}\PY{p}{)}\PY{p}{:}
    \PY{n}{substitute\PYZus{}mu\PYZus{}value} \PY{o}{=} \PY{n}{samples} \PY{o}{\PYZhy{}} \PY{n}{mu}
    \PY{n}{cov} \PY{o}{=} \PY{n}{np}\PY{o}{.}\PY{n}{fft}\PY{o}{.}\PY{n}{fftshift}\PY{p}{(}\PY{n}{np}\PY{o}{.}\PY{n}{fft}\PY{o}{.}\PY{n}{ifft}\PY{p}{(}\PY{n}{np}\PY{o}{.}\PY{n}{abs}\PY{p}{(}\PY{n}{np}\PY{o}{.}\PY{n}{fft}\PY{o}{.}\PY{n}{fft}\PY{p}{(}\PY{n}{substitute\PYZus{}mu\PYZus{}value}\PY{p}{)}\PY{p}{)}\PY{o}{*}\PY{o}{*}\PY{l+m+mi}{2} \PY{o}{/} \PY{n}{k}\PY{p}{)}\PY{p}{)}
    \PY{n}{cov} \PY{o}{=} \PY{n}{np}\PY{o}{.}\PY{n}{real}\PY{p}{(}\PY{n}{cov}\PY{p}{)}
    \PY{k}{return} \PY{n}{cov}
\end{Verbatim}
\end{tcolorbox}

    \begin{tcolorbox}[breakable, size=fbox, boxrule=1pt, pad at break*=1mm,colback=cellbackground, colframe=cellborder]
\prompt{In}{incolor}{72}{\boxspacing}
\begin{Verbatim}[commandchars=\\\{\}]
\PY{n}{fig}\PY{p}{,} \PY{n}{axarr} \PY{o}{=} \PY{n}{plt}\PY{o}{.}\PY{n}{subplots}\PY{p}{(}\PY{l+m+mi}{1}\PY{p}{,} \PY{l+m+mi}{2}\PY{p}{)}
\PY{n}{fig}\PY{o}{.}\PY{n}{set\PYZus{}figheight}\PY{p}{(}\PY{l+m+mi}{12}\PY{p}{)}
\PY{n}{fig}\PY{o}{.}\PY{n}{set\PYZus{}figwidth}\PY{p}{(}\PY{l+m+mi}{22}\PY{p}{)}
\PY{n}{fig}\PY{o}{.}\PY{n}{suptitle}\PY{p}{(}\PY{l+s+s2}{\PYZdq{}}\PY{l+s+s2}{Funkcja kowariancyjna dla różnego k}\PY{l+s+s2}{\PYZdq{}}\PY{p}{,} \PY{n}{fontsize}\PY{o}{=}\PY{l+m+mi}{16}\PY{p}{)}

\PY{n}{cov\PYZus{}k\PYZus{}3} \PY{o}{=} \PY{n}{covariance}\PY{p}{(}\PY{n}{samples\PYZus{}k\PYZus{}3}\PY{p}{,} \PY{n}{k\PYZus{}3}\PY{p}{)}
\PY{n}{cov\PYZus{}k\PYZus{}4} \PY{o}{=} \PY{n}{covariance}\PY{p}{(}\PY{n}{samples\PYZus{}k\PYZus{}4}\PY{p}{,} \PY{n}{k\PYZus{}4}\PY{p}{)}

\PY{n}{x\PYZus{}k\PYZus{}3} \PY{o}{=} \PY{n}{np}\PY{o}{.}\PY{n}{linspace}\PY{p}{(}\PY{o}{\PYZhy{}}\PY{n}{k\PYZus{}3}\PY{o}{/}\PY{o}{/}\PY{l+m+mi}{2}\PY{p}{,} \PY{n}{k\PYZus{}3}\PY{o}{/}\PY{o}{/}\PY{l+m+mi}{2}\PY{p}{,} \PY{n+nb}{len}\PY{p}{(}\PY{n}{cov\PYZus{}k\PYZus{}3}\PY{p}{)}\PY{p}{)}
\PY{n}{x\PYZus{}k\PYZus{}4} \PY{o}{=} \PY{n}{np}\PY{o}{.}\PY{n}{linspace}\PY{p}{(}\PY{o}{\PYZhy{}}\PY{n}{k\PYZus{}4}\PY{o}{/}\PY{o}{/}\PY{l+m+mi}{2}\PY{p}{,} \PY{n}{k\PYZus{}4}\PY{o}{/}\PY{o}{/}\PY{l+m+mi}{2}\PY{p}{,} \PY{n+nb}{len}\PY{p}{(}\PY{n}{cov\PYZus{}k\PYZus{}4}\PY{p}{)}\PY{p}{)}

\PY{c+c1}{\PYZsh{} Zwiększenie wartości na osi Y}
\PY{n}{y\PYZus{}max\PYZus{}k\PYZus{}3} \PY{o}{=} \PY{n}{np}\PY{o}{.}\PY{n}{max}\PY{p}{(}\PY{n}{cov\PYZus{}k\PYZus{}3}\PY{p}{)} \PY{o}{*} \PY{l+m+mf}{1.2}
\PY{n}{y\PYZus{}max\PYZus{}k\PYZus{}4} \PY{o}{=} \PY{n}{np}\PY{o}{.}\PY{n}{max}\PY{p}{(}\PY{n}{cov\PYZus{}k\PYZus{}4}\PY{p}{)} \PY{o}{*} \PY{l+m+mf}{1.2}

\PY{n}{axarr}\PY{p}{[}\PY{l+m+mi}{0}\PY{p}{]}\PY{o}{.}\PY{n}{plot}\PY{p}{(}\PY{n}{x\PYZus{}k\PYZus{}3}\PY{p}{,} \PY{n}{cov\PYZus{}k\PYZus{}3}\PY{p}{)}
\PY{n}{axarr}\PY{p}{[}\PY{l+m+mi}{0}\PY{p}{]}\PY{o}{.}\PY{n}{set\PYZus{}title}\PY{p}{(}\PY{l+s+s1}{\PYZsq{}}\PY{l+s+s1}{k=10\PYZca{}(3)}\PY{l+s+s1}{\PYZsq{}}\PY{p}{)}
\PY{n}{axarr}\PY{p}{[}\PY{l+m+mi}{0}\PY{p}{]}\PY{o}{.}\PY{n}{set\PYZus{}xlim}\PY{p}{(}\PY{o}{\PYZhy{}}\PY{n}{k\PYZus{}3}\PY{o}{/}\PY{o}{/}\PY{l+m+mi}{100}\PY{p}{,} \PY{n}{k\PYZus{}3}\PY{o}{/}\PY{o}{/}\PY{l+m+mi}{100}\PY{p}{)}
\PY{n}{axarr}\PY{p}{[}\PY{l+m+mi}{0}\PY{p}{]}\PY{o}{.}\PY{n}{set\PYZus{}ylim}\PY{p}{(}\PY{n}{np}\PY{o}{.}\PY{n}{min}\PY{p}{(}\PY{n}{cov\PYZus{}k\PYZus{}3}\PY{p}{)}\PY{p}{,} \PY{n}{y\PYZus{}max\PYZus{}k\PYZus{}3}\PY{p}{)}

\PY{n}{axarr}\PY{p}{[}\PY{l+m+mi}{1}\PY{p}{]}\PY{o}{.}\PY{n}{plot}\PY{p}{(}\PY{n}{x\PYZus{}k\PYZus{}4}\PY{p}{,} \PY{n}{cov\PYZus{}k\PYZus{}4}\PY{p}{)}
\PY{n}{axarr}\PY{p}{[}\PY{l+m+mi}{1}\PY{p}{]}\PY{o}{.}\PY{n}{set\PYZus{}title}\PY{p}{(}\PY{l+s+s1}{\PYZsq{}}\PY{l+s+s1}{k=10\PYZca{}(4)}\PY{l+s+s1}{\PYZsq{}}\PY{p}{)}
\PY{n}{axarr}\PY{p}{[}\PY{l+m+mi}{1}\PY{p}{]}\PY{o}{.}\PY{n}{set\PYZus{}xlim}\PY{p}{(}\PY{o}{\PYZhy{}}\PY{n}{k\PYZus{}4}\PY{o}{/}\PY{o}{/}\PY{l+m+mi}{500}\PY{p}{,} \PY{n}{k\PYZus{}4}\PY{o}{/}\PY{o}{/}\PY{l+m+mi}{500}\PY{p}{)}
\PY{c+c1}{\PYZsh{} axarr[1].set\PYZus{}ylim(np.min(cov\PYZus{}k\PYZus{}4\PYZus{}smooth), y\PYZus{}max\PYZus{}k\PYZus{}4)}

\PY{n}{fig}\PY{o}{.}\PY{n}{tight\PYZus{}layout}\PY{p}{(}\PY{p}{)}
\PY{n}{fig}\PY{o}{.}\PY{n}{subplots\PYZus{}adjust}\PY{p}{(}\PY{n}{top}\PY{o}{=}\PY{l+m+mf}{0.92}\PY{p}{)}

\PY{n}{plt}\PY{o}{.}\PY{n}{show}\PY{p}{(}\PY{p}{)}
\end{Verbatim}
\end{tcolorbox}

    \begin{center}
    \adjustimage{max size={0.9\linewidth}{0.9\paperheight}}{main_files/main_45_0.png}
    \end{center}
    { \hspace*{\fill} \\}
    
    \begin{enumerate}
\def\labelenumi{\arabic{enumi}.}
\tightlist
\item
  Wzrost wartości k: Wraz ze wzrostem wartości k, czyli liczby próbek
  generowanych do obliczenia funkcji kowariancji, obserwuje się większą
  precyzję estymacji tej funkcji. Dla większych wartości k, funkcja
  kowariancji jest bardziej gładka i dokładniejsza, co pozwala na lepsze
  zrozumienie zależności między próbkami szumu.
\item
  Charakterystyka funkcji kowariancji: W przypadku szumu białego o
  rozkładzie Gaussa, funkcja kowariancji powinna być równa 0 dla
  wszystkich wartości przesunięcia czasowego (opóźnienia). Oznacza to
  brak zależności między próbkami szumu w różnych momentach czasowych.
  Na wykresach możemy zauważyć, że funkcja kowariancji oscyluje wokół
  wartości 0 i nie wykazuje żadnego wyraźnego trendu.
\item
  Oś czasu: Oś czasu na wykresach reprezentuje przesunięcie czasowe
  między próbkami szumu. Na pierwszym wykresie, gdzie k=10\^{}3, oś
  czasu jest ograniczona do bardzo małego zakresu, co umożliwia dokładne
  zbadanie krótkotrwałych zależności między próbkami. Na drugim
  wykresie, gdzie k=10\^{}4, oś czasu jest bardziej rozciągnięta, co
  pozwala na analizę zależności na większą skalę czasową.
\item
  Skalowanie wykresu: Aby lepiej zobaczyć szczegóły funkcji kowariancji,
  wartości na osi Y zostały zwiększone w celu wyeksponowania
  ewentualnych małych różnic i fluktuacji. Skalowanie wykresu pomaga w
  identyfikacji subtelnych zmian w funkcji kowariancji.
\end{enumerate}

Podsumowując, funkcja kowariancji dla szumu białego o rozkładzie Gaussa
N(5,0.1) jest charakterystyczna, oscylując wokół wartości 0 bez
wyraźnego trendu. Zwiększenie wartości k prowadzi do dokładniejszej
estymacji funkcji kowariancji. Analiza funkcji kowariancji pozwala na
zrozumienie zależności między próbkami szumu w różnych punktach
czasowych.

    \hypertarget{przeprowadziux107-filtracjux119-danych-z-wykorzystaniem-filtru-dolnoprzepustowego-fir-soj-o-ruxf3ux17cnych-parametrach}{%
\subsection{3. Przeprowadzić filtrację danych z wykorzystaniem filtru
dolnoprzepustowego FIR (SOJ) o różnych
parametrach}\label{przeprowadziux107-filtracjux119-danych-z-wykorzystaniem-filtru-dolnoprzepustowego-fir-soj-o-ruxf3ux17cnych-parametrach}}

Rozdział ten skupia się na analizie i zastosowaniu filtrów
dolnoprzepustowych FIR (Skończonej Impulsowej Odpowiedzi) w celu
redukcji szumu białego o rozkładzie gaussowskim N(5,0.1) z sygnałów
danych. Szum biały jest powszechnym rodzajem zakłóceń występujących w
sygnałach i charakteryzuje się równomiernym rozkładem energii na
wszystkich częstotliwościach. Filtry dolnoprzepustowe FIR są efektywnym
narzędziem do eliminacji wysokoczęstotliwościowego szumu, pozostawiając
jednocześnie sygnał o niskich częstotliwościach nienaruszonym.

W naszym przypadku, szum biały ma rozkład gaussowski N(5,0.1), co
oznacza, że jego wartość oczekiwana wynosi 5, a odchylenie standardowe
wynosi 0.1. Celem tego rozdziału jest zastosowanie filtrów
dolnoprzepustowych FIR w celu redukcji tego szumu z sygnałów danych, aby
poprawić jakość i czytelność sygnału.

Podczas eksperymentów będziemy badać różne parametry filtru
dolnoprzepustowego FIR, takie jak długość okna filtracyjnego,
częstotliwość odcięcia i charakterystyka zwrotna. Przeanalizujemy wpływ
tych parametrów na skuteczność redukcji szumu oraz na zachowanie sygnału
oryginalnego.

W trakcie analizy porównamy wyniki filtracji dla różnych wartości
parametrów i ocenimy, jak filtr dolnoprzepustowy FIR wpływa na sygnał,
redukując szum biały o rozkładzie gaussowskim. Zbadamy zarówno
charakterystykę czasową, jak i częstotliwościową przefiltrowanego
sygnału, aby ocenić, jak efektywnie filtr redukuje szum i zachowuje
istotne informacje sygnału.

Poznanie wpływu różnych parametrów filtru dolnoprzepustowego FIR na
redukcję szumu białego o rozkładzie gaussowskim pozwoli na bardziej
skuteczną filtrację sygnałów w obecności tego rodzaju zakłóceń. Pozwoli
to na poprawę jakości sygnałów w różnorodnych dziedzinach, takich jak
przetwarzanie dźwięku, obrazu, sygnałów pomiarowych itp.

    \begin{tcolorbox}[breakable, size=fbox, boxrule=1pt, pad at break*=1mm,colback=cellbackground, colframe=cellborder]
\prompt{In}{incolor}{73}{\boxspacing}
\begin{Verbatim}[commandchars=\\\{\}]
\PY{k}{def} \PY{n+nf}{calculate\PYZus{}impulse\PYZus{}response}\PY{p}{(}\PY{n}{numtaps}\PY{p}{,} \PY{n+nb}{filter}\PY{p}{)}\PY{p}{:}
    \PY{c+c1}{\PYZsh{} Tworzenie impulsu jednostkowego}
    \PY{n}{impulse} \PY{o}{=} \PY{n}{np}\PY{o}{.}\PY{n}{zeros}\PY{p}{(}\PY{n}{numtaps}\PY{p}{)}
    \PY{n}{impulse}\PY{p}{[}\PY{n}{numtaps} \PY{o}{/}\PY{o}{/} \PY{l+m+mi}{2}\PY{p}{]} \PY{o}{=} \PY{l+m+mi}{1}
    
    \PY{c+c1}{\PYZsh{} Obliczanie odpowiedzi impulsowej filtru FIR}
    \PY{n}{impulse\PYZus{}response} \PY{o}{=} \PY{n}{signal}\PY{o}{.}\PY{n}{convolve}\PY{p}{(}\PY{n}{impulse}\PY{p}{,} \PY{n+nb}{filter}\PY{p}{,} \PY{n}{mode}\PY{o}{=}\PY{l+s+s1}{\PYZsq{}}\PY{l+s+s1}{full}\PY{l+s+s1}{\PYZsq{}}\PY{p}{)}
    
    \PY{k}{return} \PY{n}{impulse\PYZus{}response}

\PY{k}{def} \PY{n+nf}{filtration}\PY{p}{(}\PY{n}{noise}\PY{p}{,} \PY{n}{cutoff\PYZus{}param}\PY{p}{)}\PY{p}{:}
    \PY{c+c1}{\PYZsh{} definicja filtra dolnoprzepustowego FIR (SOJ) o różnych parametrach}
    \PY{n}{fs} \PY{o}{=} \PY{l+m+mi}{5000}  \PY{c+c1}{\PYZsh{} częstotliwość próbkowania 5 kHz}
    \PY{n}{filter1} \PY{o}{=} \PY{n}{signal}\PY{o}{.}\PY{n}{firwin}\PY{p}{(}\PY{n}{numtaps}\PY{o}{=}\PY{l+m+mi}{11}\PY{p}{,} \PY{n}{cutoff}\PY{o}{=}\PY{n}{cutoff\PYZus{}param}\PY{p}{,} \PY{n}{fs}\PY{o}{=}\PY{n}{fs}\PY{p}{,} \PY{n}{window}\PY{o}{=}\PY{l+s+s1}{\PYZsq{}}\PY{l+s+s1}{hamming}\PY{l+s+s1}{\PYZsq{}}\PY{p}{)}
    \PY{n}{filter2} \PY{o}{=} \PY{n}{signal}\PY{o}{.}\PY{n}{firwin}\PY{p}{(}\PY{n}{numtaps}\PY{o}{=}\PY{l+m+mi}{31}\PY{p}{,} \PY{n}{cutoff}\PY{o}{=}\PY{n}{cutoff\PYZus{}param}\PY{p}{,} \PY{n}{fs}\PY{o}{=}\PY{n}{fs}\PY{p}{,} \PY{n}{window}\PY{o}{=}\PY{l+s+s1}{\PYZsq{}}\PY{l+s+s1}{hamming}\PY{l+s+s1}{\PYZsq{}}\PY{p}{)}
    \PY{n}{filter3} \PY{o}{=} \PY{n}{signal}\PY{o}{.}\PY{n}{firwin}\PY{p}{(}\PY{n}{numtaps}\PY{o}{=}\PY{l+m+mi}{51}\PY{p}{,} \PY{n}{cutoff}\PY{o}{=}\PY{n}{cutoff\PYZus{}param}\PY{p}{,} \PY{n}{fs}\PY{o}{=}\PY{n}{fs}\PY{p}{,} \PY{n}{window}\PY{o}{=}\PY{l+s+s1}{\PYZsq{}}\PY{l+s+s1}{hamming}\PY{l+s+s1}{\PYZsq{}}\PY{p}{)}

    \PY{c+c1}{\PYZsh{} przeprowadzenie filtracji danych z wykorzystaniem filtrów}
    \PY{n}{filtered1} \PY{o}{=} \PY{n}{signal}\PY{o}{.}\PY{n}{lfilter}\PY{p}{(}\PY{n}{filter1}\PY{p}{,} \PY{l+m+mi}{1}\PY{p}{,} \PY{n}{noise}\PY{p}{)}
    \PY{n}{filtered2} \PY{o}{=} \PY{n}{signal}\PY{o}{.}\PY{n}{lfilter}\PY{p}{(}\PY{n}{filter2}\PY{p}{,} \PY{l+m+mi}{1}\PY{p}{,} \PY{n}{noise}\PY{p}{)}
    \PY{n}{filtered3} \PY{o}{=} \PY{n}{signal}\PY{o}{.}\PY{n}{lfilter}\PY{p}{(}\PY{n}{filter3}\PY{p}{,} \PY{l+m+mi}{1}\PY{p}{,} \PY{n}{noise}\PY{p}{)}

    \PY{n}{fig}\PY{p}{,} \PY{n}{axs} \PY{o}{=} \PY{n}{plt}\PY{o}{.}\PY{n}{subplots}\PY{p}{(}\PY{l+m+mi}{3}\PY{p}{,} \PY{l+m+mi}{2}\PY{p}{,} \PY{n}{figsize}\PY{o}{=}\PY{p}{(}\PY{l+m+mi}{15}\PY{p}{,} \PY{l+m+mi}{8}\PY{p}{)}\PY{p}{)}

    \PY{n}{axs}\PY{p}{[}\PY{l+m+mi}{0}\PY{p}{,} \PY{l+m+mi}{0}\PY{p}{]}\PY{o}{.}\PY{n}{plot}\PY{p}{(}\PY{n}{noise}\PY{p}{,} \PY{n}{label}\PY{o}{=}\PY{l+s+s1}{\PYZsq{}}\PY{l+s+s1}{Szum biały}\PY{l+s+s1}{\PYZsq{}}\PY{p}{)}
    \PY{n}{axs}\PY{p}{[}\PY{l+m+mi}{0}\PY{p}{,} \PY{l+m+mi}{0}\PY{p}{]}\PY{o}{.}\PY{n}{plot}\PY{p}{(}\PY{n}{filtered1}\PY{p}{,} \PY{n}{label}\PY{o}{=}\PY{l+s+s1}{\PYZsq{}}\PY{l+s+s1}{Filtr dolnoprzepustowy FIR (SOJ) o 11 punktach}\PY{l+s+s1}{\PYZsq{}}\PY{p}{)}
    \PY{n}{axs}\PY{p}{[}\PY{l+m+mi}{0}\PY{p}{,} \PY{l+m+mi}{0}\PY{p}{]}\PY{o}{.}\PY{n}{set\PYZus{}title}\PY{p}{(}\PY{l+s+s2}{\PYZdq{}}\PY{l+s+s2}{Filtracja filtrem dolnoprzepustowym FIR (SOJ) o 11 punktach}\PY{l+s+s2}{\PYZdq{}}\PY{p}{)}
    \PY{n}{axs}\PY{p}{[}\PY{l+m+mi}{0}\PY{p}{,} \PY{l+m+mi}{0}\PY{p}{]}\PY{o}{.}\PY{n}{legend}\PY{p}{(}\PY{p}{)}

    \PY{n}{axs}\PY{p}{[}\PY{l+m+mi}{1}\PY{p}{,} \PY{l+m+mi}{0}\PY{p}{]}\PY{o}{.}\PY{n}{plot}\PY{p}{(}\PY{n}{noise}\PY{p}{,} \PY{n}{label}\PY{o}{=}\PY{l+s+s1}{\PYZsq{}}\PY{l+s+s1}{Szum biały}\PY{l+s+s1}{\PYZsq{}}\PY{p}{)}
    \PY{n}{axs}\PY{p}{[}\PY{l+m+mi}{1}\PY{p}{,} \PY{l+m+mi}{0}\PY{p}{]}\PY{o}{.}\PY{n}{plot}\PY{p}{(}\PY{n}{filtered2}\PY{p}{,} \PY{n}{label}\PY{o}{=}\PY{l+s+s1}{\PYZsq{}}\PY{l+s+s1}{Filtr dolnoprzepustowy FIR (SOJ) o 31 punktach}\PY{l+s+s1}{\PYZsq{}}\PY{p}{)}
    \PY{n}{axs}\PY{p}{[}\PY{l+m+mi}{1}\PY{p}{,} \PY{l+m+mi}{0}\PY{p}{]}\PY{o}{.}\PY{n}{set\PYZus{}title}\PY{p}{(}\PY{l+s+s2}{\PYZdq{}}\PY{l+s+s2}{Filtracja filtrem dolnoprzepustowym FIR (SOJ) o 31 punktach}\PY{l+s+s2}{\PYZdq{}}\PY{p}{)}
    \PY{n}{axs}\PY{p}{[}\PY{l+m+mi}{1}\PY{p}{,} \PY{l+m+mi}{0}\PY{p}{]}\PY{o}{.}\PY{n}{legend}\PY{p}{(}\PY{p}{)}

    \PY{n}{axs}\PY{p}{[}\PY{l+m+mi}{2}\PY{p}{,} \PY{l+m+mi}{0}\PY{p}{]}\PY{o}{.}\PY{n}{plot}\PY{p}{(}\PY{n}{noise}\PY{p}{,} \PY{n}{label}\PY{o}{=}\PY{l+s+s1}{\PYZsq{}}\PY{l+s+s1}{Szum biały}\PY{l+s+s1}{\PYZsq{}}\PY{p}{)}
    \PY{n}{axs}\PY{p}{[}\PY{l+m+mi}{2}\PY{p}{,} \PY{l+m+mi}{0}\PY{p}{]}\PY{o}{.}\PY{n}{plot}\PY{p}{(}\PY{n}{filtered3}\PY{p}{,} \PY{n}{label}\PY{o}{=}\PY{l+s+s1}{\PYZsq{}}\PY{l+s+s1}{Filtr dolnoprzepustowy FIR (SOJ) o 51 punktach}\PY{l+s+s1}{\PYZsq{}}\PY{p}{)}
    \PY{n}{axs}\PY{p}{[}\PY{l+m+mi}{2}\PY{p}{,} \PY{l+m+mi}{0}\PY{p}{]}\PY{o}{.}\PY{n}{set\PYZus{}title}\PY{p}{(}\PY{l+s+s2}{\PYZdq{}}\PY{l+s+s2}{Filtracja filtrem dolnoprzepustowym FIR (SOJ) o 51 punktach}\PY{l+s+s2}{\PYZdq{}}\PY{p}{)}
    \PY{n}{axs}\PY{p}{[}\PY{l+m+mi}{2}\PY{p}{,} \PY{l+m+mi}{0}\PY{p}{]}\PY{o}{.}\PY{n}{legend}\PY{p}{(}\PY{p}{)}
    
    \PY{n}{impulse\PYZus{}response1} \PY{o}{=} \PY{n}{calculate\PYZus{}impulse\PYZus{}response}\PY{p}{(}\PY{l+m+mi}{11}\PY{p}{,} \PY{n}{filter1}\PY{p}{)}
    \PY{n}{axs}\PY{p}{[}\PY{l+m+mi}{0}\PY{p}{,} \PY{l+m+mi}{1}\PY{p}{]}\PY{o}{.}\PY{n}{stem}\PY{p}{(}\PY{n}{impulse\PYZus{}response1}\PY{p}{)}
    \PY{n}{axs}\PY{p}{[}\PY{l+m+mi}{0}\PY{p}{,} \PY{l+m+mi}{1}\PY{p}{]}\PY{o}{.}\PY{n}{set\PYZus{}title}\PY{p}{(}\PY{l+s+s2}{\PYZdq{}}\PY{l+s+s2}{Odpowiedź impulsowa filtru FIR (SOJ) o 11 punktach}\PY{l+s+s2}{\PYZdq{}}\PY{p}{)}
    
    \PY{n}{impulse\PYZus{}response2} \PY{o}{=} \PY{n}{calculate\PYZus{}impulse\PYZus{}response}\PY{p}{(}\PY{l+m+mi}{31}\PY{p}{,} \PY{n}{filter2}\PY{p}{)}
    \PY{n}{axs}\PY{p}{[}\PY{l+m+mi}{1}\PY{p}{,} \PY{l+m+mi}{1}\PY{p}{]}\PY{o}{.}\PY{n}{stem}\PY{p}{(}\PY{n}{impulse\PYZus{}response2}\PY{p}{)}
    \PY{n}{axs}\PY{p}{[}\PY{l+m+mi}{1}\PY{p}{,} \PY{l+m+mi}{1}\PY{p}{]}\PY{o}{.}\PY{n}{set\PYZus{}title}\PY{p}{(}\PY{l+s+s2}{\PYZdq{}}\PY{l+s+s2}{Odpowiedź impulsowa filtru FIR (SOJ) o 31 punktach}\PY{l+s+s2}{\PYZdq{}}\PY{p}{)}
    
    \PY{n}{impulse\PYZus{}response3} \PY{o}{=} \PY{n}{calculate\PYZus{}impulse\PYZus{}response}\PY{p}{(}\PY{l+m+mi}{51}\PY{p}{,} \PY{n}{filter3}\PY{p}{)}
    \PY{n}{axs}\PY{p}{[}\PY{l+m+mi}{2}\PY{p}{,} \PY{l+m+mi}{1}\PY{p}{]}\PY{o}{.}\PY{n}{stem}\PY{p}{(}\PY{n}{impulse\PYZus{}response3}\PY{p}{)}
    \PY{n}{axs}\PY{p}{[}\PY{l+m+mi}{2}\PY{p}{,} \PY{l+m+mi}{1}\PY{p}{]}\PY{o}{.}\PY{n}{set\PYZus{}title}\PY{p}{(}\PY{l+s+s2}{\PYZdq{}}\PY{l+s+s2}{Odpowiedź impulsowa filtru FIR (SOJ) o 51 punktach}\PY{l+s+s2}{\PYZdq{}}\PY{p}{)}
    
    \PY{n}{fig}\PY{o}{.}\PY{n}{tight\PYZus{}layout}\PY{p}{(}\PY{p}{)}
    
    \PY{n}{plt}\PY{o}{.}\PY{n}{show}\PY{p}{(}\PY{p}{)}

    \PY{k}{return} \PY{n}{filtered3}
\end{Verbatim}
\end{tcolorbox}

    \hypertarget{dla-k103}{%
\subsubsection{Dla k=10\^{}(3)}\label{dla-k103}}

    \hypertarget{fcutoff300hz}{%
\paragraph{fcutoff=300Hz}\label{fcutoff300hz}}

    \begin{tcolorbox}[breakable, size=fbox, boxrule=1pt, pad at break*=1mm,colback=cellbackground, colframe=cellborder]
\prompt{In}{incolor}{74}{\boxspacing}
\begin{Verbatim}[commandchars=\\\{\}]
\PY{n}{noise\PYZus{}after\PYZus{}filtration\PYZus{}k3\PYZus{}fc300} \PY{o}{=} \PY{n}{filtration}\PY{p}{(}\PY{n}{samples\PYZus{}k\PYZus{}3}\PY{p}{,} \PY{l+m+mi}{300}\PY{p}{)}
\PY{n}{display\PYZus{}hist}\PY{p}{(}\PY{n}{noise\PYZus{}after\PYZus{}filtration\PYZus{}k3\PYZus{}fc300}\PY{p}{,} \PY{n}{k\PYZus{}3}\PY{p}{)}
\end{Verbatim}
\end{tcolorbox}

    \begin{center}
    \adjustimage{max size={0.9\linewidth}{0.9\paperheight}}{main_files/main_51_0.png}
    \end{center}
    { \hspace*{\fill} \\}
    
    \begin{center}
    \adjustimage{max size={0.9\linewidth}{0.9\paperheight}}{main_files/main_51_1.png}
    \end{center}
    { \hspace*{\fill} \\}
    
    \hypertarget{fcutoff900hz}{%
\paragraph{fcutoff=900Hz}\label{fcutoff900hz}}

    \begin{tcolorbox}[breakable, size=fbox, boxrule=1pt, pad at break*=1mm,colback=cellbackground, colframe=cellborder]
\prompt{In}{incolor}{75}{\boxspacing}
\begin{Verbatim}[commandchars=\\\{\}]
\PY{n}{noise\PYZus{}after\PYZus{}filtration\PYZus{}k3\PYZus{}fc900} \PY{o}{=} \PY{n}{filtration}\PY{p}{(}\PY{n}{samples\PYZus{}k\PYZus{}3}\PY{p}{,} \PY{l+m+mi}{900}\PY{p}{)}
\PY{n}{display\PYZus{}hist}\PY{p}{(}\PY{n}{noise\PYZus{}after\PYZus{}filtration\PYZus{}k3\PYZus{}fc900}\PY{p}{,} \PY{n}{k\PYZus{}3}\PY{p}{)}
\end{Verbatim}
\end{tcolorbox}

    \begin{center}
    \adjustimage{max size={0.9\linewidth}{0.9\paperheight}}{main_files/main_53_0.png}
    \end{center}
    { \hspace*{\fill} \\}
    
    \begin{center}
    \adjustimage{max size={0.9\linewidth}{0.9\paperheight}}{main_files/main_53_1.png}
    \end{center}
    { \hspace*{\fill} \\}
    
    \hypertarget{fcutoff1600hz}{%
\paragraph{fcutoff=1600Hz}\label{fcutoff1600hz}}

    \begin{tcolorbox}[breakable, size=fbox, boxrule=1pt, pad at break*=1mm,colback=cellbackground, colframe=cellborder]
\prompt{In}{incolor}{76}{\boxspacing}
\begin{Verbatim}[commandchars=\\\{\}]
\PY{n}{noise\PYZus{}after\PYZus{}filtration\PYZus{}k3\PYZus{}fc1600} \PY{o}{=} \PY{n}{filtration}\PY{p}{(}\PY{n}{samples\PYZus{}k\PYZus{}3}\PY{p}{,} \PY{l+m+mi}{1600}\PY{p}{)}
\PY{n}{display\PYZus{}hist}\PY{p}{(}\PY{n}{noise\PYZus{}after\PYZus{}filtration\PYZus{}k3\PYZus{}fc1600}\PY{p}{,} \PY{n}{k\PYZus{}3}\PY{p}{)}
\end{Verbatim}
\end{tcolorbox}

    \begin{center}
    \adjustimage{max size={0.9\linewidth}{0.9\paperheight}}{main_files/main_55_0.png}
    \end{center}
    { \hspace*{\fill} \\}
    
    \begin{center}
    \adjustimage{max size={0.9\linewidth}{0.9\paperheight}}{main_files/main_55_1.png}
    \end{center}
    { \hspace*{\fill} \\}
    
    \hypertarget{dla-k104}{%
\subsubsection{Dla k=10\^{}(4)}\label{dla-k104}}

    \hypertarget{fcutoff300}{%
\paragraph{fcutoff=300}\label{fcutoff300}}

    \begin{tcolorbox}[breakable, size=fbox, boxrule=1pt, pad at break*=1mm,colback=cellbackground, colframe=cellborder]
\prompt{In}{incolor}{77}{\boxspacing}
\begin{Verbatim}[commandchars=\\\{\}]
\PY{n}{noise\PYZus{}after\PYZus{}filtration\PYZus{}k4\PYZus{}fc300} \PY{o}{=} \PY{n}{filtration}\PY{p}{(}\PY{n}{samples\PYZus{}k\PYZus{}4}\PY{p}{,} \PY{l+m+mi}{300}\PY{p}{)}
\PY{n}{display\PYZus{}hist}\PY{p}{(}\PY{n}{noise\PYZus{}after\PYZus{}filtration\PYZus{}k4\PYZus{}fc300}\PY{p}{,} \PY{n}{k\PYZus{}4}\PY{p}{)}
\end{Verbatim}
\end{tcolorbox}

    \begin{center}
    \adjustimage{max size={0.9\linewidth}{0.9\paperheight}}{main_files/main_58_0.png}
    \end{center}
    { \hspace*{\fill} \\}
    
    \begin{center}
    \adjustimage{max size={0.9\linewidth}{0.9\paperheight}}{main_files/main_58_1.png}
    \end{center}
    { \hspace*{\fill} \\}
    
    \hypertarget{fcutoff900}{%
\paragraph{fcutoff=900}\label{fcutoff900}}

    \begin{tcolorbox}[breakable, size=fbox, boxrule=1pt, pad at break*=1mm,colback=cellbackground, colframe=cellborder]
\prompt{In}{incolor}{78}{\boxspacing}
\begin{Verbatim}[commandchars=\\\{\}]
\PY{n}{noise\PYZus{}after\PYZus{}filtration\PYZus{}k4\PYZus{}fc900} \PY{o}{=} \PY{n}{filtration}\PY{p}{(}\PY{n}{samples\PYZus{}k\PYZus{}4}\PY{p}{,} \PY{l+m+mi}{900}\PY{p}{)}
\PY{n}{display\PYZus{}hist}\PY{p}{(}\PY{n}{noise\PYZus{}after\PYZus{}filtration\PYZus{}k4\PYZus{}fc900}\PY{p}{,} \PY{n}{k\PYZus{}4}\PY{p}{)}
\end{Verbatim}
\end{tcolorbox}

    \begin{center}
    \adjustimage{max size={0.9\linewidth}{0.9\paperheight}}{main_files/main_60_0.png}
    \end{center}
    { \hspace*{\fill} \\}
    
    \begin{center}
    \adjustimage{max size={0.9\linewidth}{0.9\paperheight}}{main_files/main_60_1.png}
    \end{center}
    { \hspace*{\fill} \\}
    
    \hypertarget{fcutoff1600}{%
\paragraph{fcutoff=1600}\label{fcutoff1600}}

    \begin{tcolorbox}[breakable, size=fbox, boxrule=1pt, pad at break*=1mm,colback=cellbackground, colframe=cellborder]
\prompt{In}{incolor}{79}{\boxspacing}
\begin{Verbatim}[commandchars=\\\{\}]
\PY{n}{noise\PYZus{}after\PYZus{}filtration\PYZus{}k4\PYZus{}fc1600} \PY{o}{=} \PY{n}{filtration}\PY{p}{(}\PY{n}{samples\PYZus{}k\PYZus{}4}\PY{p}{,} \PY{l+m+mi}{1600}\PY{p}{)}
\PY{n}{display\PYZus{}hist}\PY{p}{(}\PY{n}{noise\PYZus{}after\PYZus{}filtration\PYZus{}k4\PYZus{}fc1600}\PY{p}{,} \PY{n}{k\PYZus{}4}\PY{p}{)}
\end{Verbatim}
\end{tcolorbox}

    \begin{center}
    \adjustimage{max size={0.9\linewidth}{0.9\paperheight}}{main_files/main_62_0.png}
    \end{center}
    { \hspace*{\fill} \\}
    
    \begin{center}
    \adjustimage{max size={0.9\linewidth}{0.9\paperheight}}{main_files/main_62_1.png}
    \end{center}
    { \hspace*{\fill} \\}
    
    \hypertarget{dla-k105}{%
\subsubsection{Dla k=10\^{}(5)}\label{dla-k105}}

    \hypertarget{fcutoff300}{%
\paragraph{fcutoff=300}\label{fcutoff300}}

    \begin{tcolorbox}[breakable, size=fbox, boxrule=1pt, pad at break*=1mm,colback=cellbackground, colframe=cellborder]
\prompt{In}{incolor}{80}{\boxspacing}
\begin{Verbatim}[commandchars=\\\{\}]
\PY{n}{noise\PYZus{}after\PYZus{}filtration\PYZus{}k5\PYZus{}fc300} \PY{o}{=} \PY{n}{filtration}\PY{p}{(}\PY{n}{samples\PYZus{}k\PYZus{}5}\PY{p}{,} \PY{l+m+mi}{300}\PY{p}{)}
\PY{n}{display\PYZus{}hist}\PY{p}{(}\PY{n}{noise\PYZus{}after\PYZus{}filtration\PYZus{}k5\PYZus{}fc300}\PY{p}{,} \PY{n}{k\PYZus{}5}\PY{p}{)}
\end{Verbatim}
\end{tcolorbox}

    \begin{center}
    \adjustimage{max size={0.9\linewidth}{0.9\paperheight}}{main_files/main_65_0.png}
    \end{center}
    { \hspace*{\fill} \\}
    
    \begin{center}
    \adjustimage{max size={0.9\linewidth}{0.9\paperheight}}{main_files/main_65_1.png}
    \end{center}
    { \hspace*{\fill} \\}
    
    \hypertarget{fcutoff900}{%
\paragraph{fcutoff=900}\label{fcutoff900}}

    \begin{tcolorbox}[breakable, size=fbox, boxrule=1pt, pad at break*=1mm,colback=cellbackground, colframe=cellborder]
\prompt{In}{incolor}{81}{\boxspacing}
\begin{Verbatim}[commandchars=\\\{\}]
\PY{n}{noise\PYZus{}after\PYZus{}filtration\PYZus{}k5\PYZus{}fc900} \PY{o}{=} \PY{n}{filtration}\PY{p}{(}\PY{n}{samples\PYZus{}k\PYZus{}5}\PY{p}{,} \PY{l+m+mi}{900}\PY{p}{)}
\PY{n}{display\PYZus{}hist}\PY{p}{(}\PY{n}{noise\PYZus{}after\PYZus{}filtration\PYZus{}k5\PYZus{}fc900}\PY{p}{,} \PY{n}{k\PYZus{}5}\PY{p}{)}
\end{Verbatim}
\end{tcolorbox}

    \begin{center}
    \adjustimage{max size={0.9\linewidth}{0.9\paperheight}}{main_files/main_67_0.png}
    \end{center}
    { \hspace*{\fill} \\}
    
    \begin{center}
    \adjustimage{max size={0.9\linewidth}{0.9\paperheight}}{main_files/main_67_1.png}
    \end{center}
    { \hspace*{\fill} \\}
    
    \hypertarget{fcutoff1600}{%
\paragraph{fcutoff=1600}\label{fcutoff1600}}

    \begin{tcolorbox}[breakable, size=fbox, boxrule=1pt, pad at break*=1mm,colback=cellbackground, colframe=cellborder]
\prompt{In}{incolor}{82}{\boxspacing}
\begin{Verbatim}[commandchars=\\\{\}]
\PY{n}{noise\PYZus{}after\PYZus{}filtration\PYZus{}k5\PYZus{}fc1600} \PY{o}{=} \PY{n}{filtration}\PY{p}{(}\PY{n}{samples\PYZus{}k\PYZus{}5}\PY{p}{,} \PY{l+m+mi}{1600}\PY{p}{)}
\PY{n}{display\PYZus{}hist}\PY{p}{(}\PY{n}{noise\PYZus{}after\PYZus{}filtration\PYZus{}k5\PYZus{}fc1600}\PY{p}{,} \PY{n}{k\PYZus{}5}\PY{p}{)}
\end{Verbatim}
\end{tcolorbox}

    \begin{center}
    \adjustimage{max size={0.9\linewidth}{0.9\paperheight}}{main_files/main_69_0.png}
    \end{center}
    { \hspace*{\fill} \\}
    
    \begin{center}
    \adjustimage{max size={0.9\linewidth}{0.9\paperheight}}{main_files/main_69_1.png}
    \end{center}
    { \hspace*{\fill} \\}
    
    \hypertarget{dla-k106}{%
\subsubsection{Dla k=10\^{}(6)}\label{dla-k106}}

    \hypertarget{fcutoff300}{%
\paragraph{fcutoff=300}\label{fcutoff300}}

    \begin{tcolorbox}[breakable, size=fbox, boxrule=1pt, pad at break*=1mm,colback=cellbackground, colframe=cellborder]
\prompt{In}{incolor}{83}{\boxspacing}
\begin{Verbatim}[commandchars=\\\{\}]
\PY{n}{noise\PYZus{}after\PYZus{}filtration\PYZus{}k6\PYZus{}fc300} \PY{o}{=} \PY{n}{filtration}\PY{p}{(}\PY{n}{samples\PYZus{}k\PYZus{}6}\PY{p}{,} \PY{l+m+mi}{300}\PY{p}{)}
\PY{n}{display\PYZus{}hist}\PY{p}{(}\PY{n}{noise\PYZus{}after\PYZus{}filtration\PYZus{}k6\PYZus{}fc300}\PY{p}{,} \PY{n}{k\PYZus{}6}\PY{p}{)}
\end{Verbatim}
\end{tcolorbox}

    \begin{center}
    \adjustimage{max size={0.9\linewidth}{0.9\paperheight}}{main_files/main_72_0.png}
    \end{center}
    { \hspace*{\fill} \\}
    
    \begin{center}
    \adjustimage{max size={0.9\linewidth}{0.9\paperheight}}{main_files/main_72_1.png}
    \end{center}
    { \hspace*{\fill} \\}
    
    \hypertarget{fcutoff900}{%
\paragraph{fcutoff=900}\label{fcutoff900}}

    \begin{tcolorbox}[breakable, size=fbox, boxrule=1pt, pad at break*=1mm,colback=cellbackground, colframe=cellborder]
\prompt{In}{incolor}{84}{\boxspacing}
\begin{Verbatim}[commandchars=\\\{\}]
\PY{n}{noise\PYZus{}after\PYZus{}filtration\PYZus{}k6\PYZus{}fc900} \PY{o}{=} \PY{n}{filtration}\PY{p}{(}\PY{n}{samples\PYZus{}k\PYZus{}6}\PY{p}{,} \PY{l+m+mi}{900}\PY{p}{)}
\PY{n}{display\PYZus{}hist}\PY{p}{(}\PY{n}{noise\PYZus{}after\PYZus{}filtration\PYZus{}k6\PYZus{}fc900}\PY{p}{,} \PY{n}{k\PYZus{}6}\PY{p}{)}
\end{Verbatim}
\end{tcolorbox}

    \begin{center}
    \adjustimage{max size={0.9\linewidth}{0.9\paperheight}}{main_files/main_74_0.png}
    \end{center}
    { \hspace*{\fill} \\}
    
    \begin{center}
    \adjustimage{max size={0.9\linewidth}{0.9\paperheight}}{main_files/main_74_1.png}
    \end{center}
    { \hspace*{\fill} \\}
    
    \hypertarget{fcutoff1600}{%
\paragraph{fcutoff=1600}\label{fcutoff1600}}

    \begin{tcolorbox}[breakable, size=fbox, boxrule=1pt, pad at break*=1mm,colback=cellbackground, colframe=cellborder]
\prompt{In}{incolor}{85}{\boxspacing}
\begin{Verbatim}[commandchars=\\\{\}]
\PY{n}{noise\PYZus{}after\PYZus{}filtration\PYZus{}k6\PYZus{}fc1600} \PY{o}{=} \PY{n}{filtration}\PY{p}{(}\PY{n}{samples\PYZus{}k\PYZus{}6}\PY{p}{,} \PY{l+m+mi}{1600}\PY{p}{)}
\PY{n}{display\PYZus{}hist}\PY{p}{(}\PY{n}{noise\PYZus{}after\PYZus{}filtration\PYZus{}k6\PYZus{}fc1600}\PY{p}{,} \PY{n}{k\PYZus{}6}\PY{p}{)}
\end{Verbatim}
\end{tcolorbox}

    \begin{center}
    \adjustimage{max size={0.9\linewidth}{0.9\paperheight}}{main_files/main_76_0.png}
    \end{center}
    { \hspace*{\fill} \\}
    
    \begin{center}
    \adjustimage{max size={0.9\linewidth}{0.9\paperheight}}{main_files/main_76_1.png}
    \end{center}
    { \hspace*{\fill} \\}
    
    \hypertarget{dla-k107}{%
\paragraph{Dla k=10\^{}(7)}\label{dla-k107}}

    \hypertarget{fcutoff300}{%
\paragraph{fcutoff=300}\label{fcutoff300}}

    \begin{tcolorbox}[breakable, size=fbox, boxrule=1pt, pad at break*=1mm,colback=cellbackground, colframe=cellborder]
\prompt{In}{incolor}{86}{\boxspacing}
\begin{Verbatim}[commandchars=\\\{\}]
\PY{n}{noise\PYZus{}after\PYZus{}filtration\PYZus{}k7\PYZus{}fc300} \PY{o}{=} \PY{n}{filtration}\PY{p}{(}\PY{n}{samples\PYZus{}k\PYZus{}7}\PY{p}{,} \PY{l+m+mi}{300}\PY{p}{)}
\PY{n}{display\PYZus{}hist}\PY{p}{(}\PY{n}{noise\PYZus{}after\PYZus{}filtration\PYZus{}k7\PYZus{}fc300}\PY{p}{,} \PY{n}{k\PYZus{}7}\PY{p}{)}
\end{Verbatim}
\end{tcolorbox}

    \begin{center}
    \adjustimage{max size={0.9\linewidth}{0.9\paperheight}}{main_files/main_79_0.png}
    \end{center}
    { \hspace*{\fill} \\}
    
    \begin{center}
    \adjustimage{max size={0.9\linewidth}{0.9\paperheight}}{main_files/main_79_1.png}
    \end{center}
    { \hspace*{\fill} \\}
    
    \hypertarget{fcutoff900}{%
\paragraph{fcutoff=900}\label{fcutoff900}}

    \begin{tcolorbox}[breakable, size=fbox, boxrule=1pt, pad at break*=1mm,colback=cellbackground, colframe=cellborder]
\prompt{In}{incolor}{87}{\boxspacing}
\begin{Verbatim}[commandchars=\\\{\}]
\PY{n}{noise\PYZus{}after\PYZus{}filtration\PYZus{}k7\PYZus{}fc900} \PY{o}{=} \PY{n}{filtration}\PY{p}{(}\PY{n}{samples\PYZus{}k\PYZus{}7}\PY{p}{,} \PY{l+m+mi}{900}\PY{p}{)}
\PY{n}{display\PYZus{}hist}\PY{p}{(}\PY{n}{noise\PYZus{}after\PYZus{}filtration\PYZus{}k7\PYZus{}fc900}\PY{p}{,} \PY{n}{k\PYZus{}7}\PY{p}{)}
\end{Verbatim}
\end{tcolorbox}

    \begin{center}
    \adjustimage{max size={0.9\linewidth}{0.9\paperheight}}{main_files/main_81_0.png}
    \end{center}
    { \hspace*{\fill} \\}
    
    \begin{center}
    \adjustimage{max size={0.9\linewidth}{0.9\paperheight}}{main_files/main_81_1.png}
    \end{center}
    { \hspace*{\fill} \\}
    
    \hypertarget{fcutoff1600}{%
\paragraph{fcutoff=1600}\label{fcutoff1600}}

    \begin{tcolorbox}[breakable, size=fbox, boxrule=1pt, pad at break*=1mm,colback=cellbackground, colframe=cellborder]
\prompt{In}{incolor}{88}{\boxspacing}
\begin{Verbatim}[commandchars=\\\{\}]
\PY{n}{noise\PYZus{}after\PYZus{}filtration\PYZus{}k7\PYZus{}fc1600} \PY{o}{=} \PY{n}{filtration}\PY{p}{(}\PY{n}{samples\PYZus{}k\PYZus{}7}\PY{p}{,} \PY{l+m+mi}{1600}\PY{p}{)}
\PY{n}{display\PYZus{}hist}\PY{p}{(}\PY{n}{noise\PYZus{}after\PYZus{}filtration\PYZus{}k7\PYZus{}fc1600}\PY{p}{,} \PY{n}{k\PYZus{}7}\PY{p}{)}
\end{Verbatim}
\end{tcolorbox}

    \begin{center}
    \adjustimage{max size={0.9\linewidth}{0.9\paperheight}}{main_files/main_83_0.png}
    \end{center}
    { \hspace*{\fill} \\}
    
    \begin{center}
    \adjustimage{max size={0.9\linewidth}{0.9\paperheight}}{main_files/main_83_1.png}
    \end{center}
    { \hspace*{\fill} \\}
    
    \hypertarget{obliczyux107-gux119stoux15bux107-prawdopodobieux144stwa-dystrybuantux119-a-teux17c-wartoux15bux107-oczekiwanux105-wariancjux119-i-funkcjux119-kowariancyjnux105-sygnaux142u-wyjux15bciowego.-poruxf3wnaux107-wyniki-z-p-2.}{%
\subsection{4. Obliczyć gęstość prawdopodobieństwa, dystrybuantę, a też
wartość oczekiwaną, wariancję i funkcję kowariancyjną sygnału
wyjściowego. Porównać wyniki z p
2.}\label{obliczyux107-gux119stoux15bux107-prawdopodobieux144stwa-dystrybuantux119-a-teux17c-wartoux15bux107-oczekiwanux105-wariancjux119-i-funkcjux119-kowariancyjnux105-sygnaux142u-wyjux15bciowego.-poruxf3wnaux107-wyniki-z-p-2.}}

    \hypertarget{obliczenie-gux119stoux15bci-prawdopodobieux144stwa}{%
\subsubsection{Obliczenie gęstości
prawdopodobieństwa}\label{obliczenie-gux119stoux15bci-prawdopodobieux144stwa}}

\[ f(x) = \frac{1}{\sqrt{2\pi\sigma^2}} e^{-\frac{(x-\mu)^2}{2\sigma^2}} \]

Po zastosowaniu filtru dolnoprzepustowego FIR na sygnale wyjściowym,
gęstość prawdopodobieństwa uległa zmianie w porównaniu do szumu białego.
Wartości sygnału wyjściowego skupiają się wokół wartości oczekiwanej, a
rozkład jest bardziej skoncentrowany.

    \begin{tcolorbox}[breakable, size=fbox, boxrule=1pt, pad at break*=1mm,colback=cellbackground, colframe=cellborder]
\prompt{In}{incolor}{89}{\boxspacing}
\begin{Verbatim}[commandchars=\\\{\}]
\PY{n}{fig\PYZus{}2}\PY{p}{,} \PY{n}{axarr\PYZus{}2} \PY{o}{=} \PY{n}{plt}\PY{o}{.}\PY{n}{subplots}\PY{p}{(}\PY{l+m+mi}{2}\PY{p}{,} \PY{l+m+mi}{3}\PY{p}{)}
\PY{n}{fig\PYZus{}2}\PY{o}{.}\PY{n}{set\PYZus{}figheight}\PY{p}{(}\PY{l+m+mi}{12}\PY{p}{)}
\PY{n}{fig\PYZus{}2}\PY{o}{.}\PY{n}{set\PYZus{}figwidth}\PY{p}{(}\PY{l+m+mi}{20}\PY{p}{)}
\PY{n}{fig\PYZus{}2}\PY{o}{.}\PY{n}{suptitle}\PY{p}{(}\PY{l+s+s2}{\PYZdq{}}\PY{l+s+s2}{Gęstość prawdopodobieństwa dla różnego k przed filtracją}\PY{l+s+s2}{\PYZdq{}}\PY{p}{,} \PY{n}{fontsize}\PY{o}{=}\PY{l+m+mi}{16}\PY{p}{)}

\PY{n}{axarr\PYZus{}2}\PY{p}{[}\PY{l+m+mi}{0}\PY{p}{,} \PY{l+m+mi}{0}\PY{p}{]}\PY{o}{.}\PY{n}{plot}\PY{p}{(}\PY{n}{np}\PY{o}{.}\PY{n}{sort}\PY{p}{(}\PY{n}{samples\PYZus{}k\PYZus{}3}\PY{p}{)}\PY{p}{,} \PY{n}{probability\PYZus{}pdf}\PY{p}{(}\PY{n}{samples\PYZus{}k\PYZus{}3}\PY{p}{)}\PY{p}{)}
\PY{n}{axarr\PYZus{}2}\PY{p}{[}\PY{l+m+mi}{0}\PY{p}{,} \PY{l+m+mi}{0}\PY{p}{]}\PY{o}{.}\PY{n}{set\PYZus{}title}\PY{p}{(}\PY{l+s+s1}{\PYZsq{}}\PY{l+s+s1}{k=10\PYZca{}(3)}\PY{l+s+s1}{\PYZsq{}}\PY{p}{)}
\PY{n}{axarr\PYZus{}2}\PY{p}{[}\PY{l+m+mi}{0}\PY{p}{,} \PY{l+m+mi}{1}\PY{p}{]}\PY{o}{.}\PY{n}{plot}\PY{p}{(}\PY{n}{np}\PY{o}{.}\PY{n}{sort}\PY{p}{(}\PY{n}{samples\PYZus{}k\PYZus{}4}\PY{p}{)}\PY{p}{,} \PY{n}{probability\PYZus{}pdf}\PY{p}{(}\PY{n}{samples\PYZus{}k\PYZus{}4}\PY{p}{)}\PY{p}{)}
\PY{n}{axarr\PYZus{}2}\PY{p}{[}\PY{l+m+mi}{0}\PY{p}{,} \PY{l+m+mi}{1}\PY{p}{]}\PY{o}{.}\PY{n}{set\PYZus{}title}\PY{p}{(}\PY{l+s+s1}{\PYZsq{}}\PY{l+s+s1}{k=10\PYZca{}(4)}\PY{l+s+s1}{\PYZsq{}}\PY{p}{)}
\PY{n}{axarr\PYZus{}2}\PY{p}{[}\PY{l+m+mi}{0}\PY{p}{,} \PY{l+m+mi}{2}\PY{p}{]}\PY{o}{.}\PY{n}{plot}\PY{p}{(}\PY{n}{np}\PY{o}{.}\PY{n}{sort}\PY{p}{(}\PY{n}{samples\PYZus{}k\PYZus{}5}\PY{p}{)}\PY{p}{,} \PY{n}{probability\PYZus{}pdf}\PY{p}{(}\PY{n}{samples\PYZus{}k\PYZus{}5}\PY{p}{)}\PY{p}{)}
\PY{n}{axarr\PYZus{}2}\PY{p}{[}\PY{l+m+mi}{0}\PY{p}{,} \PY{l+m+mi}{2}\PY{p}{]}\PY{o}{.}\PY{n}{set\PYZus{}title}\PY{p}{(}\PY{l+s+s1}{\PYZsq{}}\PY{l+s+s1}{k=10\PYZca{}(5)}\PY{l+s+s1}{\PYZsq{}}\PY{p}{)}
\PY{n}{axarr\PYZus{}2}\PY{p}{[}\PY{l+m+mi}{1}\PY{p}{,} \PY{l+m+mi}{0}\PY{p}{]}\PY{o}{.}\PY{n}{plot}\PY{p}{(}\PY{n}{np}\PY{o}{.}\PY{n}{sort}\PY{p}{(}\PY{n}{samples\PYZus{}k\PYZus{}6}\PY{p}{)}\PY{p}{,} \PY{n}{probability\PYZus{}pdf}\PY{p}{(}\PY{n}{samples\PYZus{}k\PYZus{}6}\PY{p}{)}\PY{p}{)}
\PY{n}{axarr\PYZus{}2}\PY{p}{[}\PY{l+m+mi}{1}\PY{p}{,} \PY{l+m+mi}{0}\PY{p}{]}\PY{o}{.}\PY{n}{set\PYZus{}title}\PY{p}{(}\PY{l+s+s1}{\PYZsq{}}\PY{l+s+s1}{k=10\PYZca{}(6)}\PY{l+s+s1}{\PYZsq{}}\PY{p}{)}
\PY{n}{axarr\PYZus{}2}\PY{p}{[}\PY{l+m+mi}{1}\PY{p}{,} \PY{l+m+mi}{1}\PY{p}{]}\PY{o}{.}\PY{n}{plot}\PY{p}{(}\PY{n}{np}\PY{o}{.}\PY{n}{sort}\PY{p}{(}\PY{n}{samples\PYZus{}k\PYZus{}7}\PY{p}{)}\PY{p}{,} \PY{n}{probability\PYZus{}pdf}\PY{p}{(}\PY{n}{samples\PYZus{}k\PYZus{}7}\PY{p}{)}\PY{p}{)}
\PY{n}{axarr\PYZus{}2}\PY{p}{[}\PY{l+m+mi}{1}\PY{p}{,} \PY{l+m+mi}{1}\PY{p}{]}\PY{o}{.}\PY{n}{set\PYZus{}title}\PY{p}{(}\PY{l+s+s1}{\PYZsq{}}\PY{l+s+s1}{k=10\PYZca{}(7)}\PY{l+s+s1}{\PYZsq{}}\PY{p}{)}
\PY{n}{axarr\PYZus{}2}\PY{p}{[}\PY{l+m+mi}{1}\PY{p}{,} \PY{l+m+mi}{2}\PY{p}{]}\PY{o}{.}\PY{n}{set\PYZus{}visible}\PY{p}{(}\PY{k+kc}{False}\PY{p}{)}

\PY{c+c1}{\PYZsh{} Tight layout often produces nice results}
\PY{c+c1}{\PYZsh{} but requires the title to be spaced accordingly}
\PY{n}{fig\PYZus{}2}\PY{o}{.}\PY{n}{tight\PYZus{}layout}\PY{p}{(}\PY{p}{)}
\PY{n}{fig\PYZus{}2}\PY{o}{.}\PY{n}{subplots\PYZus{}adjust}\PY{p}{(}\PY{n}{top}\PY{o}{=}\PY{l+m+mf}{0.92}\PY{p}{)}

\PY{n}{fig\PYZus{}1}\PY{p}{,} \PY{n}{axarr\PYZus{}1} \PY{o}{=} \PY{n}{plt}\PY{o}{.}\PY{n}{subplots}\PY{p}{(}\PY{l+m+mi}{2}\PY{p}{,} \PY{l+m+mi}{3}\PY{p}{)}
\PY{n}{fig\PYZus{}1}\PY{o}{.}\PY{n}{set\PYZus{}figheight}\PY{p}{(}\PY{l+m+mi}{12}\PY{p}{)}
\PY{n}{fig\PYZus{}1}\PY{o}{.}\PY{n}{set\PYZus{}figwidth}\PY{p}{(}\PY{l+m+mi}{20}\PY{p}{)}
\PY{n}{fig\PYZus{}1}\PY{o}{.}\PY{n}{suptitle}\PY{p}{(}\PY{l+s+s2}{\PYZdq{}}\PY{l+s+s2}{Gęstość prawdopodobieństwa dla różnego k po filtracji}\PY{l+s+s2}{\PYZdq{}}\PY{p}{,} \PY{n}{fontsize}\PY{o}{=}\PY{l+m+mi}{16}\PY{p}{)}

\PY{n}{axarr\PYZus{}1}\PY{p}{[}\PY{l+m+mi}{0}\PY{p}{,} \PY{l+m+mi}{0}\PY{p}{]}\PY{o}{.}\PY{n}{plot}\PY{p}{(}\PY{n}{np}\PY{o}{.}\PY{n}{sort}\PY{p}{(}\PY{n}{noise\PYZus{}after\PYZus{}filtration\PYZus{}k3\PYZus{}fc1600}\PY{p}{)}\PY{p}{,} \PY{n}{probability\PYZus{}pdf}\PY{p}{(}\PY{n}{noise\PYZus{}after\PYZus{}filtration\PYZus{}k3\PYZus{}fc1600}\PY{p}{)}\PY{p}{)}
\PY{n}{axarr\PYZus{}1}\PY{p}{[}\PY{l+m+mi}{0}\PY{p}{,} \PY{l+m+mi}{0}\PY{p}{]}\PY{o}{.}\PY{n}{set\PYZus{}title}\PY{p}{(}\PY{l+s+s1}{\PYZsq{}}\PY{l+s+s1}{k=10\PYZca{}(3)}\PY{l+s+s1}{\PYZsq{}}\PY{p}{)}
\PY{n}{axarr\PYZus{}1}\PY{p}{[}\PY{l+m+mi}{0}\PY{p}{,} \PY{l+m+mi}{1}\PY{p}{]}\PY{o}{.}\PY{n}{plot}\PY{p}{(}\PY{n}{np}\PY{o}{.}\PY{n}{sort}\PY{p}{(}\PY{n}{noise\PYZus{}after\PYZus{}filtration\PYZus{}k4\PYZus{}fc1600}\PY{p}{)}\PY{p}{,} \PY{n}{probability\PYZus{}pdf}\PY{p}{(}\PY{n}{noise\PYZus{}after\PYZus{}filtration\PYZus{}k4\PYZus{}fc1600}\PY{p}{)}\PY{p}{)}
\PY{n}{axarr\PYZus{}1}\PY{p}{[}\PY{l+m+mi}{0}\PY{p}{,} \PY{l+m+mi}{1}\PY{p}{]}\PY{o}{.}\PY{n}{set\PYZus{}title}\PY{p}{(}\PY{l+s+s1}{\PYZsq{}}\PY{l+s+s1}{k=10\PYZca{}(4)}\PY{l+s+s1}{\PYZsq{}}\PY{p}{)}
\PY{n}{axarr\PYZus{}1}\PY{p}{[}\PY{l+m+mi}{0}\PY{p}{,} \PY{l+m+mi}{2}\PY{p}{]}\PY{o}{.}\PY{n}{plot}\PY{p}{(}\PY{n}{np}\PY{o}{.}\PY{n}{sort}\PY{p}{(}\PY{n}{noise\PYZus{}after\PYZus{}filtration\PYZus{}k5\PYZus{}fc1600}\PY{p}{)}\PY{p}{,} \PY{n}{probability\PYZus{}pdf}\PY{p}{(}\PY{n}{noise\PYZus{}after\PYZus{}filtration\PYZus{}k5\PYZus{}fc1600}\PY{p}{)}\PY{p}{)}
\PY{n}{axarr\PYZus{}1}\PY{p}{[}\PY{l+m+mi}{0}\PY{p}{,} \PY{l+m+mi}{2}\PY{p}{]}\PY{o}{.}\PY{n}{set\PYZus{}title}\PY{p}{(}\PY{l+s+s1}{\PYZsq{}}\PY{l+s+s1}{k=10\PYZca{}(5)}\PY{l+s+s1}{\PYZsq{}}\PY{p}{)}
\PY{n}{axarr\PYZus{}1}\PY{p}{[}\PY{l+m+mi}{1}\PY{p}{,} \PY{l+m+mi}{0}\PY{p}{]}\PY{o}{.}\PY{n}{plot}\PY{p}{(}\PY{n}{np}\PY{o}{.}\PY{n}{sort}\PY{p}{(}\PY{n}{noise\PYZus{}after\PYZus{}filtration\PYZus{}k6\PYZus{}fc1600}\PY{p}{)}\PY{p}{,} \PY{n}{probability\PYZus{}pdf}\PY{p}{(}\PY{n}{noise\PYZus{}after\PYZus{}filtration\PYZus{}k6\PYZus{}fc1600}\PY{p}{)}\PY{p}{)}
\PY{n}{axarr\PYZus{}1}\PY{p}{[}\PY{l+m+mi}{1}\PY{p}{,} \PY{l+m+mi}{0}\PY{p}{]}\PY{o}{.}\PY{n}{set\PYZus{}title}\PY{p}{(}\PY{l+s+s1}{\PYZsq{}}\PY{l+s+s1}{k=10\PYZca{}(6)}\PY{l+s+s1}{\PYZsq{}}\PY{p}{)}
\PY{n}{axarr\PYZus{}1}\PY{p}{[}\PY{l+m+mi}{1}\PY{p}{,} \PY{l+m+mi}{1}\PY{p}{]}\PY{o}{.}\PY{n}{plot}\PY{p}{(}\PY{n}{np}\PY{o}{.}\PY{n}{sort}\PY{p}{(}\PY{n}{noise\PYZus{}after\PYZus{}filtration\PYZus{}k7\PYZus{}fc1600}\PY{p}{)}\PY{p}{,} \PY{n}{probability\PYZus{}pdf}\PY{p}{(}\PY{n}{noise\PYZus{}after\PYZus{}filtration\PYZus{}k7\PYZus{}fc1600}\PY{p}{)}\PY{p}{)}
\PY{n}{axarr\PYZus{}1}\PY{p}{[}\PY{l+m+mi}{1}\PY{p}{,} \PY{l+m+mi}{1}\PY{p}{]}\PY{o}{.}\PY{n}{set\PYZus{}title}\PY{p}{(}\PY{l+s+s1}{\PYZsq{}}\PY{l+s+s1}{k=10\PYZca{}(7)}\PY{l+s+s1}{\PYZsq{}}\PY{p}{)}
\PY{n}{axarr\PYZus{}1}\PY{p}{[}\PY{l+m+mi}{1}\PY{p}{,} \PY{l+m+mi}{2}\PY{p}{]}\PY{o}{.}\PY{n}{set\PYZus{}visible}\PY{p}{(}\PY{k+kc}{False}\PY{p}{)}

\PY{c+c1}{\PYZsh{} Tight layout often produces nice results}
\PY{c+c1}{\PYZsh{} but requires the title to be spaced accordingly}
\PY{n}{fig\PYZus{}1}\PY{o}{.}\PY{n}{tight\PYZus{}layout}\PY{p}{(}\PY{p}{)}
\PY{n}{fig\PYZus{}1}\PY{o}{.}\PY{n}{subplots\PYZus{}adjust}\PY{p}{(}\PY{n}{top}\PY{o}{=}\PY{l+m+mf}{0.92}\PY{p}{)}

\PY{n}{plt}\PY{o}{.}\PY{n}{show}\PY{p}{(}\PY{p}{)}
\end{Verbatim}
\end{tcolorbox}

    \begin{center}
    \adjustimage{max size={0.9\linewidth}{0.9\paperheight}}{main_files/main_86_0.png}
    \end{center}
    { \hspace*{\fill} \\}
    
    \begin{center}
    \adjustimage{max size={0.9\linewidth}{0.9\paperheight}}{main_files/main_86_1.png}
    \end{center}
    { \hspace*{\fill} \\}
    
    \hypertarget{obliczenie-dystrybuanty}{%
\subsubsection{Obliczenie dystrybuanty}\label{obliczenie-dystrybuanty}}

\[ F(x) = \frac{1}{2}\left[1 + \operatorname{erf}\left(\frac{x-\mu}{\sigma\sqrt{2}}\right)\right] \]

Dystrybuanta sygnału wyjściowego również różni się od dystrybuanty szumu
białego. Po zastosowaniu filtru dolnoprzepustowego FIR, dystrybuanta
staje się bardziej ``płaska'' i gładka, co wskazuje na redukcję wysokich
częstotliwości i wygładzenie sygnału.

    \begin{tcolorbox}[breakable, size=fbox, boxrule=1pt, pad at break*=1mm,colback=cellbackground, colframe=cellborder]
\prompt{In}{incolor}{90}{\boxspacing}
\begin{Verbatim}[commandchars=\\\{\}]
\PY{c+c1}{\PYZsh{} Tight layout often produces nice results}
\PY{c+c1}{\PYZsh{} but requires the title to be spaced accordingly}
\PY{n}{fig\PYZus{}1}\PY{o}{.}\PY{n}{tight\PYZus{}layout}\PY{p}{(}\PY{p}{)}
\PY{n}{fig\PYZus{}1}\PY{o}{.}\PY{n}{subplots\PYZus{}adjust}\PY{p}{(}\PY{n}{top}\PY{o}{=}\PY{l+m+mf}{0.92}\PY{p}{)}

\PY{n}{fig\PYZus{}2}\PY{p}{,} \PY{n}{axarr\PYZus{}2} \PY{o}{=} \PY{n}{plt}\PY{o}{.}\PY{n}{subplots}\PY{p}{(}\PY{l+m+mi}{2}\PY{p}{,} \PY{l+m+mi}{3}\PY{p}{)}
\PY{n}{fig\PYZus{}2}\PY{o}{.}\PY{n}{set\PYZus{}figheight}\PY{p}{(}\PY{l+m+mi}{12}\PY{p}{)}
\PY{n}{fig\PYZus{}2}\PY{o}{.}\PY{n}{set\PYZus{}figwidth}\PY{p}{(}\PY{l+m+mi}{20}\PY{p}{)}
\PY{n}{fig\PYZus{}2}\PY{o}{.}\PY{n}{suptitle}\PY{p}{(}\PY{l+s+s2}{\PYZdq{}}\PY{l+s+s2}{Dystrybuanta dla różnego k}\PY{l+s+s2}{\PYZdq{}}\PY{p}{,} \PY{n}{fontsize}\PY{o}{=}\PY{l+m+mi}{16}\PY{p}{)}

\PY{n}{axarr\PYZus{}2}\PY{p}{[}\PY{l+m+mi}{0}\PY{p}{,} \PY{l+m+mi}{0}\PY{p}{]}\PY{o}{.}\PY{n}{plot}\PY{p}{(}\PY{n}{np}\PY{o}{.}\PY{n}{sort}\PY{p}{(}\PY{n}{samples\PYZus{}k\PYZus{}3}\PY{p}{)}\PY{p}{,} \PY{n}{norm\PYZus{}cdf}\PY{p}{(}\PY{n}{samples\PYZus{}k\PYZus{}3}\PY{p}{)}\PY{p}{)}
\PY{n}{axarr\PYZus{}2}\PY{p}{[}\PY{l+m+mi}{0}\PY{p}{,} \PY{l+m+mi}{0}\PY{p}{]}\PY{o}{.}\PY{n}{set\PYZus{}title}\PY{p}{(}\PY{l+s+s1}{\PYZsq{}}\PY{l+s+s1}{k=10\PYZca{}(3)}\PY{l+s+s1}{\PYZsq{}}\PY{p}{)}
\PY{n}{axarr\PYZus{}2}\PY{p}{[}\PY{l+m+mi}{0}\PY{p}{,} \PY{l+m+mi}{1}\PY{p}{]}\PY{o}{.}\PY{n}{plot}\PY{p}{(}\PY{n}{np}\PY{o}{.}\PY{n}{sort}\PY{p}{(}\PY{n}{samples\PYZus{}k\PYZus{}4}\PY{p}{)}\PY{p}{,} \PY{n}{norm\PYZus{}cdf}\PY{p}{(}\PY{n}{samples\PYZus{}k\PYZus{}4}\PY{p}{)}\PY{p}{)}
\PY{n}{axarr\PYZus{}2}\PY{p}{[}\PY{l+m+mi}{0}\PY{p}{,} \PY{l+m+mi}{1}\PY{p}{]}\PY{o}{.}\PY{n}{set\PYZus{}title}\PY{p}{(}\PY{l+s+s1}{\PYZsq{}}\PY{l+s+s1}{k=10\PYZca{}(4)}\PY{l+s+s1}{\PYZsq{}}\PY{p}{)}
\PY{n}{axarr\PYZus{}2}\PY{p}{[}\PY{l+m+mi}{0}\PY{p}{,} \PY{l+m+mi}{2}\PY{p}{]}\PY{o}{.}\PY{n}{plot}\PY{p}{(}\PY{n}{np}\PY{o}{.}\PY{n}{sort}\PY{p}{(}\PY{n}{samples\PYZus{}k\PYZus{}5}\PY{p}{)}\PY{p}{,} \PY{n}{norm\PYZus{}cdf}\PY{p}{(}\PY{n}{samples\PYZus{}k\PYZus{}5}\PY{p}{)}\PY{p}{)}
\PY{n}{axarr\PYZus{}2}\PY{p}{[}\PY{l+m+mi}{0}\PY{p}{,} \PY{l+m+mi}{2}\PY{p}{]}\PY{o}{.}\PY{n}{set\PYZus{}title}\PY{p}{(}\PY{l+s+s1}{\PYZsq{}}\PY{l+s+s1}{k=10\PYZca{}(5)}\PY{l+s+s1}{\PYZsq{}}\PY{p}{)}
\PY{n}{axarr\PYZus{}2}\PY{p}{[}\PY{l+m+mi}{1}\PY{p}{,} \PY{l+m+mi}{0}\PY{p}{]}\PY{o}{.}\PY{n}{plot}\PY{p}{(}\PY{n}{np}\PY{o}{.}\PY{n}{sort}\PY{p}{(}\PY{n}{samples\PYZus{}k\PYZus{}6}\PY{p}{)}\PY{p}{,} \PY{n}{norm\PYZus{}cdf}\PY{p}{(}\PY{n}{samples\PYZus{}k\PYZus{}6}\PY{p}{)}\PY{p}{)}
\PY{n}{axarr\PYZus{}2}\PY{p}{[}\PY{l+m+mi}{1}\PY{p}{,} \PY{l+m+mi}{0}\PY{p}{]}\PY{o}{.}\PY{n}{set\PYZus{}title}\PY{p}{(}\PY{l+s+s1}{\PYZsq{}}\PY{l+s+s1}{k=10\PYZca{}(6)}\PY{l+s+s1}{\PYZsq{}}\PY{p}{)}
\PY{n}{axarr\PYZus{}2}\PY{p}{[}\PY{l+m+mi}{1}\PY{p}{,} \PY{l+m+mi}{1}\PY{p}{]}\PY{o}{.}\PY{n}{plot}\PY{p}{(}\PY{n}{np}\PY{o}{.}\PY{n}{sort}\PY{p}{(}\PY{n}{samples\PYZus{}k\PYZus{}7}\PY{p}{)}\PY{p}{,} \PY{n}{norm\PYZus{}cdf}\PY{p}{(}\PY{n}{samples\PYZus{}k\PYZus{}7}\PY{p}{)}\PY{p}{)}
\PY{n}{axarr\PYZus{}2}\PY{p}{[}\PY{l+m+mi}{1}\PY{p}{,} \PY{l+m+mi}{1}\PY{p}{]}\PY{o}{.}\PY{n}{set\PYZus{}title}\PY{p}{(}\PY{l+s+s1}{\PYZsq{}}\PY{l+s+s1}{k=10\PYZca{}(7)}\PY{l+s+s1}{\PYZsq{}}\PY{p}{)}
\PY{n}{axarr\PYZus{}2}\PY{p}{[}\PY{l+m+mi}{1}\PY{p}{,} \PY{l+m+mi}{2}\PY{p}{]}\PY{o}{.}\PY{n}{set\PYZus{}visible}\PY{p}{(}\PY{k+kc}{False}\PY{p}{)}

\PY{c+c1}{\PYZsh{} Tight layout often produces nice results}
\PY{c+c1}{\PYZsh{} but requires the title to be spaced accordingly}
\PY{n}{fig\PYZus{}2}\PY{o}{.}\PY{n}{tight\PYZus{}layout}\PY{p}{(}\PY{p}{)}
\PY{n}{fig\PYZus{}2}\PY{o}{.}\PY{n}{subplots\PYZus{}adjust}\PY{p}{(}\PY{n}{top}\PY{o}{=}\PY{l+m+mf}{0.92}\PY{p}{)}

\PY{n}{fig\PYZus{}1}\PY{p}{,} \PY{n}{axarr\PYZus{}1} \PY{o}{=} \PY{n}{plt}\PY{o}{.}\PY{n}{subplots}\PY{p}{(}\PY{l+m+mi}{2}\PY{p}{,} \PY{l+m+mi}{3}\PY{p}{)}
\PY{n}{fig\PYZus{}1}\PY{o}{.}\PY{n}{set\PYZus{}figheight}\PY{p}{(}\PY{l+m+mi}{12}\PY{p}{)}
\PY{n}{fig\PYZus{}1}\PY{o}{.}\PY{n}{set\PYZus{}figwidth}\PY{p}{(}\PY{l+m+mi}{20}\PY{p}{)}
\PY{n}{fig\PYZus{}1}\PY{o}{.}\PY{n}{suptitle}\PY{p}{(}\PY{l+s+s2}{\PYZdq{}}\PY{l+s+s2}{Dystrybuanta dla różnego k po filtracji}\PY{l+s+s2}{\PYZdq{}}\PY{p}{,} \PY{n}{fontsize}\PY{o}{=}\PY{l+m+mi}{16}\PY{p}{)}

\PY{n}{axarr\PYZus{}1}\PY{p}{[}\PY{l+m+mi}{0}\PY{p}{,} \PY{l+m+mi}{0}\PY{p}{]}\PY{o}{.}\PY{n}{plot}\PY{p}{(}\PY{n}{np}\PY{o}{.}\PY{n}{sort}\PY{p}{(}\PY{n}{noise\PYZus{}after\PYZus{}filtration\PYZus{}k3\PYZus{}fc1600}\PY{p}{)}\PY{p}{,} \PY{n}{norm\PYZus{}cdf}\PY{p}{(}\PY{n}{noise\PYZus{}after\PYZus{}filtration\PYZus{}k3\PYZus{}fc1600}\PY{p}{)}\PY{p}{)}
\PY{n}{axarr\PYZus{}1}\PY{p}{[}\PY{l+m+mi}{0}\PY{p}{,} \PY{l+m+mi}{0}\PY{p}{]}\PY{o}{.}\PY{n}{set\PYZus{}title}\PY{p}{(}\PY{l+s+s1}{\PYZsq{}}\PY{l+s+s1}{k=10\PYZca{}(3)}\PY{l+s+s1}{\PYZsq{}}\PY{p}{)}
\PY{n}{axarr\PYZus{}1}\PY{p}{[}\PY{l+m+mi}{0}\PY{p}{,} \PY{l+m+mi}{1}\PY{p}{]}\PY{o}{.}\PY{n}{plot}\PY{p}{(}\PY{n}{np}\PY{o}{.}\PY{n}{sort}\PY{p}{(}\PY{n}{noise\PYZus{}after\PYZus{}filtration\PYZus{}k4\PYZus{}fc1600}\PY{p}{)}\PY{p}{,} \PY{n}{norm\PYZus{}cdf}\PY{p}{(}\PY{n}{noise\PYZus{}after\PYZus{}filtration\PYZus{}k4\PYZus{}fc1600}\PY{p}{)}\PY{p}{)}
\PY{n}{axarr\PYZus{}1}\PY{p}{[}\PY{l+m+mi}{0}\PY{p}{,} \PY{l+m+mi}{1}\PY{p}{]}\PY{o}{.}\PY{n}{set\PYZus{}title}\PY{p}{(}\PY{l+s+s1}{\PYZsq{}}\PY{l+s+s1}{k=10\PYZca{}(4)}\PY{l+s+s1}{\PYZsq{}}\PY{p}{)}
\PY{n}{axarr\PYZus{}1}\PY{p}{[}\PY{l+m+mi}{0}\PY{p}{,} \PY{l+m+mi}{2}\PY{p}{]}\PY{o}{.}\PY{n}{plot}\PY{p}{(}\PY{n}{np}\PY{o}{.}\PY{n}{sort}\PY{p}{(}\PY{n}{noise\PYZus{}after\PYZus{}filtration\PYZus{}k5\PYZus{}fc1600}\PY{p}{)}\PY{p}{,} \PY{n}{norm\PYZus{}cdf}\PY{p}{(}\PY{n}{noise\PYZus{}after\PYZus{}filtration\PYZus{}k5\PYZus{}fc1600}\PY{p}{)}\PY{p}{)}
\PY{n}{axarr\PYZus{}1}\PY{p}{[}\PY{l+m+mi}{0}\PY{p}{,} \PY{l+m+mi}{2}\PY{p}{]}\PY{o}{.}\PY{n}{set\PYZus{}title}\PY{p}{(}\PY{l+s+s1}{\PYZsq{}}\PY{l+s+s1}{k=10\PYZca{}(5)}\PY{l+s+s1}{\PYZsq{}}\PY{p}{)}
\PY{n}{axarr\PYZus{}1}\PY{p}{[}\PY{l+m+mi}{1}\PY{p}{,} \PY{l+m+mi}{0}\PY{p}{]}\PY{o}{.}\PY{n}{plot}\PY{p}{(}\PY{n}{np}\PY{o}{.}\PY{n}{sort}\PY{p}{(}\PY{n}{noise\PYZus{}after\PYZus{}filtration\PYZus{}k6\PYZus{}fc1600}\PY{p}{)}\PY{p}{,} \PY{n}{norm\PYZus{}cdf}\PY{p}{(}\PY{n}{noise\PYZus{}after\PYZus{}filtration\PYZus{}k6\PYZus{}fc1600}\PY{p}{)}\PY{p}{)}
\PY{n}{axarr\PYZus{}1}\PY{p}{[}\PY{l+m+mi}{1}\PY{p}{,} \PY{l+m+mi}{0}\PY{p}{]}\PY{o}{.}\PY{n}{set\PYZus{}title}\PY{p}{(}\PY{l+s+s1}{\PYZsq{}}\PY{l+s+s1}{k=10\PYZca{}(6)}\PY{l+s+s1}{\PYZsq{}}\PY{p}{)}
\PY{n}{axarr\PYZus{}1}\PY{p}{[}\PY{l+m+mi}{1}\PY{p}{,} \PY{l+m+mi}{1}\PY{p}{]}\PY{o}{.}\PY{n}{plot}\PY{p}{(}\PY{n}{np}\PY{o}{.}\PY{n}{sort}\PY{p}{(}\PY{n}{noise\PYZus{}after\PYZus{}filtration\PYZus{}k7\PYZus{}fc1600}\PY{p}{)}\PY{p}{,} \PY{n}{norm\PYZus{}cdf}\PY{p}{(}\PY{n}{noise\PYZus{}after\PYZus{}filtration\PYZus{}k7\PYZus{}fc1600}\PY{p}{)}\PY{p}{)}
\PY{n}{axarr\PYZus{}1}\PY{p}{[}\PY{l+m+mi}{1}\PY{p}{,} \PY{l+m+mi}{1}\PY{p}{]}\PY{o}{.}\PY{n}{set\PYZus{}title}\PY{p}{(}\PY{l+s+s1}{\PYZsq{}}\PY{l+s+s1}{k=10\PYZca{}(7)}\PY{l+s+s1}{\PYZsq{}}\PY{p}{)}
\PY{n}{axarr\PYZus{}1}\PY{p}{[}\PY{l+m+mi}{1}\PY{p}{,} \PY{l+m+mi}{2}\PY{p}{]}\PY{o}{.}\PY{n}{set\PYZus{}visible}\PY{p}{(}\PY{k+kc}{False}\PY{p}{)}

\PY{n}{plt}\PY{o}{.}\PY{n}{show}\PY{p}{(}\PY{p}{)}
\end{Verbatim}
\end{tcolorbox}

    \begin{center}
    \adjustimage{max size={0.9\linewidth}{0.9\paperheight}}{main_files/main_88_0.png}
    \end{center}
    { \hspace*{\fill} \\}
    
    \begin{center}
    \adjustimage{max size={0.9\linewidth}{0.9\paperheight}}{main_files/main_88_1.png}
    \end{center}
    { \hspace*{\fill} \\}
    
    \hypertarget{obliczenie-wartoux15bci-oczekiwanej}{%
\subsubsection{Obliczenie wartości
oczekiwanej}\label{obliczenie-wartoux15bci-oczekiwanej}}

\[ E(X) = \int_{-\infty}^{\infty} x f(x) dx \]

Wartość oczekiwana sygnału wyjściowego po filtracji jest zbliżona do
wartości oczekiwanej szumu białego, jednak może ulec nieznacznym
przesunięciom ze względu na charakterystykę filtru.

    \begin{tcolorbox}[breakable, size=fbox, boxrule=1pt, pad at break*=1mm,colback=cellbackground, colframe=cellborder]
\prompt{In}{incolor}{91}{\boxspacing}
\begin{Verbatim}[commandchars=\\\{\}]
\PY{c+c1}{\PYZsh{}Dla k=10\PYZca{}(3)}
\PY{n}{expected\PYZus{}value\PYZus{}k1\PYZus{}after\PYZus{}filtration} \PY{o}{=} \PY{n}{describe}\PY{p}{(}\PY{n}{noise\PYZus{}after\PYZus{}filtration\PYZus{}k3\PYZus{}fc1600}\PY{p}{)}\PY{o}{.}\PY{n}{mean}
\PY{c+c1}{\PYZsh{}Dla k=10\PYZca{}(4)}
\PY{n}{expected\PYZus{}value\PYZus{}k2\PYZus{}after\PYZus{}filtration} \PY{o}{=} \PY{n}{describe}\PY{p}{(}\PY{n}{noise\PYZus{}after\PYZus{}filtration\PYZus{}k4\PYZus{}fc1600}\PY{p}{)}\PY{o}{.}\PY{n}{mean}
\PY{c+c1}{\PYZsh{}Dla k=10\PYZca{}(5)}
\PY{n}{expected\PYZus{}value\PYZus{}k3\PYZus{}after\PYZus{}filtration} \PY{o}{=} \PY{n}{describe}\PY{p}{(}\PY{n}{noise\PYZus{}after\PYZus{}filtration\PYZus{}k5\PYZus{}fc1600}\PY{p}{)}\PY{o}{.}\PY{n}{mean}
\PY{c+c1}{\PYZsh{}Dla k=10\PYZca{}(6)}
\PY{n}{expected\PYZus{}value\PYZus{}k4\PYZus{}after\PYZus{}filtration} \PY{o}{=} \PY{n}{describe}\PY{p}{(}\PY{n}{noise\PYZus{}after\PYZus{}filtration\PYZus{}k6\PYZus{}fc1600}\PY{p}{)}\PY{o}{.}\PY{n}{mean}
\PY{c+c1}{\PYZsh{}Dla k=10\PYZca{}(7)}
\PY{n}{expected\PYZus{}value\PYZus{}k5\PYZus{}after\PYZus{}filtration} \PY{o}{=} \PY{n}{describe}\PY{p}{(}\PY{n}{noise\PYZus{}after\PYZus{}filtration\PYZus{}k7\PYZus{}fc1600}\PY{p}{)}\PY{o}{.}\PY{n}{mean}
\end{Verbatim}
\end{tcolorbox}

    \hypertarget{obliczenie-wariancji}{%
\subsubsection{Obliczenie wariancji}\label{obliczenie-wariancji}}

\[ \operatorname{Var}(X) = E\left[(X - E(X))^2\right] = \int_{-\infty}^{\infty} (x - E(X))^2 f(x) dx \]

Wariancja sygnału wyjściowego jest mniejsza niż wariancja szumu białego.
Filtr dolnoprzepustowy FIR redukuje wysokie częstotliwości, co prowadzi
do zmniejszenia zmienności sygnału i zmniejszenia wariancji.

    \begin{tcolorbox}[breakable, size=fbox, boxrule=1pt, pad at break*=1mm,colback=cellbackground, colframe=cellborder]
\prompt{In}{incolor}{92}{\boxspacing}
\begin{Verbatim}[commandchars=\\\{\}]
\PY{c+c1}{\PYZsh{}Dla k=10\PYZca{}(3)}
\PY{n}{variance\PYZus{}k1\PYZus{}after\PYZus{}filtration} \PY{o}{=} \PY{n}{describe}\PY{p}{(}\PY{n}{noise\PYZus{}after\PYZus{}filtration\PYZus{}k3\PYZus{}fc1600}\PY{p}{)}\PY{o}{.}\PY{n}{variance}
\PY{c+c1}{\PYZsh{}Dla k=10\PYZca{}(4)}
\PY{n}{variance\PYZus{}k2\PYZus{}after\PYZus{}filtration} \PY{o}{=} \PY{n}{describe}\PY{p}{(}\PY{n}{noise\PYZus{}after\PYZus{}filtration\PYZus{}k4\PYZus{}fc1600}\PY{p}{)}\PY{o}{.}\PY{n}{variance}
\PY{c+c1}{\PYZsh{}Dla k=10\PYZca{}(5)}
\PY{n}{variance\PYZus{}k3\PYZus{}after\PYZus{}filtration} \PY{o}{=} \PY{n}{describe}\PY{p}{(}\PY{n}{noise\PYZus{}after\PYZus{}filtration\PYZus{}k5\PYZus{}fc1600}\PY{p}{)}\PY{o}{.}\PY{n}{variance}
\PY{c+c1}{\PYZsh{}Dla k=10\PYZca{}(6)}
\PY{n}{variance\PYZus{}k4\PYZus{}after\PYZus{}filtration} \PY{o}{=} \PY{n}{describe}\PY{p}{(}\PY{n}{noise\PYZus{}after\PYZus{}filtration\PYZus{}k6\PYZus{}fc1600}\PY{p}{)}\PY{o}{.}\PY{n}{variance}
\PY{c+c1}{\PYZsh{}Dla k=10\PYZca{}(7)}
\PY{n}{variance\PYZus{}k5\PYZus{}after\PYZus{}filtration} \PY{o}{=} \PY{n}{describe}\PY{p}{(}\PY{n}{noise\PYZus{}after\PYZus{}filtration\PYZus{}k7\PYZus{}fc1600}\PY{p}{)}\PY{o}{.}\PY{n}{variance}
\end{Verbatim}
\end{tcolorbox}

    Podsumowanie obliczeń wartości oczekiwanej oraz kowariancji względem
różnych k

    \begin{tcolorbox}[breakable, size=fbox, boxrule=1pt, pad at break*=1mm,colback=cellbackground, colframe=cellborder]
\prompt{In}{incolor}{93}{\boxspacing}
\begin{Verbatim}[commandchars=\\\{\}]
\PY{n+nb}{dict} \PY{o}{=} \PY{p}{\PYZob{}}\PY{l+s+s1}{\PYZsq{}}\PY{l+s+s1}{k}\PY{l+s+s1}{\PYZsq{}} \PY{p}{:} \PY{p}{[}\PY{l+s+s1}{\PYZsq{}}\PY{l+s+s1}{k\PYZca{}(3)}\PY{l+s+s1}{\PYZsq{}}\PY{p}{,} \PY{l+s+s1}{\PYZsq{}}\PY{l+s+s1}{k\PYZca{}(4)}\PY{l+s+s1}{\PYZsq{}}\PY{p}{,} \PY{l+s+s1}{\PYZsq{}}\PY{l+s+s1}{k\PYZca{}(5)}\PY{l+s+s1}{\PYZsq{}}\PY{p}{,} \PY{l+s+s1}{\PYZsq{}}\PY{l+s+s1}{k\PYZca{}(6)}\PY{l+s+s1}{\PYZsq{}}\PY{p}{,} \PY{l+s+s1}{\PYZsq{}}\PY{l+s+s1}{k\PYZca{}(7)}\PY{l+s+s1}{\PYZsq{}}\PY{p}{]}\PY{p}{,}
        \PY{l+s+s1}{\PYZsq{}}\PY{l+s+s1}{wartość oczekiwana}\PY{l+s+s1}{\PYZsq{}} \PY{p}{:} \PY{p}{[}\PY{n}{expected\PYZus{}value\PYZus{}k1}\PY{p}{,} \PY{n}{expected\PYZus{}value\PYZus{}k2}\PY{p}{,} \PY{n}{expected\PYZus{}value\PYZus{}k3}\PY{p}{,} \PY{n}{expected\PYZus{}value\PYZus{}k4}\PY{p}{,} \PY{n}{expected\PYZus{}value\PYZus{}k5}\PY{p}{]}\PY{p}{,}
        \PY{l+s+s1}{\PYZsq{}}\PY{l+s+s1}{wariancja}\PY{l+s+s1}{\PYZsq{}} \PY{p}{:} \PY{p}{[}\PY{n}{variance\PYZus{}k1}\PY{p}{,} \PY{n}{variance\PYZus{}k2}\PY{p}{,} \PY{n}{variance\PYZus{}k3}\PY{p}{,} \PY{n}{variance\PYZus{}k4}\PY{p}{,} \PY{n}{variance\PYZus{}k5}\PY{p}{]}\PY{p}{,}
        \PY{l+s+s1}{\PYZsq{}}\PY{l+s+s1}{wartość oczekiwana po filtracji}\PY{l+s+s1}{\PYZsq{}} \PY{p}{:} \PY{p}{[}\PY{n}{expected\PYZus{}value\PYZus{}k1\PYZus{}after\PYZus{}filtration}\PY{p}{,} \PY{n}{expected\PYZus{}value\PYZus{}k2\PYZus{}after\PYZus{}filtration}\PY{p}{,} \PY{n}{expected\PYZus{}value\PYZus{}k3\PYZus{}after\PYZus{}filtration}\PY{p}{,} \PY{n}{expected\PYZus{}value\PYZus{}k4\PYZus{}after\PYZus{}filtration}\PY{p}{,} \PY{n}{expected\PYZus{}value\PYZus{}k5\PYZus{}after\PYZus{}filtration}\PY{p}{]}\PY{p}{,}
        \PY{l+s+s1}{\PYZsq{}}\PY{l+s+s1}{wariancja po filtracji}\PY{l+s+s1}{\PYZsq{}} \PY{p}{:} \PY{p}{[}\PY{n}{variance\PYZus{}k1\PYZus{}after\PYZus{}filtration}\PY{p}{,} \PY{n}{variance\PYZus{}k2\PYZus{}after\PYZus{}filtration}\PY{p}{,} \PY{n}{variance\PYZus{}k3\PYZus{}after\PYZus{}filtration}\PY{p}{,} \PY{n}{variance\PYZus{}k4\PYZus{}after\PYZus{}filtration}\PY{p}{,} \PY{n}{variance\PYZus{}k5\PYZus{}after\PYZus{}filtration}\PY{p}{]}\PY{p}{\PYZcb{}}

\PY{n}{df} \PY{o}{=} \PY{n}{pd}\PY{o}{.}\PY{n}{DataFrame}\PY{p}{(}\PY{n+nb}{dict}\PY{p}{)}

\PY{n}{df}\PY{o}{.}\PY{n}{style}
\end{Verbatim}
\end{tcolorbox}

            \begin{tcolorbox}[breakable, size=fbox, boxrule=.5pt, pad at break*=1mm, opacityfill=0]
\prompt{Out}{outcolor}{93}{\boxspacing}
\begin{Verbatim}[commandchars=\\\{\}]
<pandas.io.formats.style.Styler at 0x1e1841c61a0>
\end{Verbatim}
\end{tcolorbox}

\begin{tabular}{llrrrr}
\toprule
{} &      k &  wartość oczekiwana &  wariancja &  wartość oczekiwana po filtracji &  wariancja po filtracji \\
\midrule
0 &  k\textasciicircum (3) &            5.004410 &   0.010628 &                         4.879487 &                0.614888 \\
1 &  k\textasciicircum (4) &            4.999147 &   0.009954 &                         4.986632 &                0.068301 \\
2 &  k\textasciicircum (5) &            5.000310 &   0.010048 &                         4.999056 &                0.012505 \\
3 &  k\textasciicircum (6) &            5.000073 &   0.009984 &                         4.999949 &                0.006865 \\
4 &  k\textasciicircum (7) &            4.999993 &   0.009997 &                         4.999980 &                0.006316 \\
\bottomrule
\end{tabular}

        
    \hypertarget{obliczenie-funkcji-kowariancyjnej}{%
\subsubsection{Obliczenie funkcji
kowariancyjnej}\label{obliczenie-funkcji-kowariancyjnej}}

Macierz kowariancji sygnału wyjściowego może różnić się od macierzy
kowariancji szumu białego. Filtracja dolnoprzepustowa wpływa na
korelację między próbkami sygnału, co może prowadzić do zmiany
wzajemnego wpływu próbek na siebie.

\[ \operatorname{Cov}(X_i, X_j) = \begin{cases} \sigma^2 & i = j \\ 0 & i \neq j \end{cases} \]

gdzie X\_i i X\_j to próbki szumu białego, a σ\^{}2 to wariancja szumu
białego.

    \begin{tcolorbox}[breakable, size=fbox, boxrule=1pt, pad at break*=1mm,colback=cellbackground, colframe=cellborder]
\prompt{In}{incolor}{94}{\boxspacing}
\begin{Verbatim}[commandchars=\\\{\}]
\PY{n}{fig}\PY{p}{,} \PY{n}{axarr} \PY{o}{=} \PY{n}{plt}\PY{o}{.}\PY{n}{subplots}\PY{p}{(}\PY{l+m+mi}{1}\PY{p}{,} \PY{l+m+mi}{2}\PY{p}{)}
\PY{n}{fig}\PY{o}{.}\PY{n}{set\PYZus{}figheight}\PY{p}{(}\PY{l+m+mi}{12}\PY{p}{)}
\PY{n}{fig}\PY{o}{.}\PY{n}{set\PYZus{}figwidth}\PY{p}{(}\PY{l+m+mi}{22}\PY{p}{)}
\PY{n}{fig}\PY{o}{.}\PY{n}{suptitle}\PY{p}{(}\PY{l+s+s2}{\PYZdq{}}\PY{l+s+s2}{Funkcja kowariancyjna dla różnego k przed filtracją}\PY{l+s+s2}{\PYZdq{}}\PY{p}{,} \PY{n}{fontsize}\PY{o}{=}\PY{l+m+mi}{16}\PY{p}{)}

\PY{n}{cov\PYZus{}k\PYZus{}3} \PY{o}{=} \PY{n}{covariance}\PY{p}{(}\PY{n}{samples\PYZus{}k\PYZus{}3}\PY{p}{,} \PY{n}{k\PYZus{}3}\PY{p}{)}
\PY{n}{cov\PYZus{}k\PYZus{}4} \PY{o}{=} \PY{n}{covariance}\PY{p}{(}\PY{n}{samples\PYZus{}k\PYZus{}4}\PY{p}{,} \PY{n}{k\PYZus{}4}\PY{p}{)}

\PY{n}{x\PYZus{}k\PYZus{}3} \PY{o}{=} \PY{n}{np}\PY{o}{.}\PY{n}{linspace}\PY{p}{(}\PY{o}{\PYZhy{}}\PY{n}{k\PYZus{}3}\PY{o}{/}\PY{o}{/}\PY{l+m+mi}{2}\PY{p}{,} \PY{n}{k\PYZus{}3}\PY{o}{/}\PY{o}{/}\PY{l+m+mi}{2}\PY{p}{,} \PY{n+nb}{len}\PY{p}{(}\PY{n}{cov\PYZus{}k\PYZus{}3}\PY{p}{)}\PY{p}{)}
\PY{n}{x\PYZus{}k\PYZus{}4} \PY{o}{=} \PY{n}{np}\PY{o}{.}\PY{n}{linspace}\PY{p}{(}\PY{o}{\PYZhy{}}\PY{n}{k\PYZus{}4}\PY{o}{/}\PY{o}{/}\PY{l+m+mi}{2}\PY{p}{,} \PY{n}{k\PYZus{}4}\PY{o}{/}\PY{o}{/}\PY{l+m+mi}{2}\PY{p}{,} \PY{n+nb}{len}\PY{p}{(}\PY{n}{cov\PYZus{}k\PYZus{}4}\PY{p}{)}\PY{p}{)}

\PY{c+c1}{\PYZsh{} Zwiększenie wartości na osi Y}
\PY{n}{y\PYZus{}max\PYZus{}k\PYZus{}3} \PY{o}{=} \PY{n}{np}\PY{o}{.}\PY{n}{max}\PY{p}{(}\PY{n}{cov\PYZus{}k\PYZus{}3}\PY{p}{)} \PY{o}{*} \PY{l+m+mf}{1.2}
\PY{n}{y\PYZus{}max\PYZus{}k\PYZus{}4} \PY{o}{=} \PY{n}{np}\PY{o}{.}\PY{n}{max}\PY{p}{(}\PY{n}{cov\PYZus{}k\PYZus{}4}\PY{p}{)} \PY{o}{*} \PY{l+m+mf}{1.2}

\PY{n}{axarr}\PY{p}{[}\PY{l+m+mi}{0}\PY{p}{]}\PY{o}{.}\PY{n}{plot}\PY{p}{(}\PY{n}{x\PYZus{}k\PYZus{}3}\PY{p}{,} \PY{n}{cov\PYZus{}k\PYZus{}3}\PY{p}{)}
\PY{n}{axarr}\PY{p}{[}\PY{l+m+mi}{0}\PY{p}{]}\PY{o}{.}\PY{n}{set\PYZus{}title}\PY{p}{(}\PY{l+s+s1}{\PYZsq{}}\PY{l+s+s1}{k=10\PYZca{}(3)}\PY{l+s+s1}{\PYZsq{}}\PY{p}{)}
\PY{n}{axarr}\PY{p}{[}\PY{l+m+mi}{0}\PY{p}{]}\PY{o}{.}\PY{n}{set\PYZus{}xlim}\PY{p}{(}\PY{o}{\PYZhy{}}\PY{n}{k\PYZus{}3}\PY{o}{/}\PY{o}{/}\PY{l+m+mi}{100}\PY{p}{,} \PY{n}{k\PYZus{}3}\PY{o}{/}\PY{o}{/}\PY{l+m+mi}{100}\PY{p}{)}
\PY{n}{axarr}\PY{p}{[}\PY{l+m+mi}{0}\PY{p}{]}\PY{o}{.}\PY{n}{set\PYZus{}ylim}\PY{p}{(}\PY{n}{np}\PY{o}{.}\PY{n}{min}\PY{p}{(}\PY{n}{cov\PYZus{}k\PYZus{}3}\PY{p}{)}\PY{p}{,} \PY{n}{y\PYZus{}max\PYZus{}k\PYZus{}3}\PY{p}{)}

\PY{n}{axarr}\PY{p}{[}\PY{l+m+mi}{1}\PY{p}{]}\PY{o}{.}\PY{n}{plot}\PY{p}{(}\PY{n}{x\PYZus{}k\PYZus{}4}\PY{p}{,} \PY{n}{cov\PYZus{}k\PYZus{}4}\PY{p}{)}
\PY{n}{axarr}\PY{p}{[}\PY{l+m+mi}{1}\PY{p}{]}\PY{o}{.}\PY{n}{set\PYZus{}title}\PY{p}{(}\PY{l+s+s1}{\PYZsq{}}\PY{l+s+s1}{k=10\PYZca{}(4)}\PY{l+s+s1}{\PYZsq{}}\PY{p}{)}
\PY{n}{axarr}\PY{p}{[}\PY{l+m+mi}{1}\PY{p}{]}\PY{o}{.}\PY{n}{set\PYZus{}xlim}\PY{p}{(}\PY{o}{\PYZhy{}}\PY{n}{k\PYZus{}4}\PY{o}{/}\PY{o}{/}\PY{l+m+mi}{500}\PY{p}{,} \PY{n}{k\PYZus{}4}\PY{o}{/}\PY{o}{/}\PY{l+m+mi}{500}\PY{p}{)}

\PY{n}{fig}\PY{o}{.}\PY{n}{tight\PYZus{}layout}\PY{p}{(}\PY{p}{)}
\PY{n}{fig}\PY{o}{.}\PY{n}{subplots\PYZus{}adjust}\PY{p}{(}\PY{n}{top}\PY{o}{=}\PY{l+m+mf}{0.92}\PY{p}{)}

\PY{n}{plt}\PY{o}{.}\PY{n}{show}\PY{p}{(}\PY{p}{)}


\PY{n}{fig}\PY{p}{,} \PY{n}{axarr} \PY{o}{=} \PY{n}{plt}\PY{o}{.}\PY{n}{subplots}\PY{p}{(}\PY{l+m+mi}{1}\PY{p}{,} \PY{l+m+mi}{2}\PY{p}{)}
\PY{n}{fig}\PY{o}{.}\PY{n}{set\PYZus{}figheight}\PY{p}{(}\PY{l+m+mi}{12}\PY{p}{)}
\PY{n}{fig}\PY{o}{.}\PY{n}{set\PYZus{}figwidth}\PY{p}{(}\PY{l+m+mi}{22}\PY{p}{)}
\PY{n}{fig}\PY{o}{.}\PY{n}{suptitle}\PY{p}{(}\PY{l+s+s2}{\PYZdq{}}\PY{l+s+s2}{Funkcja kowariancyjna dla różnego k po filtracji}\PY{l+s+s2}{\PYZdq{}}\PY{p}{,} \PY{n}{fontsize}\PY{o}{=}\PY{l+m+mi}{16}\PY{p}{)}

\PY{c+c1}{\PYZsh{} Wygładzanie funkcji kowariancji za pomocą filtru dolnoprzepustowego Butterworth}
\PY{n}{order} \PY{o}{=} \PY{l+m+mi}{11}  \PY{c+c1}{\PYZsh{} Stopień filtru}
\PY{n}{cutoff\PYZus{}freq} \PY{o}{=} \PY{l+m+mf}{0.1}  \PY{c+c1}{\PYZsh{} Częstotliwość odcięcia}
\PY{n}{b}\PY{p}{,} \PY{n}{a} \PY{o}{=} \PY{n}{butter}\PY{p}{(}\PY{n}{order}\PY{p}{,} \PY{n}{cutoff\PYZus{}freq}\PY{p}{,} \PY{n}{btype}\PY{o}{=}\PY{l+s+s1}{\PYZsq{}}\PY{l+s+s1}{low}\PY{l+s+s1}{\PYZsq{}}\PY{p}{,} \PY{n}{analog}\PY{o}{=}\PY{k+kc}{False}\PY{p}{)}

\PY{n}{cov\PYZus{}k\PYZus{}3\PYZus{}smooth} \PY{o}{=} \PY{n}{filtfilt}\PY{p}{(}\PY{n}{b}\PY{p}{,} \PY{n}{a}\PY{p}{,} \PY{n}{cov\PYZus{}k\PYZus{}3}\PY{p}{)}
\PY{n}{cov\PYZus{}k\PYZus{}4\PYZus{}smooth} \PY{o}{=} \PY{n}{filtfilt}\PY{p}{(}\PY{n}{b}\PY{p}{,} \PY{n}{a}\PY{p}{,} \PY{n}{cov\PYZus{}k\PYZus{}4}\PY{p}{)}

\PY{c+c1}{\PYZsh{} Zwiększenie wartości na osi Y}
\PY{n}{y\PYZus{}max\PYZus{}k\PYZus{}3} \PY{o}{=} \PY{n}{np}\PY{o}{.}\PY{n}{max}\PY{p}{(}\PY{n}{cov\PYZus{}k\PYZus{}3\PYZus{}smooth}\PY{p}{)} \PY{o}{*} \PY{l+m+mf}{1.2}
\PY{n}{y\PYZus{}max\PYZus{}k\PYZus{}4} \PY{o}{=} \PY{n}{np}\PY{o}{.}\PY{n}{max}\PY{p}{(}\PY{n}{cov\PYZus{}k\PYZus{}4\PYZus{}smooth}\PY{p}{)} \PY{o}{*} \PY{l+m+mf}{1.2}

\PY{n}{x\PYZus{}k\PYZus{}3} \PY{o}{=} \PY{n}{np}\PY{o}{.}\PY{n}{linspace}\PY{p}{(}\PY{o}{\PYZhy{}}\PY{n}{k\PYZus{}3}\PY{o}{/}\PY{o}{/}\PY{l+m+mi}{2}\PY{p}{,} \PY{n}{k\PYZus{}3}\PY{o}{/}\PY{o}{/}\PY{l+m+mi}{2}\PY{p}{,} \PY{n+nb}{len}\PY{p}{(}\PY{n}{cov\PYZus{}k\PYZus{}3\PYZus{}smooth}\PY{p}{)}\PY{p}{)}
\PY{n}{x\PYZus{}k\PYZus{}4} \PY{o}{=} \PY{n}{np}\PY{o}{.}\PY{n}{linspace}\PY{p}{(}\PY{o}{\PYZhy{}}\PY{n}{k\PYZus{}4}\PY{o}{/}\PY{o}{/}\PY{l+m+mi}{2}\PY{p}{,} \PY{n}{k\PYZus{}4}\PY{o}{/}\PY{o}{/}\PY{l+m+mi}{2}\PY{p}{,} \PY{n+nb}{len}\PY{p}{(}\PY{n}{cov\PYZus{}k\PYZus{}4\PYZus{}smooth}\PY{p}{)}\PY{p}{)}

\PY{n}{axarr}\PY{p}{[}\PY{l+m+mi}{0}\PY{p}{]}\PY{o}{.}\PY{n}{plot}\PY{p}{(}\PY{n}{x\PYZus{}k\PYZus{}3}\PY{p}{,} \PY{n}{cov\PYZus{}k\PYZus{}3\PYZus{}smooth}\PY{p}{)}
\PY{n}{axarr}\PY{p}{[}\PY{l+m+mi}{0}\PY{p}{]}\PY{o}{.}\PY{n}{set\PYZus{}title}\PY{p}{(}\PY{l+s+s1}{\PYZsq{}}\PY{l+s+s1}{k=10\PYZca{}(3)}\PY{l+s+s1}{\PYZsq{}}\PY{p}{)}
\PY{n}{axarr}\PY{p}{[}\PY{l+m+mi}{0}\PY{p}{]}\PY{o}{.}\PY{n}{set\PYZus{}xlim}\PY{p}{(}\PY{o}{\PYZhy{}}\PY{n}{k\PYZus{}3}\PY{o}{/}\PY{o}{/}\PY{l+m+mi}{20}\PY{p}{,} \PY{n}{k\PYZus{}3}\PY{o}{/}\PY{o}{/}\PY{l+m+mi}{20}\PY{p}{)}
\PY{n}{axarr}\PY{p}{[}\PY{l+m+mi}{0}\PY{p}{]}\PY{o}{.}\PY{n}{set\PYZus{}ylim}\PY{p}{(}\PY{n}{np}\PY{o}{.}\PY{n}{min}\PY{p}{(}\PY{n}{cov\PYZus{}k\PYZus{}3\PYZus{}smooth}\PY{p}{)}\PY{p}{,} \PY{n}{y\PYZus{}max\PYZus{}k\PYZus{}3}\PY{p}{)}

\PY{n}{axarr}\PY{p}{[}\PY{l+m+mi}{1}\PY{p}{]}\PY{o}{.}\PY{n}{plot}\PY{p}{(}\PY{n}{x\PYZus{}k\PYZus{}4}\PY{p}{,} \PY{n}{cov\PYZus{}k\PYZus{}4\PYZus{}smooth}\PY{p}{)}
\PY{n}{axarr}\PY{p}{[}\PY{l+m+mi}{1}\PY{p}{]}\PY{o}{.}\PY{n}{set\PYZus{}title}\PY{p}{(}\PY{l+s+s1}{\PYZsq{}}\PY{l+s+s1}{k=10\PYZca{}(4)}\PY{l+s+s1}{\PYZsq{}}\PY{p}{)}
\PY{n}{axarr}\PY{p}{[}\PY{l+m+mi}{1}\PY{p}{]}\PY{o}{.}\PY{n}{set\PYZus{}xlim}\PY{p}{(}\PY{o}{\PYZhy{}}\PY{n}{k\PYZus{}4}\PY{o}{/}\PY{o}{/}\PY{l+m+mi}{20}\PY{p}{,} \PY{n}{k\PYZus{}4}\PY{o}{/}\PY{o}{/}\PY{l+m+mi}{20}\PY{p}{)}
\PY{n}{axarr}\PY{p}{[}\PY{l+m+mi}{1}\PY{p}{]}\PY{o}{.}\PY{n}{set\PYZus{}ylim}\PY{p}{(}\PY{n}{np}\PY{o}{.}\PY{n}{min}\PY{p}{(}\PY{n}{cov\PYZus{}k\PYZus{}4\PYZus{}smooth}\PY{p}{)}\PY{p}{,} \PY{n}{y\PYZus{}max\PYZus{}k\PYZus{}4}\PY{p}{)}

\PY{n}{fig}\PY{o}{.}\PY{n}{tight\PYZus{}layout}\PY{p}{(}\PY{p}{)}
\PY{n}{fig}\PY{o}{.}\PY{n}{subplots\PYZus{}adjust}\PY{p}{(}\PY{n}{top}\PY{o}{=}\PY{l+m+mf}{0.92}\PY{p}{)}

\PY{n}{plt}\PY{o}{.}\PY{n}{show}\PY{p}{(}\PY{p}{)}
\end{Verbatim}
\end{tcolorbox}

    \begin{center}
    \adjustimage{max size={0.9\linewidth}{0.9\paperheight}}{main_files/main_96_0.png}
    \end{center}
    { \hspace*{\fill} \\}
    
    \begin{center}
    \adjustimage{max size={0.9\linewidth}{0.9\paperheight}}{main_files/main_96_1.png}
    \end{center}
    { \hspace*{\fill} \\}
    
    Ogólne wnioski:

Filtracja dolnoprzepustowa FIR może być użytecznym narzędziem w
przetwarzaniu sygnałów, szczególnie w przypadkach, gdzie występuje szum
o wysokich częstotliwościach lub gdy zależy nam na wygładzeniu sygnału i
redukcji zmienności. Jest to popularna metoda w dziedzinach takich jak
przetwarzanie sygnałów audio, obrazów, telekomunikacji, a także w
analizie danych i filtracji szumów.


    % Add a bibliography block to the postdoc
    
    
    
\end{document}
